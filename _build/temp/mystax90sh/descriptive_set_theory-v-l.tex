\subsection{The Axiom of Constructibility}

We can add to $\ZF$ the axiom that all sets are constructible, i.e.

\begin{quote}
$(\VL) \qquad \forall x \exists y \: (y \text{ is an ordinal } \wedge \; x \in L_y).$
\end{quote}

This axiom is usually denoted by $\VL$. We may be tempted to think that $L$ is then trivially a model of $\ZF + \VL$. But this is not at all clear, since this has to hold \textbf{relative to $L$}, i.e. $(\VL)^L$.

This means that
\begin{equation*}
\forall x \in L \: \exists y \in L \: (y \text{ is an ordinal } \wedge \; (x \in L_y)^L).
\end{equation*}

To verify this, we need to make sure that \textit{inside} $L$, $L$ ``means the same as'' $L$. This is, of course, an absoluteness property, and we therefore revisit the complexity of the formulas defining the constructible universe.

We have seen that the map $a \mapsto \mathcal{P}_{\Op{Def}}(a)$ is $\Sigma_1$. This important implications for the map $\alpha \mapsto L_\alpha$.

\begin{proposition}\label{prop-map-lalpha}The map $\alpha \mapsto L_\alpha$ is $\Delta_1$.

\end{proposition}\begin{proof}We first show that the mapping is $\Sigma_1$. The mapping is obtained by ordinal recursion over the function $a \mapsto \mathcal{P}_{\Op{Def}}(a)$.

In general, if a function $G: \V \to \V$ is $\Sigma_1$ and $F: \Ord \to \V$ is obtained by recursion from $G$, i.e. $F(\alpha) = G(F\Rest{\alpha})$, then $F$ is also $\Sigma_1$. This is because

\begin{align*}
    y= F(\alpha) \; \leftrightarrow \; \alpha \in \Ord \: \wedge \: \exists f \: & ( f \text{ function } \wedge \Op{dom}(f) = \alpha \\
        & \quad \wedge \forall \beta < \alpha (f(\beta) = G(f \Rest{\beta}) \wedge y = G(f)).
\end{align*}

Applying some of the various prefix transformations for $\Sigma_1$-formulas, and using that being an ordinal, being an function, being the domain of a function, etc., are all $\Delta_0$ properties, the above formula can be shown to be $\Sigma_1$, too.

In our case, $G$ is a function that applies either $\mathcal{P}_{\Op{Def}}$ or $\bigcup$ (both at most $\Delta_1$), depending on whether the input is a function defined on a successor ordinal or a limit ordinal (or applies the identity if neither is the case). Fortunately, this case distinction is also $\Delta_0$, and hence we obtain that $G: \alpha \mapsto L_\alpha$ is $\Sigma_1$.

Finally, as in Theorem~\ref{thm-definability-Pdef}, observe that if $G$ is a $\Sigma_1$ function with a $\Delta_1$ domain ($\Ord$), then $G$ is actually $\Delta_1$, since we have

\begin{equation}
G(x) \neq y \; \Leftrightarrow \; \exists z (G(x)=z \: \wedge \: y \neq z)
\end{equation}

so the complement of the graph of $G$ is $\Sigma_1$-definable, too.

\end{proof}\begin{corollary}\label{cor-complexity-l}\begin{itemize}
\item (\textbf{1}) $\quad$   The relations $x = L_\alpha$ and $x \in L_\alpha$ are $\Delta_1$.
\item (\textbf{2}) $\quad$   The predicate $x \in L$ is $\Sigma_1$.
\item (\textbf{3}) $\quad$   The axiom $\VL$ is $\Pi_2$.
\end{itemize}

\end{corollary}We can relativize the definition of $L$ to other classes $M$. If $M$ is is an inner model, then the development of $L$ can be done \textit{relative to $M$}. Since $M$ is a $\ZF$-model, it has to contain all the sets $L_\alpha^M$ (as we developed definability and proved facts about it \textit{inside} $\ZF$). As $M$ is transitive, the mapping $G: \alpha \to L_\alpha$ is absolute for $M$ and we obtain, for all $\alpha$,

\begin{equation}
L_\alpha^M = L_\alpha.
\end{equation}

\begin{theorem}\label{thm-l-absolute}\begin{itemize}
\item (\textbf{1}) $\quad$   If $M$ is any transitive proper class model of $\ZF$, then $L = L^M \subseteq  M$.
\item (\textbf{2}) $\quad$   $L$ is a model of $\ZF + \VL$.
\end{itemize}

\end{theorem}\begin{proof}(1) follows immediately from the fact that for such $M$, $L_\alpha^M = L_\alpha$.

(2) We have

\begin{align*}
    (\VL)^L & \leftrightarrow \: \forall x\in L \exists y \in L \: (y \text{ is an ordinal } \wedge \; x \in L_y)^L & \\
        & \leftrightarrow \: \forall x\in L \exists \alpha \: (x \in L_\alpha)^L  & \qquad \text{($\Ord \subset L$ and absolute)}\\
        & \leftrightarrow \: \forall x\in L \exists \alpha \: (x \in L_\alpha)    & \qquad \text{(by (1))} 
\end{align*}

The last statement is true since $L = \bigcup_{\alpha} L_\alpha$.

\end{proof}\subsubsection{Constructibility and the Axiom of Choice}

Every well-ordering on a transitive set $X$ can be extended to a well-ordering of $\mathcal{P}_{\Op{Def}}(X)$.

Note that every element of $\mathcal{P}_{\Op{Def}}(X)$ is determined by a pair $(\psi, \vec{a})$, where $\psi$ is a set-theoretic formula, and $\vec{a} = (a_1, \dots, a_n) \in X^{<\omega}$ is a finite sequence of parameters.

For each $z \in \mathcal{P}_{\Op{Def}}(X)$ there may exist more than one such pair (i.e.{\textbackslash}~$z$ can have more than one definition), but by well-ordering the pairs $(\psi, \vec{a})$, we can assign each $z \in \mathcal{P}_{\Op{Def}}(X)$ its \textbf{least} definition, and subsequently order $\mathcal{P}_{\Op{Def}}(X)$ by comparing least definitions. Elements already in $X$ will form an initial segment.

Such an order on the pairs $(\psi, \vec{a})$ can be obtained in a \textbf{definable way}: First use the order on $X$ to order $X^{<\omega}$ length-lexicographically, order the formulas through their Gödel numbers, and finally put

\begin{equation}
(\psi,\vec{a}) < (\varphi, \vec{b}) \quad \text{ iff } \quad \psi < \varphi \text { or } (\psi < \varphi \text { and } \vec{a} < \vec{b}).
\end{equation}

Based on this, we can order all levels of $L$ so that the following hold:

\begin{itemize}
\item (\textbf{1}) $\quad$   $<_L \Rest{V_\omega}$ is a standard well-ordering of $V_\omega$   (as for example given by a canonical isomorphism $(V_\omega, \in) \leftrightarrow (\Nat, E)$, see \cite{Ackermann:1937a})
\item (\textbf{2}) $\quad$   $<_L\Rest{L_{\alpha+1}}$ is the order on $\mathcal{P}_{\Op{Def}}(L_\alpha)$ induced by $<_L|L_\alpha$.
\item (\textbf{3}) $\quad$   $<_L\Rest{L_\lambda} = \bigcup_{\alpha < \lambda} <_L \Rest{L_\alpha}$ for a limit ordinal $\lambda > \omega$.
\end{itemize}

It is straightforward to verify that this is indeed a well-ordering on $L$. Moreover, the relation $<_L$ is $\Delta_1$. (To verify this, we have to spell out all the details of the above definition. This is a little involved, so we skip this here and refer to \cite{jech2003a}.)

\begin{theorem}\label{thm-l-ac}$\VL$ implies $\AC$

\end{theorem}Since $L$ is a model of $\ZF+\VL$, we obtain

\begin{corollary}\label{cor-con(ac)}If $\ZF$ is consistent, then $\ZF+\AC (= \ZFC)$ is consistent, too.

\end{corollary}\subsubsection{Condensation and the Continuum Hypothesis}

We now show that $\VL$ implies the Continuum Hypothesis. The proof works by showing that under $\VL$, every subset of a cardinal $\kappa$ will be constructed by stage $\kappa^+$. This is made possible by a ``\textbf{condensation}'' argument: If any subset $x$ of $\kappa$ is in $L$, then it must show up at some stage $L_\lambda$. $\kappa$ and $x$ generate an elementary substructure $M$ of $L_\lambda$ or cardinality $\kappa$. If we could show that this \textbf{$M$ itself must be an $L_\beta$}, we can use the fact that the cardinality of the $L_\alpha$ behaves ``tamely'' along the ordinals, as evidenced by the following.

\begin{proposition}\label{prop-card-lalpha}For all $\alpha \geq \omega$, $|L_{\alpha}| = |\alpha|$.

\end{proposition}\begin{proof}We know that $\alpha \subseteq L_\alpha$. Hence $|\alpha| \leq |L_\alpha|$. To show $|\alpha| \geq |L_\alpha|$, we work by induction on $\alpha$.

If $\alpha = \beta +1$, then by Proposition~\ref{prop-basics-L}(4), $|L_\alpha| = |L_\beta| = |\beta| \leq |\alpha|$.

If $\alpha$ is limit, then $L_\alpha$ is a union of $|\alpha|$ many sets of cardinality $\leq |\alpha|$ (by inductive hypothesis), hence of cardinality $\leq |\alpha|$.

\end{proof}But why would an elementary substructure of an $L_\lambda$ have to be itself an $L_\beta$? This is where the absoluteness of the construction of $L$ strikes yet again!

\begin{lemma}[Condensation lemma]\label{lem-condensation}There is a finite set $T$ of axioms of $\ZF - \text{Power Set}$ so that if $M$ is a transitive set with $M\models T + \VL$, then $M = L_\lambda$ for some limit ordinal $\lambda$.

\end{lemma}\begin{proof}Let the axioms of $T$ be \textit{Pairing}, \textit{Union}, \textit{Set Existence}, together with all (instances of) axioms of $\ZF$ used to prove that all the theorems leading up to the fact that for all $\alpha$, $L_\alpha$  exists and that $\alpha \mapsto L_\alpha$ is $\Delta_1$ (and hence absolute). (We have proved only finitely meany theorems so far so we only needed finitely many axioms!)

Suppose for a transitive set $M$, $M\models T + \VL$. Let $\lambda$ be the least ordinal not in $M$.
We must have that $\Ord^M = \lambda$, by absoluteness of
ordinal.  Moreover, $\lambda$ must be a limit ordinal since for each $\alpha \in M$, $\alpha \cup \{\alpha\}$ is in $M$ since $M$ satisfies \textit{Pairing} and \textit{Union}.

We have that

\begin{equation}
M\models \forall x \exists \alpha\in \Ord (x \in L_\alpha),
\end{equation}

thus

\begin{equation}
\forall x \in M  \exists \alpha < \lambda (x \in L^M_\alpha).
\end{equation}

By absoluteness of $\alpha \mapsto L_\alpha$, we have $L^M_\alpha = L_\alpha$ and therefore

\begin{equation}
M \subseteq \bigcup_{\alpha \in M} L_\alpha =  \bigcup_{\alpha < \lambda} L_\alpha = L_\lambda.
\end{equation}

On the other hand, for each $\alpha < \lambda$, $L_\alpha^M$ exists in $M$ (since $T$ is strong enough to prove this), and by absoluteness

\begin{equation}
L_\lambda =   \bigcup_{\alpha < \lambda} L_\alpha =  \bigcup_{\alpha \in M} L^M_\alpha \subseteq M.
\end{equation}

\end{proof}We now put condensation to use as described above.

\begin{lemma}\label{lemma-l-gch}Suppose $V=L$. If $\kappa$ is a cardinal and $x \subseteq \kappa$, then $x \in L_{\kappa^+}$.

\end{lemma}\begin{proof}Since we assume $\VL$, there exists limit $\lambda > \kappa$ such that $x \in L_\lambda$ and such that $L_\lambda \models T + \VL$, where $T$ is as in the \textbf{condensation lemma}. Such a $\lambda$ exists by the \textbf{reflection theorem} (Theorem~\ref{thm-reflection}).  Let $X = \kappa \cup \{x\}$. By choice of $\lambda$, $X \subseteq L_\lambda$.

By the \href{https://en.wikipedia.org/wiki/L\%C3\%B6wenheim\%E2\%80\%93Skolem\_theorem}{Löwenheim-Skolem Theorem}, there exists an \textbf{elementary substructure} $N \preceq L_\lambda$ such that

\begin{equation*}
\tag{$*$}
    X \subseteq N \subseteq L_\lambda \quad \text{ and } \quad |N| = |X|.
\end{equation*}

$N$ is not necessarily transitive, but since it is well-founded we can take its \textbf{Mostowski collapse} (Theorem~\ref{thm-Mostowski-collapse}) and obtain a \textbf{transitive} set $M$
together with an \textbf{isomorphism} $\pi: (N,\in) \to (M,\in)$.

Since $\kappa$ is contained in both $M$ and $N$, and is already transitive, it is straightforward to show via induction that $\pi(\alpha) = \alpha$ for all $\alpha \in \kappa$. Since $x \subseteq \kappa$, this also yields $\pi(x) = x$. This implies in turn that $x \in M$.

As $(M,\in)$ is isomorphic to $(N,\in)$ and $N \preceq L_\lambda$, $M$ satisfies the same sentences as $(L_\lambda, \in)$. In particular, $M \models T + \VL$. By the \textbf{condensation lemma}, $M = L_\beta$ for some $\beta$.

This implies, by Proposition~\ref{prop-card-Lalpha},

\begin{equation}
|\beta| = |L_\beta| = |M| = |N| = |X| = \kappa < \kappa^+ \leq \lambda.
\end{equation}

Since $x \in L_\beta$ and $\beta < \kappa^+$, it follows that $x \in L_{\kappa^+}$, as desired.

\end{proof}\begin{theorem}[Gödel]\label{thm-l-gch}If $\VL$, then for all cardinals $\kappa$, $2^\kappa = \kappa^+$.

\end{theorem}\begin{proof}If $\VL$, then by Lemma~\ref{lemma-L-GCH}, $\mathcal{P}(\kappa) \subseteq L_{\kappa^+}$. With Proposition~\ref{prop-card-Lalpha}, we obtain

\begin{equation}
2^\kappa = |\mathcal{P}(\kappa)| \leq |L_{\kappa^+}| = \kappa^+.
\end{equation}

\end{proof}\begin{corollary}\label{cor-con(zfc+gch)}If $\ZF$ is consistent, so is $\ZF + \AC + \mathsf{GCH}$.

\end{corollary}