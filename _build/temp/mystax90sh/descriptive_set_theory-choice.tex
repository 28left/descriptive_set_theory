\subsection{The Axiom of Choice}

In the previous lectures, a number of \textbf{regularity principles} for sets of real numbers emerged:

\begin{itemize}
\item \textbf{(PS)} ~~~ the \textbf{perfect subset property},
\item \textbf{(LM)} ~~~ \textbf{Lebesgue measurability},
\item \textbf{(BP)}  ~~~ the \textbf{Baire property}.
\end{itemize}

We have seen that the Borel sets in $\Real$ have all these properties. In this lecture we will show how to construct counterexamples for each of these principles. The proofs make essential use if the \textbf{Axiom of Choice}:

\begin{quote}
\textbf{(AC)} ~~~ Every set $\mathcal{X}$ of non-empty sets has a choice-function.
\end{quote}

A \textbf{choice function} for $\mathcal{X}$ is a function $f$ that assigns every set $Y \in \mathcal{X}$ an element $y \in Y$.

One of the most famous applications of the Axiom of Choice is Vitali's construction of a non-Lebesgue measurable set.

\begin{theorem}[Vitali]\label{thm-vitali-nonmeasurable}There exists a set $A \subseteq \Real$ that is not Lebesgue measurable.

\end{theorem}\begin{proof}Put
\begin{equation*}
x \sim y \; \text{ if and only if }\;  x-y \in \Rat.
\end{equation*}
It is straightforward to check that this is an equivalence relation on $\Real$.
Using a choice function on the equivalence classes of $\sim$ intersected with the unit interval $[0,1]$, we pick from each equivalence class a representative from $[0,1]$, and collect them in a set $S$.

If we let, for $r \in \Rat$,
\begin{equation*}
S_r = \{s+r \colon s \in S \},
\end{equation*}
then

\begin{equation}
S_r \cap S_t \quad \text{ for $r \neq t$}.
\end{equation}

Suppose $S$ is measurable. Then so is each $S_r$, and $\lambda(S_r) = \lambda(S)$.

If $\lambda(S) = 0$, then $\lambda(\Real) = 0$, which is impossible.
On the other hand, if $\lambda(S) > 0$, then, by countable additivity,
\begin{equation*}
2 = \lambda([0,2]) \geq \lambda\left(\bigcup_{r\in \Rat\cap[0,1]} S_r\right) = \sum_{r\in \Rat\cap[0,1]} \lambda(S) = \infty,
\end{equation*}
contradiction.

\end{proof}The Axiom of Choice is equivalent to a number of other principles. We will use the \textbf{Well-ordering Principle}:

\begin{quote}
\textbf{(WO)} ~~~ Every set $X$ can be well-ordered.
\end{quote}

This means that one can define a binary relation $<$ on $X$ so that every non-empty subset of $X$ has a $<$-minimal element.

We use (WO) to construct a set $B\subseteq \Real$ such neither $B$ nor $\Real\setminus B$ contains a perfect subset. Such sets are called \textbf{Bernstein sets}.

\begin{theorem}\label{thm-bernstein}There exists a Bernstein set.

\end{theorem}\begin{proof}Let $\mathcal{P}$ be the set of perfect subsets of $\Real$. We can well-order this set, say
\begin{equation*}
\mathcal{P} = \{P_\xi \colon \xi < 2^{\aleph_0} \}.
\end{equation*}
Note that every perfect subset corresponds to Cantor-Scheme, which can be coded by a real number (see \ref{ch_perfect}). Therefore, there are at most $2^{\aleph_0}$-many perfect subsets of $\Real$, and it is not hard to see that there are exactly $2^{\aleph_0}$-many.

Furthermore, we assume each $P_\xi$ is well-ordered.

Pick $a_0 \neq b_0$ from $P_0$. Assume we have chosen $\xi < 2^{\aleph_0}$, and $\{a_\beta\colon \beta < \xi \}$ and $\{b_\beta\colon \beta < \xi \}$ so that
\begin{equation*}
a_\beta, b_\beta \in P_\beta \quad \text{ and } \quad \text{all $a_\beta, b_\gamma$ pairwise distinct},
\end{equation*}
we can choose $a_\xi, b_\xi \in P_\xi$ to be the first two elements of $P_\xi \setminus \bigcup_{\gamma < \xi} \{a_\gamma, b_\gamma\}$. This is possible since a perfect subset of $\Real$ has cardinality $2^{\aleph_0}$, and $\xi< 2^{\aleph_0}$.

Put
\begin{equation*}
A = \{a_\xi \colon \xi < 2^{\aleph_0} \} \qquad B = \{b_\xi \colon \xi < 2^{\aleph_0} \}.
\end{equation*}
Neither $A$ nor $B$ has a perfect subset by construction, and since $A \subseteq \Real\setminus B$, $B$ is a Bernstein set.

\end{proof}\begin{proposition}\label{prop-bernstein-bp}A Bernstein set does not have the Baire property.

\end{proposition}\begin{proof}Assume for a contradiction a Bernstein set $B$ has the Baire property. By an~exercise in the previous chapter, we can write $B = M \cup G$, where $M$ is meager and $G$ is $G_\delta$.

At least one of $B$, $\Real\setminus B$ is not meager. Wlog assume $B$ is not meager. (If not, obtain the representation ``meager $\cup$ $G_\delta$'' above for $\Real\setminus B$ and proceed analogously.) Then $G \subseteq B$ must be non-meager, too, and hence is an uncountable $G_\delta$ set. By Theorem~\ref{thm-subsets-Polish}, $G$ is Polish and hence must contain a perfect subset, contradiction.

\end{proof}\begin{framed}
\textbf{Exercise}\\
Show that a Bernstein set is not Lebesgue measurable.
\end{framed}

The existence of arbitrary choice functions appears to be a rather strong assumption. It has consequences that seem paradoxical in the sense that they conflict with basic intuitions we have about objects and they behavior with respect to size or other characteristics. Arguably the most famous example is the \textbf{Banach-Tarski Paradox}, which uses the Axiom of Choice to partition a ball in $\Real^3$ into finitely many pieces, and then, using rigid transformations (i.e.{\textbackslash}~rotations and translations), to assemble them into two balls of the original size.

On the other hand, the Axiom of Choice implies or is even equivalent to many principles that are applied throughout many areas of mathematics, such as the existence of bases of vector spaces, Zorn's Lemma, Tychonoff's Theorem on the compactness of product spaces, the Hahn-Banach Theorem, or the Prime Ideal Theorem.

For some applications, however, a weaker form of the Axiom of Choice is sufficient.

The \textbf{Axiom of Countable Choice}:

\begin{quote}
\textbf{(AC\textsubscript{$\omega$})} ~~~ Every \textit{countable} family $\mathcal{X}$ of non-empty sets has a choice-function.
\end{quote}

Stronger than Countable Choice, but still weaker than the full Axiom of Choice is  \textbf{Axiom of Dependent Choice}:

\begin{quote}
\textbf{(DC)} ~~~ If $E$ is a binary relation on a non-empty set $A$, and if for every $a \in A$ there exists $b \in A$ such that $a \: E\: b$, then there exists a function $f:\Nat \to A$ such that for all $n \in \Nat$, $f(n) \: E \: f(n+1)$.
\end{quote}

A seminal result by \citet{Solovay:1970a} showed that DC is no longer sufficient to prove the existence of non-regular sets in the above sense. He constructed (though under a large cardinal assumption) a model of ZF+DC in which every set of real numbers is Lebesgue measurable, has the Baire property, and has the perfect subset property.