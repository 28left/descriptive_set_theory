\subsection{Trees}

Let $A$ be a set. Recall that the set of all finite sequences over $A$ is denoted by $\Str[A]$, while $A^\Nat$ denotes the set of all infinite sequences over $A$. Given $\alpha \in A^\N$, $n \in \N$, $\alpha\Rest{n}$ denotes the initial segment of $\alpha$ of length $n$.

\begin{definition}\label{def-tree}A \textbf{tree} on $A$ is a set $T \subseteq \Str[A]$ that is \textbf{closed under prefixes}, that is

\begin{equation}
\forall \sigma, \tau \; [\tau \in T \: \& \: \sigma \Sleq \tau \; \Rightarrow \; \sigma \in T ]
\end{equation}

\end{definition}We call the elements of $T$ \textbf{nodes}.

A sequence $\alpha \in A^\Nat$ is an \textbf{infinite path through} or \textbf{infinite branch of} $T$ if for all $n$, $\alpha\Rest{n} = (\alpha_0, \alpha_1, \dots, \alpha_{n -1}) \in T$. We denote the set of infinite paths through $T$ by $[T]$.

An important criterion for a tree to have infinite paths is the following.

\begin{theorem}[König’s Lemma]\label{thm-koenigs-lemma}Any tree $T$ with infinitely many nodes that is \textbf{finite branching} (i.e. each node has at most finitely many immediate extensions) has an infinite path.

\end{theorem}\begin{proof}We construct an infinite path inductively.

Let $T_\sigma$ denote the tree ``above'' $\sigma$, i.e. $T_\sigma = \{ \tau \in \Str[A] \colon \sigma\Conc\tau \in T\}$. If $T$ is finite branching,  by the \textbf{pigeonhole principle}, at least one of the sets $T_\sigma$ for $|\sigma| = 1$ must be infinite. Pick such a $\sigma$ and let $\alpha\Rest{1} = \sigma$.

Repeat the argument for $T = T_\sigma$ and continue inductively. This yields a sequence $\alpha \in [T]$.

\end{proof}If $[T] = \emptyset$, we call $T$ \textbf{well-founded}. The motivation behind this is that $T$ is well-founded if and only if the \textbf{inverse prefix} relation

\begin{equation}
\sigma \preceq \tau \quad :\Leftrightarrow \quad \sigma \Sgeq \tau
\end{equation}

is well-founded, i.e. it does not have an infinite descending chain.

If $T \neq \emptyset$ is well-founded, we can assign $T$ an ordinal number, its \textbf{rank} $\rho(T)$.

\begin{itemize}
\item If $\sigma$ is a \textbf{terminal node}, i.e. $\sigma$ has no extensions in $T$, then let $\rho_T(\sigma) = 0$.


\item If $\sigma$ is not terminal, and $\rho_T(\tau)$ has been defined for all $\tau \Sgr \sigma$, we set $\rho_T(\sigma) = \sup \{\rho_T(\tau)+1 \colon \tau \in T, \tau \Sgr \sigma \}$.


\item Finally, set $\rho(T) = \sup\{\rho_T(\sigma) + 1 \colon \sigma\in T \} = \rho_T(\Estr)+1$, where $\Estr$ denotes the empty string.
\end{itemize}

\subsubsection{Orderings on trees}

Now suppose $A$ is linearly ordered by a relation $\leq_A$.
The \textbf{lexicographical ordering} $\leq_{\lex}$ of $\Str[A]$ is defined as

\begin{equation*}
\sigma \leq_{\lex} \tau \quad :\Leftrightarrow \quad \\
        \sigma = \tau \; \text{ or } \; \exists i <|\sigma|,|\tau| \: \left( \sigma_i <_A \tau_i \text{ and } \forall j < i \; \sigma_j = \tau_j  \right )
\end{equation*}

This ordering extends to $A^\Nat$ in a natural way.

\begin{proposition}\label{prop-leftmost-branch}If $\leq_A$ is a well-ordering of $A$ and $T$ is a tree on $A$ with $[T] \neq \emptyset$, then $[T]$ has a $\leq_{\lex}$-minimal element, the \textbf{leftmost branch}.

\end{proposition}\begin{proof}We \textbf{prune} the tree $T$ by deleting any node that is not on an infinite branch. This yields a subtree $T' \subseteq T$ with $[T'] = [T]$.

Let $T'_n = \{\sigma \in T' \colon |\sigma| = n \}$. Since $\leq_A$ is a well-ordering on $A$, $T'_1$ must have a $\leq_{\lex}$-least element. Denote it by $\alpha\Rest{1}$. Since $T'$ is pruned, $\alpha\Rest{1}$ must have an extension in $T$, and we can repeat the argument to obtain $\alpha\Rest{2}$.

Continuing inductively, we define an infinite path $\alpha$ through $T'$, and it is straightforward to check that $\alpha$ is a  $\leq_{\lex}$-minimal element of $[T']$ and hence of $[T]$.

\end{proof}We can combine the $\leq_{\lex}$-ordering with the inverse prefix order to obtain a linear ordering of $\Str[A]$. This ordering has the nice property that if $A$ is well-ordered and $T$ is well-founded, then the ordering restricted to $T$ is a well-ordering.

\begin{definition}\label{def-kleene-brouwer}The \textbf{Kleene-Brouwer ordering} $\leq_{\KB}$ of $\Str[A]$ is defined as follows.

$\qquad \sigma \leq_{\KB} \tau$ $\quad :\Leftrightarrow \quad$  $\sigma \Sgeq \tau \;\;$  or $\; \; \sigma \leq_{\lex} \tau$.

\end{definition}This means $\sigma$ is smaller than $\tau$ if it is a proper extension of $\tau$ or ``to the left'' of $\tau$.

We now have

\begin{proposition}\label{prop-kb-wellorder}Assume $(A,\leq)$ is a well-ordered set. Then for any tree $T$ on $A$,

$\qquad$ $T$ is well-founded $\quad \Leftrightarrow \quad$  $\leq_{\KB}$ restricted to $T$ is a well-ordering.

\end{proposition}\begin{proof}Suppose $T$ is not well-founded. Let $\alpha \in [T]$. Then $\alpha\Rest{0}, \alpha\Rest{1}, \dots$ is an infinite descending sequence with respect to $\leq_{\KB}$.

Conversely, suppose $\sigma_0 >_{\KB} \sigma_1 >_{\KB} \dots$ is an infinite descending sequence in $T$. By the definition of $>_{\KB}$, this implies $\sigma_1(0) \geq_A \sigma_2(0) \geq_A \dots$ as a sequence in $A$. Since $A$ is well-ordered, this sequence must eventually be constant, say $\sigma_n(0) = a_0$ for all $n \geq n_0$.

Since the $\sigma_n$ are descending, by the definition of $\leq_{\KB}$ it follows that $|\sigma_n| \geq 2$ for $n > n_0$. Hence we can consider the sequence $\sigma_{n_0+1}(1) \geq_A \sigma_{n_0+2}(1) \geq_A  \dots$ in $A$. Again, this must be constant $= a_1$ eventually. Inductively, we obtain a sequence $\alpha = (a_0, a_1, a_2, \dots) \in [T]$, that is, $T$ is not well-founded.

\end{proof}\begin{framed}
\textbf{Caution}\\
The order type of a well-founded tree under $\leq_{\KB}$ usually is not equal to its rank $\rho(T)$.
\end{framed}

\subsubsection{Coding trees}

We can also define an ordering on $\Str[A]$ via an injective mapping from $\Str[A]$ to some linearly ordered set $A$. We will use this repeatedly for the case $A = \Nat$ and $A = \{0,1\}$.

For $A = \Nat$, we can use the standard coding mapping
\begin{equation*}
\pi: (a_0, a_1, \dots, a_n) \mapsto p_0^{a_0+1}p_1^{a_1}\cdots p_n^{a_n},
\end{equation*}
where $p_k$ is the $k$th prime number. This embeds $\Nstr$ into $\Nat$, and we can well-order $\Nstr$ by letting $\sigma < \tau$ if and only if $\pi(\sigma) < \pi(\tau)$.

For $A = \{0,1\}$ we set
\begin{equation*}
\pi: (b_0,b_1, \dots, b_{n-1}) \mapsto (2^n-1) + \sum_{i=0}^{n-1} b_i 2^i.
\end{equation*}

These two mappings allows us henceforth to see \textbf{trees as subsets of the natural numbers}. This will be an important component in exploring the relation between topological and arithmetical complexity.

\subsubsection{Trees and closed sets}

Let $A$ be a set with the discrete topology. Consider $A^\Nat$ with the product topology (and compatible metric) defined in Lecture~2.

\begin{proposition}\label{prop-trees-closed}A set $F \subseteq A^\Nat$ is closed if and only if there exists a tree $T$ on $A$ such that $F = [T]$.

\end{proposition}\begin{proof}Suppose $F$ is closed. Let
\begin{equation*}
T_F = \{\sigma \in \Str[A] \colon \sigma \Sle \alpha \text{ for some }\alpha \in F\}.
\end{equation*}
Then clearly $F \subset [T_F]$. Suppose $\alpha \in [T_F]$. This means for any $n$, $\alpha\Rest{n} \in T_F$, which implies that there exists $\beta_n \in F$ such that $\alpha_n \Sle \beta_n$. The sequence $(\beta_n)$ converges to $\alpha$, and since $F$ is closed, $\alpha \in F$.

For the other direction, suppose $F = [T]$. Let $\alpha \in A^\Nat \setminus F$. Then there exists an $n$ such that $\alpha\Rest{n} \not\in T$. Since a tree is closed under prefixes, no extension of $\alpha\Rest{n}$ can be in $T$. This implies $\Cyl{\alpha\Rest{n}} \subseteq A^\Nat \setminus F$, and hence $A^\Nat \setminus F$ is open.

\end{proof}\subsubsection{Continuous mappings on product spaces}

Let $f: A^\Nat \to A^\Nat$ be continuous. We define a mapping $\phi: \Str[A] \to \Str[A]$ by setting
\begin{equation*}
\phi(\sigma) = \text{ the longest $\tau$ with $|\tau| \leq |\sigma|$ such that } \Cyl{\sigma} \subseteq f^{-1}(\Cyl{\tau}).
\end{equation*}

This mapping has the following properties:

\begin{enumerate}
\item It is \textbf{monotone}, i.e. $\sigma \Sleq \tau$ implies $\phi(\sigma) \Sleq \phi(\tau)$.
\item For any $\alpha \in A^\Nat$ we have $\lim_n |\phi(\alpha\Rest{n})| = \infty$.
This follows directly from the continuity of $f$: For any neighborhood $\Cyl{\tau}$ of $f(\alpha)$ there exists a neighborhood $\Cyl{\sigma}$ of $\alpha$ such that $f(\Cyl{\sigma}) \subseteq \Cyl{\tau}$. But $\tau$ has to be of the form $\tau = f(\alpha)\Rest{m}$, and $\sigma$ of the form $\alpha\Rest{n}$. Hence for any $m$ there must exist an $n$ such that $\phi(\alpha\Rest{n}) \Sgeq f(\alpha)\Rest{m}$.
\end{enumerate}

On the other hand, if a function $\phi: \Str[A] \to \Str[A]$ satisfies (1.) and (2.), it induces a function $\phi^*: A^\Nat \to A^\Nat$ by letting
\begin{equation*}
\phi^*(\alpha) = \lim_n \phi(\alpha\Rest{n}) = \text{ the unique sequence extending all $\phi(\alpha\Rest{n})$}.
\end{equation*}

This $\phi^*$ is indeed continuous: The preimage of $\Cyl{\tau}$ under $\phi^*$ is given by
\begin{equation*}
(\phi^*)^{-1}(\Cyl{\tau}) = \bigcup \{\Cyl{\sigma} \colon \phi(\sigma) \Sgeq \tau \},
\end{equation*}
which is an open set.

We have shown the following.

\begin{proposition}\label{prop-product-continuous}A mapping $f:A^\Nat \to A^\Nat$ is continuous if and only if there exists a mapping $\phi$ satisfying (1) and (2) such that $f = \phi^*$.

\end{proposition}Note that we can completely describe a topological concept, continuity, through a relation between finite strings.