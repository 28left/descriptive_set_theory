\subsection{Coding Borel Sets}

In this chapter, we take a further look at Borel subsets of $\Baire$. As is common in this setting, we call the elements of $\Baire$ \textbf{reals}. This is motivated by the fact that, via the continued fration expansion, $\Baire$ is~homeomorphic~to~the~set~of~irrational~real~numbers. Suppose $U \subseteq \Baire$ is open. Then there exists a set $W \subseteq \Nstr$ such that

\begin{equation}
U = \bigcup_{\sigma \in W} \Cyl{\sigma}.
\end{equation}

Using a standard (effective) coding procedure, we can identify a finite sequence of natural numbers with a natural number, and thus can see $W$ as a subset of $\Nat$.

If we provide a Turing machine with oracle $W$, we can semi-effectively test for membership in $U$ as follows. Assume we want to determine whether some $\alpha \in \Baire$ is in $U$. Write $\alpha$ on another oracle tape, and start scanning the $W$ oracle. If we retrieve a $\sigma$ that coincides with an initial segment of $\alpha$, we know $\alpha \in U$. On the other hand, if $\alpha \in U$, then we will eventually find some $\alpha\Rest{n}$ in $W$. If $\alpha \not\in U$, then the search will run forever. In other words, given $W$, $U$ is \textbf{semi-decidable}, or, extending terminology from subsets of $\Nat$ to subsets of $\Baire$, $U$ is
\textbf{recusively enumerable} relative to $W$.

Similarly, we can identify a closed set $F$ with the code for the tree

\begin{equation}
T_F = \{\alpha\Rest{n} \colon \alpha\in F,\, n\in \Nat \}.
\end{equation}

Then determining whether $\alpha \in F$ is \textbf{co-r.e.} in (the code of) $T_F$. If $\alpha \not\in F$ we will learn so after a finite amount of time.

These simple observations suggest the following general approach to Borel sets.

In this lecture we will fully develop this correspondence. Later, we will see that it even extends beyond the finite level.

\subsubsection{Some notation for reals, strings, and numbers}

We fix a computable bijection $\pi: \Nat \to \Nstr$. In general, we will often use string and their images under $\pi$ interchangeably, that is, for example, if $A \subset \Nat$, we will write $\sigma \in A$ to denote $\pi(\sigma) \in A$.
We will also freely identify infinite binary sequences with the set of natural numbers they represent as their characteristic function.

Furthermore, let $\Tup{.,.}$ be the standard coding function for pairs,

\begin{equation}
\Tup{x,y} = \frac{(x+y)(x+y+1)}{2}+y.
\end{equation}

Finally, let us define the following operation on elements of Baire (or Cantor) space: Given $\beta\in \Baire$,

\begin{itemize}
\item let $\beta'$ be the real defined by $\beta'(n) = \beta(n+1)$. (\textit{We cut the first entry.})
\item for $m \geq 0$, let $(\beta)_m$ be the \textbf{$m$-th column} of $\beta$, $(\beta)_m(n) = \beta(\Tup{m,n})$.
\end{itemize}

\subsubsection{Borel codes of finite order}

Borel codes are defined inductively.

\begin{definition}\label{def-borel-codes}Let $\gamma \in \Baire$.

\begin{itemize}
\item Suppose $\gamma \in \Baire$ is such that $\gamma(0) = 1$ and $\gamma' \in \Baire$. $\gamma$ \textbf{is a $\bSigma^0_1$ code} for the open set
\end{itemize}
\begin{equation*}
U = \bigcup_{\gamma'(\sigma) = 0} \Cyl{\sigma}
\end{equation*}

\begin{itemize}
\item If $\gamma$ is such that $\gamma(0)=2$ and $\gamma'$ is a $\bSigma^0_n$ code for $A \subseteq \Baire$, we say $\gamma$ \textbf{is a $\bPi^0_n$ code} for $\Co{A}$.


\item If $\gamma$ is such that $\gamma(0)=3$ and for each $m$, $(\gamma')_m$ is a $\bPi^0_n$ code of a set $A_m$, we say $\gamma$ \textbf{is a $\bSigma^0_{n+1}$ code} for $\bigcup_n A_n$.
\end{itemize}

\end{definition}The first position in each code indicates the kind of set it codes -- an open set, a complement, or a union.

Note that the definition of Borel code actually assigns codes to \textbf{representations of sets}. A Borel set can have (and has) multiple codes, just as it has multiple representations. We can, for example, represent an open set by different sets $W$ of initial segments.

Moreover, every $\bSigma^0_1$ set is also $\bSigma^0_2$, and thus a set has codes which reflect the ``more complicated'' definition of the $\bSigma^0_1$ set as a union of closed sets. It is useful to keep this distinction between a Borel set and its Borel representation in mind.

The following is a straightforward induction.

\begin{proposition}\label{prop-borel-codes}Every $\bSigma^0_n$ ($\bPi^0_n$) set has a $\bSigma^0_n$ ($\bPi^0_n$) Borel code, and every $\bSigma^0_n$ ($\bPi^0_n$) code represents a $\bSigma^0_n$ ($\bPi^0_n$) set.

\end{proposition}
%  The proposition actually holds for *all* Borel sets, which can be proved by **transfinite induction**. However, we have not introduced Borel sets of transfinite order yet, so we state the existence of codes only for $\bPi^0_n$ ($\bSigma^0_n$) sets. In the next chapter, we will study the *full* Borel hierarchy, and then it should be clear that the above proposition extends to all Borel sets.

%  ## The tree structure of Borel codes
% 
% Each Borel code induces a tree structure that reflects how the corresponding Borel set is built up from closed sets.
% The terminal nodes are given by codes for closed sets (the ones starting with "$2$"), since they are the ``building blocks'' of the Borel sets and hence are not extend/split further. A "$3$-code" represents a node  with just one immediate successor, while a "$4$-code" corresponds to a node with infinitely many immediate successors. Given a Borel code $\gamma$, we **denote the corresponding tree by $T_\gamma$**.
% 
% The tree of a Borel code defined this way **is always well-founded** (i.e. has no infinite path), since a Borel code is defined via a well-founded recursion. The rank of the tree is a countable ordinal. 
% 
% How hard is it to decide whether a given real is a Borel code? We will see later that this question is quite difficult. More precisely, we will see that the **set $\Op{Bc}$ of all Borel codes is not Borel**. Deciding whether a tree on $\Nat$ is well-founded will play a fundamental role in this regard.
% 
% The tree structure of a code also lets us assign levels to a Borel code similar to the levels of the Borel hierarchy:
% 
% - Trees with a single node $(2)$ correspond to **$\bPi^0_1$ codes**. 
% - If $T$ is a $\bPi^0_n$ ($\bSigma^0_n$) code, then the tree with new top node $(3)$ represents a $\bSigma^0_n$ ($\bPi^0_n$) code. 
% - And if $T_n$ is a sequence of $\bPi^0_n$ codes, then the tree with new top node $(4)$ and each $T_n$ directly below it corresponds to a **$\bSigma^0_{n+1}$ code**.
%

\subsubsection{Computing with Borel codes}

Suppose $\gamma$ is a \textbf{computable} code for an $F_\sigma$ set $B$. We may assume $\gamma$ is of the form $(3,\gamma')$, with each column $(\gamma')_m$ being of the form $(2,1,(\alpha)_m)$, coding a closed set $F_m$.

With this, we can express membership in $B$ as follows:

\begin{align*}
    \beta \in B \quad & \Leftrightarrow \quad \exists m \: [\text{$\beta$ is in the set coded by $(\gamma')_m$}] \\
        & \Leftrightarrow \quad \exists m \forall n \: [\beta\Rest{n} \text{ is not in the set coded by } (\alpha)_m]. \\
        & \Leftrightarrow \quad \exists m \forall n \: [(\alpha)_m(\beta\Rest{n}) \neq 0 ].
\end{align*}

Note that, since we assume $\gamma$ to be computable, the \textbf{inner predicate} $R(m,\sigma)$ given by

\begin{equation}
R(m,\sigma) :\iff (\alpha)_m(\sigma) \neq 0
\end{equation}

is \textbf{decidable}, that is, it can be decided by a Turing machine.

Hence any $\bSigma^0_2$ set $B$ with a computable code can be represented in the following form:

\begin{quote}
There exists a decidable predicate $R(m,\sigma)$ such that
\end{quote}
\begin{equation*}
\beta \in B \quad \Leftrightarrow \quad \exists m \: \forall n \; \neg R(m, \beta\Rest{n}).
\end{equation*}

Conversely, if $R(m,\sigma)$ is a (decidable) predicate, let

\begin{equation}
F_m = \{ \beta \colon \forall n \: R(m,\beta\Rest{n}) \}.
\end{equation}

We claim that $F_m$ is closed: Define a tree $T_m$ by letting

\begin{equation}
\sigma \in T_m : \iff \forall \tau \Sleq \sigma \: R(m, \tau).
\end{equation}

Then $[T_m] = F_m$. Moreover,

\begin{equation}
\beta \in \bigcup_m F_m \iff \exists m \forall n \: R(m,\beta\Rest{n})
\end{equation}

Thus, there seems to be a close connection between $F_\sigma$ sets with computable Borel codes and sets definable by $\Sigma^0_2$ formulas over computable predicates. Given that we introduced the notation $\bSigma^0_2$ for $F_\sigma$ sets earlier, this is perhaps not very surprising.

In this analysis, there seems to be nothing specific about the $F_\sigma$ used in the example. Indeed, it can be extended to Borel sets of finite order, which we will do next.

We will next introduce the
\textbf{lightface} Borel hierarchy and show that it corresponds to Borel sets of finite order with recursive codes. Using \textbf{relativization}, we then obtain a complete characterization of Borel sets of finite order: \textit{They are precisely those sets definable by arithmetical formulas, relative to a real parameter.}

But before we do that, we observe a basic fact about how we can compute with codes.

\begin{lemma}\label{lem-borel-codes-clopen}Suppose $\gamma$ is a Borel code of finite order representing a set $B \subseteq \Baire$. Suppose further $C \subseteq \Baire$ is clopen and both $C$ and its complement have computable $\bSigma^0_1$ codes. We can, uniformly in $\gamma$, compute Borel codes for $B \cap C$ and $B \cup C$ of the same Borel complexity as $\gamma$.

\end{lemma}\begin{lemma}\label{lem-borel-codes-shift}Suppose $\gamma$ is a Borel code of finite order representing a set $B \subseteq \Baire$. Then can, uniformly in $\gamma$ and $k$, compute Borel codes of the same Borel complexity as $\gamma$ for the set

\begin{equation}
B'_k = \{ \delta \colon (k, \delta) \in B\}
\end{equation}

\end{lemma}We leave the proofs as an exercise. Proceed by induction on the Borel complexity of $\gamma$.

\subsubsection{The effective Borel hierarchy}

\begin{definition}[The Lightface Hierarchy]\label{def-effective-borel}A set $A \subseteq \Baire$ is

\begin{itemize}
\item (lightface) $\Sigma^0_1$ if there exists a computable predicate $R(\sigma)$ such that
\end{itemize}
\begin{equation*}
\alpha \in A \iff \exists k \: R(\alpha\Rest{k}),
\end{equation*}
\begin{itemize}
\item (lightface) $\Pi^0_n$ if $\Co{A}$ is $\Sigma^0_n$,
\item (lightface) $\Sigma^0_{n+1}$ if there exists a $\Pi^0_n$ set $P$ such that
\end{itemize}
\begin{equation*}
\alpha \in A \iff \exists n \; (n,\alpha) \in P.
\end{equation*}
\end{definition}The following result is at the heart of the effective theory.

\begin{proposition}\label{prop-computable-codes}Let $A \subseteq \Baire$. Then

\begin{quote}
$A$ is (lightface) $\Sigma^0_n$ ($\Pi^0_n$) iff $A$ has a computable $\bSigma^0_n$ ($\bPi^0_n$) code.
\end{quote}

\end{proposition}\begin{proof}($\Rightarrow$) We proceed by induction on the Borel complexity.

Suppose $A$ is $\Sigma^0_1$. Let $R$ be computable such that $A = \{ \alpha \colon \exists n \: R(\alpha\Rest{n})\}$. Let

\begin{equation}
W = \{ \sigma \in \Nstr \colon R(\sigma)\}.
\end{equation}

We have $\alpha \in A$ if and only if $\alpha \in \bigcup_{\sigma \in W} \Cyl{\sigma}$.
Since $R$ is decidable, $W$ is computable and $\gamma \in \Baire$ given by

\begin{equation}
\gamma(n) = 
	\begin{cases}
		1 & n = 0, \\
		0 & n \geq 1 \: \& \: \pi(n-1) \in W,\\
		17 & n \geq 1 \: \& \: \pi(n-1) \notin W, 
	\end{cases}
\end{equation}

is a computable $\bSigma^0_1$ code for $A$.

If $A$ is $\Pi^0_n$, then $A = \Co{B}$ for some $\Sigma^0_n$ $B$. By inductive hypothesis, $B$ has a computable $\bSigma^0_n$ code $\gamma$. Then $(2,\gamma)$ is a computable $\bPi^0_n$ code for $\Co{A}$.

Finally, assume that $A$ is $\Sigma^0_{n+1}$. Let $P$ be $\Pi^0_n$ such that $\alpha \in A \iff \exists n \; (n,\alpha) \in P$.

By inductive hypothesis, $P$ has a computable $\bPi^0_n$ code $\gamma$.
If we let $P_m = \{\beta \colon (m,\beta) \in P\}$, then $A = \bigcup P_m$. Thus, it suffices to show that we can uniformly obtain codes for $P_m$. This follows from Lemma~\ref{lem-Borel-codes-shift}.

($\Leftarrow$) We proceed by induction on the complexity of the code $\gamma$.

If $\gamma$ is of the form $(1,\alpha)$, with $\alpha$ coding an open set $U$. Then

\begin{equation}
\alpha \in U \iff \exists n \: \alpha(\Rest{n})= 0.
\end{equation}

Since $\gamma$ is assumed to be computable, the computable relation

\begin{equation}
R(\sigma) :\iff \alpha(\sigma) = 0
\end{equation}

witnesses that $U$ is $\Pi^0_1$.

If $\gamma = (2, \alpha)$ is a $\bPi^0_n$ code, then $\alpha$ is a $\bSigma^0_n$ code. By inductive hypothesis, the set coded by $\alpha$ is $\Sigma^0_n$, so by definition of the effective hierarchy and the Borel codes, $\gamma$ codes a $\Pi^0_n$ set.

Finally, assume $\gamma = (3,\alpha)$ is a $\bSigma^0_{n+1}$ code for a set $B$. Then each $(\alpha)_m$ is a $\bPi^0_n$ code for a set $A_m$.

\begin{lemma}\label{lem-borel-codes-inverse-shift}If $(\alpha_m)$ is a uniformly computable sequence of $\bPi^0_n$ codes for sets $A_m$, respectively, then there exists a $\bPi^0_n$ code $\alpha$ for the set

\begin{equation}
A = \{(m,\beta) \colon \beta \in A_m\}
\end{equation}

\end{lemma}\begin{proof}Similar to Lemma~\ref{lem-Borel-codes-shift}

\end{proof}By inductive hypothesis, the set $A$ as in the Lemma is $\Pi^0_n$ and we have

\begin{equation}
\beta \in B \iff \exists m (m, \beta )\in A,
\end{equation}

which implies $B$ is $\Sigma^0_{n+1}$.

\end{proof}\subsubsection{Relativization}

Using relativized computations via oracles, we can define a relativized version of the effective Borel hierarchy. This way we can capture \textit{all} Borel sets of finite order, not just the ones with computable codes.

\begin{definition}Let $\gamma \in \Baire$. A set $A \subseteq \Baire$ is

\begin{itemize}
\item \textbf{(a)} $\Sigma^0_1(\gamma)$ if there exists a predicate $R(x)$ computable in $\gamma$ such that
\end{itemize}
\begin{equation*}
\alpha \in A \iff \exists n \: R(\alpha\Rest{n}),
\end{equation*}
\begin{itemize}
\item \textbf{(b)} $\Pi^0_n(\gamma)$ if $\Co{A}$ is $\Sigma^0_n(\gamma)$,
\item \textbf{(c)} $\Sigma^0_{n+1}(\gamma)$ if there exists a $\Pi^0_n(\gamma)$ set $P$ such that
\end{itemize}
\begin{equation*}
\alpha \in A \iff \exists n \;[(n,\alpha) \in P].
\end{equation*}
\end{definition}A straightforward relativization gives the following analogue of Proposition~\ref{prop-computable-codes}.

\begin{proposition}\label{prop-relative-codes}Let $A \subseteq \Baire$ and $\gamma \in \Baire$. Then

\begin{quote}
$A$ is $\Sigma^0_n(\gamma)$ ($\Pi^0_n(\gamma)$) if and only if $A$ has a $\bSigma^0_n$ ($\bPi^0_n$) code computable in $\gamma$.
\end{quote}

\end{proposition}We can now present the \textbf{fundamental theorem of effective descriptive set theory}.

\begin{theorem}\label{thm-fundamental}A set $A \subseteq \Baire$ is $\bSigma^0_n$ ($\bPi^0_n$) if and only if it is $\Sigma^0_n(\gamma)$ $(\Pi^0_n(\gamma))$ for some $\gamma \in \Baire$.

\end{theorem}\begin{proof}If $A$ is $\bSigma^0_n$, then by Proposition~\ref{prop-Borel-codes} it has a $\bSigma^0_n$-code $\gamma$, and by Proposition~\ref{prop-relative-codes}, $A$ is $\Sigma^0_n(\gamma)$. The other direction follows immediately from Proposition~\ref{prop-relative-codes}.

The argument for $\bPi^0_n$ is completely analogous.

\end{proof}\subsubsection{Definability in Arithmetic}

One of the fundamental insights of computability theory is the close relation between computability and definability in arithmetic. The recursively enumerable subsets of $\Nat$ are precisely the sets $\Sigma_1$-definable over the standard model of arithmetic, $(\Nat,+,\cdot,0,1)$, and \textbf{Post's Theorem} uses this result to establish a rigid connection between levels of arithmetical complexity and computational complexity.

As indicated above, we can use this relation to give a characterization of the Borel sets of finite order in terms of definability. Since we are dealing with subsets of $\Baire$, that is, with sets of functions on $\Nat$ rather than just functions on $\Nat$, we will work in the framework of \textbf{second order arithmetic}.

The
\textbf{language of second order arithmetic} has two kinds of variables: \textbf{number variables} $x,y,z, \dots$ (and sometimes $k,l,m,n$ if they are not used as metavariables), to be interpreted as elements of $\Nat$, and \textbf{function variables} $\alpha,\beta,\gamma,\dots$, intended to range over functions from $\Nat$ into $\Nat$, i.e. elements of Baire space, i.e. reals. The non-logical symbols are the binary function symbols $+,\cdot$, the binary relation symbol $<$, the \textbf{application function} symbol $\Ap$, and the constants $\underline{0}, \underline{1}$.  \textbf{Numerical terms} are defined in usual way using $+,\cdot,\underline{0},\underline{1}$, and involve only number variables. \textbf{Atomic formulas} are $t_1 = t_2$, $t_1 < t_2$, and $\Ap(\alpha,t_1) = t_2$, where $t_1, t_2$ are numerical terms.

The
\textbf{standard model of second order arithmetic} is the structure

\begin{equation}
\mathcal{A}^2 = (\Nat, \Baire, \Ap, +, \cdot, <, 0, 1),
\end{equation}

where $+$ and $\cdot$ are the usual operations on natural numbers, $<$ is the standard ordering of $\Nat$. The two domains are connected by the binary operation $\Ap: \Baire \times \Nat \to \Nat$, defined as

\begin{equation}
\Ap(\alpha,x) = \alpha(x).
\end{equation}

A relation $R \subseteq \Nat^m \times (\Baire)^n$ is \textbf{definable over $\mathcal{A}^2$} if there exists a formula $\varphi$ of second order arithmetic such that for any $x_1, \dots, x_m \in \Nat$ and $\alpha_1, \dots \alpha_n \in \Baire$,

\begin{equation}
R(x_1, \dots, x_m, \alpha_1, \dots \alpha_n) \quad \text{iff} \quad \mathcal{A}^2 \models \varphi[x_1, \dots, x_m, \alpha_1, \dots \alpha_n].
\end{equation}

\begin{theorem}\label{lightface-definability}A set $A \subseteq \Baire$ is $\Sigma^0_n$ $(\Pi^0_n)$ if and only if it is definable over $\mathcal{A}^2$ by a $\Sigma^0_n$ $(\Pi^0_n)$ formula.

\end{theorem}Here, $\Sigma^0_n$ $(\Pi^0_n)$ formula means that we can \textbf{only quantify over number variables}, as opposed to $\Sigma^1_n$ $(\Pi^1_n)$ formulas, where we can also quantify over function variables.

The proof is a straightforward extension of the standard argument for subsets of $\Nat$.

To formulate the fundamental Theorem~\ref{thm-fundamental} in terms of definability, we need the concept of \textbf{relative definability}. We add a new constant function symbol $\underline{\gamma}$ to the language. Given a function $\gamma$, a relation is
\textbf{definable in $\gamma$} if it is definable over the structure

\begin{equation}
\mathcal{A}^2(\gamma) = (\Nat, \Baire, \Ap, +, \cdot, <, 0, 1, \gamma),
\end{equation}

where the symbol $\underline{\gamma}$ is interpreted as $\gamma$.

Then the following holds.

\begin{theorem}\label{thm-borel-arith}A set $A \subseteq \Baire$ is $\bSigma^0_n$ $(\bPi^0_n)$ if and only if it is definable in $\gamma$ by a $\Sigma^0_n$ $(\Pi^0_n)$ formula, for some $\gamma \in \Baire$.

\end{theorem}This theorem facilitates the description of Borel sets considerably. As an example, consider the set

\begin{equation}
A = \{ \alpha \colon \text{$\alpha$ eventually constant} \}.
\end{equation}

We have

\begin{equation}
\alpha \in A \iff \exists n \forall m [ m \geq n \: \Rightarrow \: \alpha(n) = \alpha(m) ]
\end{equation}

The right hand side is a $\Sigma^0_2$-formula. Hence the set $A$ is $\Sigma^0_2$.