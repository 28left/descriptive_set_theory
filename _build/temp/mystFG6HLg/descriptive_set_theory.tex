% Created with jtex v.1.0.20
\documentclass[a4paper,11pt,oneside]{book}
\usepackage[top=2cm, bottom=2cm, left=2cm, right=2cm]{geometry}
\usepackage[T1]{fontenc}
\usepackage[utf8]{inputenc}
\usepackage{lmodern}
\usepackage{graphicx}
\usepackage{caption}
\usepackage{natbib}
\usepackage{xcolor}
\usepackage{changepage}
\usepackage{framed}
\usepackage{hyperref}
\usepackage{amssymb}
\bibliographystyle{abbrvnat}

% Make list items more compact
\usepackage{enumitem}
\setlist[itemize]{noitemsep, topsep=0pt}

%%%%%%%%%%%%%%%%%%%%%%%%%%%%%%%%%%%%%%%%%%%%%%%%%%
%%%%%%%%%%%%%%%%%%%%  imports  %%%%%%%%%%%%%%%%%%%
\usepackage{amsmath}
\usepackage{amsthm}
\usepackage{url}
%%%%%%%%%%%%%%%%%  math commands  %%%%%%%%%%%%%%%%
\newcommand{\R}{\mathbb{R}}
\newcommand{\Op}[1]{\operatorname{#1}}
\newcommand{\Ord}{\operatorname{Ord}}
\newcommand{\Real}{\mathbb{R}}
\newcommand{\eps}{\varepsilon}
\newcommand{\C}{\mathbb{C}}
\newcommand{\Ci}{\mathbb{T}}
\newcommand{\N}{\mathbb{N}}
\newcommand{\Nat}{\mathbb{N}}
\newcommand{\Cant}{2^{\Nat}}
\newcommand{\Baire}{\Nat^{\Nat}}
\newcommand{\Str}{2^{<\Nat}}
\newcommand{\Sle}{\subset}
\newcommand{\Sleq}{\subseteq}
\newcommand{\Cyl}[1]{N_{#1}}
\newcommand{\Ury}{\mathbb{U}}
\newcommand{\Q}{\mathbb{Q}}
\newcommand{\Lip}{\Op{Lip}}
\newcommand{\diam}{\mathrm{diam}}
\newcommand{\Cl}[1]{\overline{#1}}
\newcommand{\Rest}[1]{\mid_{#1}}
\newcommand{\Conc}{\mbox{}^\frown}
\newcommand{\Co}[1]{\neg \, #1}
\newcommand{\Sgeq}{\supseteq}
\newcommand{\Sgr}{\supset}
\newcommand{\Estr}{\varnothing}
\newcommand{\lex}{\Op{lex}}
\newcommand{\KB}{\Op{KB}}
\newcommand{\Nstr}{\Nat^{<\Nat}}
\newcommand{\Pow}{\mathcal{P}}
\newcommand{\bPi}{\pmb{\Pi}}
\newcommand{\Integer}{\mathbb{Z}}
\newcommand{\bSigma}{\pmb{\Sigma}}
\newcommand{\Rat}{\mathbb{Q}}
\newcommand{\Tup}[1]{\langle #1 \rangle}
\newcommand{\Ap}{\Op{Ap}}
\newcommand{\Qu}{\mathsf{Q}}
\newcommand{\bDelta}{\pmb{\Delta}}
\newcommand{\BP}[1]{\bPi^0_{#1}}
\newcommand{\BS}[1]{\bSigma^0_{#1}}
\newcommand{\PS}[1]{\bSigma^{1}_{#1}}
\newcommand{\PP}[1]{\bPi^1_{#1}}
\newcommand{\ZF}{\mathsf{ZF}}
\newcommand{\AC}{\mathsf{AC}}
\newcommand{\DC}{\mathsf{DC}}
\newcommand{\CH}{\mathsf{CH}}
\newcommand{\W}{\Op{W}}
\newcommand{\WF}{\Op{WF}}
\newcommand{\WOrd}{\Op{WOrd}}
\newcommand{\Norm}[1]{\parallel \! #1 \!\parallel}
\newcommand{\ZFC}{\mathsf{ZFC}}
\newcommand{\V}{\Op{V}}
\newcommand{\GN}[1]{\ulcorner #1 \urcorner}
\newcommand{\Const}[1]{\underline{#1}}
\newcommand{\VL}{\mathsf{V=L}}
%%%%%%%%%%%%%%%%%%%%%%%%%%%%%%%%%%%%%%%%%%%%%%%%%%
%%%%%%%%%%%%%%%%%%%%%%%%%%%%%%%%%%%%%%%%%%%%%%%%%%
%%%%%%%%%%%%%%%%%%%%  theorem  %%%%%%%%%%%%%%%%%%%
\newtheorem{theorem}{Theorem}[section]
\newtheorem{corollary}{Corollary}[theorem]
\newtheorem{lemma}[theorem]{Lemma}
\newtheorem{proposition}{Proposition}[section]
\newtheorem{definition}{Definition}[section]
\newtheorem{example}{Example}[section]
\newtheorem{remark}{Remark}[section]
\newtheorem{axiom}{Axiom}[section]
\newtheorem{conjecture}{Conjecture}[section]
\newtheorem{observation}{Observation}[section]
%%%%%%%%%%%%%%%%%%%%%%%%%%%%%%%%%%%%%%%%%%%%%%%%%%




\hypersetup{
  colorlinks,
  linkcolor={black},
  citecolor={black},
  urlcolor={black}
}

% Style quotes
\definecolor{darkblue}{rgb}{0.0, 0.0, 0.55}
\definecolor{quoteshade}{rgb}{0.95, 0.95, 1}
\renewenvironment{quote}{%
  \def\FrameCommand{%
    \hspace{1pt}%
    {\color{darkblue}\vrule width 2pt}%
    {\color{quoteshade}\vrule width 4pt}%
    \colorbox{quoteshade}%
  }%
  \MakeFramed {\advance\hsize-\width \FrameRestore}%
  \noindent\hspace{-8pt}% disable indenting first paragraph
  \begin{adjustwidth}{0pt}{0pt}% adjust as needed
  \vspace{2pt}\vspace{2pt}%
}
{%
  \vspace{2pt}\end{adjustwidth}\endMakeFramed%
}

\title{\Huge \textbf{Descriptive Set Theory}}
\author{\textsc{}}

\begin{document}
\sloppy

\frontmatter
\maketitle

\tableofcontents
% \listoffigures
% \listoftables

\mainmatter

% \include{preface.tex}
% \include{acknowledgements.tex}

\section{Introduction: Perfect Subsets of the Real Line}

Descriptive set theory nowadays is understood as the study of definable subsets of Polish Spaces. Many of its problems and techniques arose out of efforts to answer basic questions about the real numbers. A prominent example is the \textit{Continuum Hypothesis} ($\CH$):

\begin{framed}
\textbf{Continuum Hypothesis (Cantor, 1890s)}\\
If $A \subseteq \Real$ is uncountable, then there exists a bijection between $A$ and $\Real$. That is, is every uncountable subset of $\Real$ is of the same cardinality as $\Real$.
\end{framed}

Early approaches tried to show that $\CH$ holds for a number of sets with an easy topological structure.

For closed sets, the situation is less clear. Given a set $A \subseteq \Real$, we call $x \in \Real$ a \textbf{limit point} of $A$ if

\begin{equation}
\forall \epsilon > 0 \: \exists z \in A \: [z \neq x \: \& \: z \in U_\eps(x)],
\end{equation}

where $U_\eps(x)$ denotes the standard $\eps$-neighborhood of $x$ in $\Real$

\begin{definition}\label{def-perfect}A non-empty set $P \subseteq \Real$ is \textbf{perfect} if it is closed and every point of $P$ is a limit point.

\end{definition}In other words, a perfect set is a closed set that has no isolated points. We can also deduce that for a perfect set $P$, every neighborhood of a point $p \in P$ contains infinitely many points from $P$.

\begin{figure}[!htbp]
\centering
\includegraphics[width=0.7\linewidth]{files/Cantor_set-7f15a3eb647d25cf38092c5cd78d7432.png}
\caption*{Cantor set}
\end{figure}

Obviously, $\Real$ itself is perfect, as is any closed interval in $\Real$. There are totally disconnected perfect sets, such as the \textbf{middle-third Cantor set} in $[0,1]$

\begin{framed}
\textbf{Theorem (Cantor, 1884)}\\
A perfect subset of $\Real$ has the same cardinality as $\Real$.
\end{framed}

\begin{framed}
\textbf{Hint}\\
\begin{itemize}
\item Argue it suffices to construct an injection from $\Cant$ (the set of all infinite binary sequences) into the perfect set.
\item Start with any point $x$ in the perfect set and open neighborhood $U_\eps(x)$. Use the perfect set property to find two points $x_0, x_1$ distinct from $x$ and each other in $U_\eps(x)$.
\item These points will `guide' the mapping: All sequences in $\Cant$ starting with 0 will be mapped to a point close to $x_0$, while all sequences starting with 1 will be a mapped to a point close to $x_1$.
\item Now iterate with $x_0$ and $x_1$ in place of $x$.
\end{itemize}
\end{framed}

\begin{proof}Let $P \subseteq \Real$ be perfect. We construct an injection from the set $\Cant$ of all infinite binary sequences into $P$. An infinite binary sequence $\xi = \xi_0 \xi_1 \xi_2 \dots$ can be identified with a real number $\in [0,1]$ via the mapping

\begin{equation}
\xi \mapsto \sum_{i \geq 0} \xi_i 2^{-i-1}.
\end{equation}

Note that this mapping is onto. It follows that the cardinality of $P$ is at least as large as the cardinality of $[0,1]$. The \href{https://en.wikipedia.org/wiki/Schr\%C3\%B6der\%E2\%80\%93Bernstein\_theorem}{Schröder-Bernstein Theorem} (for a proof see e.g. \cite{jech2003a}) implies that $|P| = |\R|$.

To construct the desired injection, choose $x \in P$ and let $\eps_0 = 1 = 2^0$. Since $P$ is perfect, $P \cap U_{\eps_0}(x)$ is infinite. Let $x_0 \neq x_1$ be two points in $P \cap U_{\eps_0}(x)$, distinct from $x$. Let $\eps_1$ be such that $\eps_1 \leq 1/2$, $U_{\eps_1}(x_0), U_{\eps_1}(x_1) \subseteq U_{\eps_0}(x)$, and $\overline{U_{\eps_1}(x_0)} \cap \overline{U_{\eps_1}(x_1)} = \emptyset$, where $\overline{U}$ denotes the closure of $U$.

We can iterate this procedure recursively with smaller and smaller diameters, using the fact that $P$ is perfect. This gives rise to a so-called \textbf{Cantor scheme}, a family of open balls $(U_\sigma)$ satisfying certain \textit{nesting conditions}. Here the index $\sigma$ is a finite binary sequence, also called a \textit{string}. A Cantor scheme is defined by the following properties.

\begin{enumerate}
\item $\diam(U_\sigma) \leq 2^{ -|\sigma|}$, where $|\sigma|$ denotes the length of $\sigma$.
\item If $\tau$ is a proper extension of $\sigma$, then $U_\tau \subset U_\sigma$.
\item If $\tau$ and $\sigma$ are \textit{incompatible} (i.e. neither extends the other), then
\begin{equation*}
U_\tau \cap U_\sigma = \emptyset.
\end{equation*}

\item The center of each $U_\sigma$, call it $x_\sigma$, is in $P$.
\end{enumerate}

\begin{figure}[!htbp]
\centering
\includegraphics[width=0.7\linewidth]{files/Cantor_Scheme-0b52d5260abb5f7f4d0a22abe7957d4c.png}
\caption[]{Nested structure of a Cantor scheme}
\label{Cantor Scheme}
\end{figure}

Let $\xi$ be an infinite binary sequence. Given $n \geq 0$, we denote by $\xi\Rest{n}$ the string formed by the first $n$ bits of $\xi$, i.e.

\begin{equation}
\xi\Rest{n} = \xi_0 \xi_1 \dots \xi_{n-1}.
\end{equation}

The finite initial segments give rise to a sequence $x_{\xi\Rest{n}}$ of centers. By properties (1.) and (2.), this is a Cauchy sequence. By (4.), the sequence lies in $P$. Since $P$ is closed, the limit $x_\xi$ is in $P$. By (3.), the mapping $\xi \mapsto x_\xi$ is well-defined and injective.

\end{proof}Thus, to show that a set of reals has the same cardinality as $\R$, it suffices to show the set contains a perfect subset. The next theorem establishes that the Continuum Hypothesis holds for all closed subsets of $\R$.

\begin{framed}
\textbf{Cantor-Bendixson Theorem}\\
Every uncountable closed subset of $\Real$ contains a perfect subset.
\end{framed}

\begin{framed}
\textbf{Hint}\\
Consider the set of \textbf{condensation points}, i.e. the set of all points for which any open neighborhood has \textit{uncountable} intersection with the given closed set.
\end{framed}

\begin{proof}Let $C \subseteq \Real$ be uncountable and closed. We say $z \in \Real$ is a \textit{condensation point} of $C$ if

\begin{equation}
\forall \eps > 0 \:[ U_\eps(z) \cap C \text{ uncountable}].
\end{equation}

Let $D$ be the set of all condensation points of $C$. Note that $D \subseteq C$, since every condensation point is clearly a limit point and $C$ is closed.

Furthermore, we claim that $D$ is perfect. Clearly $D$ is closed. Suppose $z \in D$ and $\eps > 0$. Then $U_\eps(z) \cap C$ is uncountable. We would like to conclude that $U_\eps(z) \cap D$ is uncountable, too, since this would mean in particular that $U_\eps(z) \cap D$ is infinite. The conclusion holds if $C \setminus D$ is countable.

To show that $C\setminus D$ is countable, assume that $y \in C \setminus D$. Then, for some $\delta > 0$, $U_\delta(y) \cap C$ is countable. We can find and interval $I(y) \subseteq U_\delta(y)$ that contains $y$ and has rational endpoints. There are at most countably many intervals with rational endpoints and hence for each $y \in C \setminus D$ there are at most countably many choices for $I(y)$. Thus, we have

\begin{equation}
C\setminus D \subseteq \bigcup_{y \in C \setminus D} I_y \cap C.
\end{equation}

The right hand side is a countable union of countable sets, hence countable.

\end{proof}We will later encounter an alternative (more constructive) proof that gives additional information about the complexity of the closed set $C$. For now we conclude with the fact we were aiming to prove in this lecture.

\begin{corollary}Every closed subset of $\Real$ is either countable or of the cardinality of the continuum.

\end{corollary}The results of this lecture give us a blueprint on how to verify the Continuum Hypothesis for a given family $\mathcal{F}$ of sets (of reals):

A family $\mathcal{F}$ of sets (of reals) has the \textbf{perfect set property} if every set in $\mathcal{F}$ is either countable or has a perfect subset.

\begin{framed}
\textbf{Question}\\
Which families of sets have the perfect set property?
\end{framed}

\section{Ordinals and cardinals}

\include{descriptive_set_theory-ordinals}

\section{Polish spaces}

\subsection{Polish Spaces}

The proofs in the \href{/perfect-subsets-r}{introduction section} are quite general, that is, they make little use of specific properties of $\Real$. If we scan the arguments carefully, we see that we can replace $\Real$ by any metric space that is \textbf{complete and contains a countable basis of the topology}.

\subsubsection{Review of some concepts from topology}

\paragraph{Basis}

Let $(X, \mathcal{O})$ be a topological space. A family $\mathcal{B} \subseteq \mathcal{O}$ of subsets if $X$ is a \textbf{basis} for the topology if every open set from $\mathcal{O}$ is the union of elements of $\mathcal{B}$. For example, the open intervals with rational endpoints form a basis of the standard topology of $\Real$. A family $\mathcal{S} \subseteq \mathcal{O}$ is a \textbf{subbasis} if the set of finite intersections of sets in $\mathcal{S}$ is a basis for the topology.

Finally, if $\mathcal{S}$ is any family of subsets of $X$, the \textbf{topology generated by $\mathcal{S}$} is the smallest topology on $X$ containing $\mathcal{S}$. It consists of all unions of finite intersections of sets in $\mathcal{S} \cup \{X,\emptyset\}$.

\paragraph{Density}

A set $D \subset X$ is \textbf{dense} if for any open $U \neq \emptyset$ there exists $z \in D \cap U$. If a topological space $(X, \mathcal{O})$ has a countable dense subset, the space is called \textbf{separable}.

\paragraph{Products}

If $(X_i)_{i \in I}$ is a family of topological spaces, one defines the \textbf{product topology} on $\Pi_{i \in I} X_i$ to be the topology generated by the sets $\pi_i^{ -1}(U)$, where $i \in I$, $U \subseteq X_i$ is open, and $\pi_i: \Pi_{i \in I} X_i \to X_i$ is the $i$th projection.


\bigskip
\centerline{\rule{13cm}{0.4pt}}
\bigskip

Now suppose $(X,d)$ is a metric space. With each point $x \in X$ and every $\eps > 0$ we associate an \textbf{$\eps$-neighborhood} or \textbf{$\eps$-ball}

\begin{equation}
U_\eps(x) = \{y \in X \colon d(x,y)<\eps\}.
\end{equation}

The topology generated by the $\eps$-neighborhoods is called the \textit{topology of the metric space} $(X,d)$. If this topology agrees with a given topology $\mathcal{O}$ on $X$, we say the metric $d$ is \textbf{compatible} with the topology $\mathcal{O}$. If for a topological space $(X, \mathcal{O})$ there exists a compatible metric, $(X, \mathcal{O})$ is called \textbf{metrizable}.

If a topological space $(X,\mathcal{O})$ is separable and metrizable, then the balls with center in a countable dense subset $D$ and rational radius form a \textit{countable base of the topology}.

\subsubsection{Polish spaces -- the basics}

\begin{definition}\label{def-polish}A \textbf{Polish space} is a separable topological space $X$ for which exists a compatible metric $d$ such that $(X,d)$ is a complete metric space.

\end{definition}There may be many different compatible metrics that make $X$ complete. If $X$ is already given as a complete metric space with countable dense subset, then we call $X$ a \textit{Polish metric space}.

The standard example is, of course, $\Real$, the set of real numbers. One can obtain other Polish spaces using the following basic observations. (We leave the proof as an exercise.)

\begin{proposition}\label{properties-polish}\begin{enumerate}
\item A closed subset of a Polish space is Polish.
\item The product of a countable (in particular, finite) sequence of Polish spaces is Polish.
\item Any topological space homeomorphic to a Polish space is Polish.
\end{enumerate}

\end{proposition}We conclude that $\Real^n$, $\C$, $\C^n$, the unit interval $[0,1]$, the unit circle $\Ci = \{z \in \C \colon |z| = 1\}$, and the infinite dimensional spaces $\Real^\Nat$ and $[0,1]^\Nat$ (the \textit{Hilbert cube}) are Polish spaces.

Any countable set with the \textbf{discrete topology} is Polish, by means of the \textbf{discrete metric} $d(x,y) = 1 \: \Leftrightarrow \: x \neq y$.

Some subsets of Polish spaces are Polish but not closed.

\begin{framed}
\textbf{Exercise}\\
By choosing a suitable metric, show that $(0,1)$, the open unit interval, is a Polish space.
\end{framed}

We will later characterize all subsets of Polish spaces that are Polish themselves.

\subsubsection{Product spaces}\label{polish-product-spaces}

In a certain sense, the most important Polish spaces are of the form $A^\Nat$, where $A$ is a countable set carrying the discrete topology. The standard cases are

\begin{quote}
$\Cant$, the Cantor space $\qquad$ and $\qquad$ $\Baire$, the Baire space.
\end{quote}

We will, for now, denote elements from $A^\Nat$ by lower case greek letters from the beginning of the alphabet. The $n$-th term of $\alpha$ we denote by either $\alpha(n)$ or $\alpha_n$, whichever is more convenient.

We endow $A$ with the discrete topology.
The product topology on these spaces has a convenient characterization. Given a set $A$, let $\Str[A]$ be the sets of all finite sequences over $A$.
Given $\sigma, \tau \in A^{<\Nat}$, we write $\sigma \Sleq \tau$ to indicate that $\sigma$ is an initial segment of $\tau$. $\Sle$ means the initial segment is proper. This notation extends naturally to hold between elements of $\Str[A]$ and $A^\Nat$, $\sigma \Sle \alpha$ meaning that $\sigma$ is a finite initial segment of $\alpha$.

A basis for the product topology on $A^\Nat$ is given by the \textbf{cylinder sets}

\begin{equation}
\Cyl{\sigma} = \{\alpha \in A^\Nat \colon \sigma \Sle \alpha \},
\end{equation}

that is, the set of all infinite sequences extending $\sigma$. The complement of a cylinder is a union of cylinders and hence open. Therefore, each set $\Cyl{\sigma}$ is clopen.

A compatible metric is given by

\begin{equation}
d(\alpha,\beta) = \begin{cases}
    2^{-N} & \text{ where } N \text{ is least such that $\alpha_N \neq \beta_N$ }\\
    0 & \text{ if $\alpha = \beta$}.
\end{cases}
\end{equation}

The representation of the topology via cylinders (which are characterized by finite objects) allows for a combinatorial treatment of many questions and will be essential later on.

\begin{proposition}[Topological properties of ]\label{prop-topological-product}\textbf{$A^\Nat$}\\
Let $A$ be a countable set, equipped with the discrete topology. Suppose $A^\Nat$ is equipped with the product topology. Then the following hold.

\begin{enumerate}
\item $A^\Nat$ is Polish.
\item $A^\Nat$ is \textit{zero-dimensional}, i.e. it has a basis of \textit{clopen} sets.
\item $A^\Nat$ is compact if and only if $A$ is finite.
\end{enumerate}

\end{proposition}Via the mapping

\begin{equation}
\alpha \mapsto \sum_{i = 0}^\infty \frac{2\alpha_i}{3^{i+1}},
\end{equation}

$\Cant$ is homeomorphic to the middle-third Cantor set in $\Real$, whereas the \textbf{continued fraction} mapping

\begin{equation}
\beta \mapsto \beta_0 + \cfrac{1}
        {\beta_1 + \cfrac{1}{
                \beta_2 + \cfrac{1}{
                        \beta_3 + \ldots}}}
\end{equation}

provides a homeomorphism between $\Z \times (\N\setminus\{0\})^\N$ and the irrational real numbers.

The universal role played by the discrete product spaces is manifested in the following results.

\begin{theorem}\label{thm-cantor-embedding}Every uncountable Polish space contains a homeomorphic embedding of Cantor space $\Cant$.

\end{theorem}The proof is similar to the proof of Theorem~(Cantor,~1884). Note that the proof actually constructs an embedding of $\Cant$. The continuity of the mapping is straightforward.

In a similar way we can adapt the proof of Cantor-Bendixson~Theorem to show that the \textit{perfect subset property} holds for closed subsets of Polish spaces.

\begin{theorem}[Cantor-Bendixson Theorem for Polish spaces]\label{thm-cantor_bendixson-polish}Every uncountable closed subset of a Polish space contains a perfect subset.

\end{theorem}Finally, we can characterize Polish spaces as continuous images of Baire space.

\begin{theorem}\label{thm-polish-cont-image-baire}Every Polish space $X$ is the continuous image of $\Baire$.

\end{theorem}\begin{proof}Let $d$ be a compatible metric on $X$, and let $D = \{x_i \colon i \in \Nat\}$ be a countable dense subset of $X$. Every point in $X$ is the limit of a sequence in $D$. Define a mapping $g:\Baire \to X$ by putting

\begin{equation}
\alpha = \alpha(0)\, \alpha(1)\, \alpha(2)\dots \mapsto \lim_n x_{\alpha(n)}.
\end{equation}

The problem is, of course, that the limit on the right hand side not necessarily exists. We have to proceed more carefully.
Given $\alpha \in \Nat$, we put $y^\alpha_0 = x_{\alpha(0)}$ and
define iteratively

\begin{equation}
y^\alpha_{n+1} = \begin{cases}
    x_{\alpha(n+1)} & \text{ if $d(y^\alpha_n,x_{\alpha(n+1)}) < 2^{-n}$}, \\
    y^\alpha_n & \text{ otherwise }.
\end{cases}
\end{equation}

The resulting sequence $(y^\alpha_n)$ is clearly Cauchy in $X$, and hence converges to some point $y^\alpha \in X$, by completeness. We define

\begin{equation}
f(\alpha) = y^\alpha.
\end{equation}

$f$ is continuous, since if $\alpha$ and $\beta$ agree up to length $N$ (that is, their distance is at most $2^{ -N}$ with respect to the above metric), then the sequences $(y^\alpha_n)$ and $(y^\beta_n)$ will agree up to index $N$, and all further terms are within $2^{ -N}$ of $y^\alpha_N$ and $y^\beta_N$, respectively.

Finally, since $D$ is dense in $X$, $f$ is a surjection.

\end{proof}

\subsection{Excursion: The Urysohn Space}

Recall that a mapping $f: X \to Y$ between two metric spaces $(X,d_X)$ and $(Y,d_Y)$ is an \textbf{isometry} if

\begin{equation}
d_Y(f(x),f(y)) = d_X(x,y) \quad \text{ for all $x,y \in X$},
\end{equation}

that is, an isometry is a mapping that preserves distances. The function $f$ is also called an \textit{isometric embedding} of $X$ into $Y$. $X$ and $Y$ are \textit{isometric} if there exists a bijective isometry between them.

\subsubsection{Universal spaces}

\begin{theorem}There exists a Polish metric space $\Ury$ such that every Polish metric space isometrically embeds into $\Ury$.

\end{theorem}A concrete example of such a space is $\mathcal{C}[0,1]$.

\begin{framed}
\textbf{Exercise}\\
Show that the set $\mathcal{C}[0,1]$ of all continuous, real-valued functions on $[0,1]$ with the metric

\begin{equation}
d(f,g) = \sup\{|f(x) - g(x)| \colon x \in [0,1] \}
\end{equation}

contains an isomorphic copy of any Polish metric space.
\end{framed}

But this example is not quite what we have in mind here. There exists another space with a stronger, more ``intrinsic'' universality property. This space was first constructed by Pavel Urysohn in 1927 \cite{Urysohn:1927a}.

The construction features an \textbf{amalgamation principle} that has surfaced in other areas like model theory or graph theory.

\subsubsection{Extensions of finite isometries and Urysohn universality}

Suppose $X$ is a Polish metric space. Let $D = \{x_1, x_2, \dots\}$ be a countable, dense subset. We first observe that, to isometrically embed $X$ into another Polish space, it is sufficient to embed $D$.

\begin{lemma}If $Y$ is Polish, then any isometric embedding $e$ of $D$ into $Y$ extends to an isometric embedding $e^*$ of $X$ into $Y$.

\end{lemma}\begin{proof}Given $z \in X$, let $(x_{i_n})$ be a sequence in $D$ converging to $z$. Since $(x_{i_n})$ converges, it is Cauchy.

$e$ is an isometry, and thus $y_n := e(x_{i_n})$ is Cauchy, and since $Y$ is Polish, $(y_n)$ converges to some $y \in Y$. Put $e^*(z) = y$.

To see that this mapping is well-defined, let $(x_{j_n})$ be another sequence with $x_{j_n} \to z$. Then $d(x_{i_n}, x_{j_n}) \to 0$, and hence $d(e(x_{i_n}),e(x_{j_n}) = d(y_n, e(x_{j_n}))\to 0$, implying $e(x_{j_n}) \to y$.

Furthermore, suppose $w = \lim x_{k_n}$ is another point in $X$. Then (since a metric is a continuous mapping from $Y\times Y \to \Real$)

\begin{equation}
d(e^*(z), e^*(w)) = \lim d( e(x_{i_n}),  e(x_{k_n})) = \lim d(  x_{i_n}, x_{k_n}) = d(z,w).
\end{equation}

Thus $e^*$ is an isometry.

\end{proof}In order to embed $D$, we can now exploit the inductive structure of $\Nat$ and reduce the task to extending finite isometries.

Suppose we have constructed an isometry $e$ between $F_N = \{x_1, \dots, x_N \} \subset D$ and a space $Y$. We would like to extend the isometry to include $x_{N+1}$. For this we have to find an element $y \in Y$ such that for all $i \leq N$

\begin{equation}
d_Y(y, e(x_i)) = d_X(x_{N+1}, x_i).
\end{equation}

This extension property gives rise to the following definition.

\begin{definition}A Polish metric space $(Y,d_Y)$ is \textbf{Urysohn universal} if for every finite subspace $F \subset Y$ and any extension $F^* = F \cup \{x^*\}$ with metric a $d^*$ such that

\begin{equation}
d^*|_{F\times F} = d_Y|_{F\times F},
\end{equation}

there exists a point $u \in Y$ such that

\begin{equation}
d_{Y}(u,x) = d^*(x^*,x) \quad \text{ for all $x \in F$}.
\end{equation}

\end{definition}As outlined above, the extension property of Urysohn universal spaces implies the desired isometric embedding property.

\begin{proposition}\label{prop-urysohn-embedding}Let $U$ be a Urysohn universal Polish metric space. For any Polish metric space $(X,d)$ there exists an isometric embedding from $X$ into $U$.

\end{proposition}But the extension property also implies a strong intrinsic extension property for the Urysohn space itself.

\begin{proposition}\label{prop-urysohn-extension}Let $U$ be a Urysohn universal Polish metric space. Every isometry between finite subsets of $\Ury$ extends to an isometry of $U$ onto itself.

\end{proposition}The proof applies the \href{https://en.wikipedia.org/wiki/Back-and-forth\_method}{Back-and-forth method} that you may know from the rationals: every order-isomorphism between finite subsets of $\Q$ extends to an automorphism of $(\Q,<)$.

This property (which can be formulated for structures in general) is also known as \textbf{homogeneity}. It plays an important role, for example, in
model theory \cite{Macpherson:2011a} and in the topological dynamics of automorphism groups of countable structures \cite{Kechris-Pestov-Todorcevic:2005a}.

\begin{framed}
\textbf{Exercise}\\
Show that any two Urysohn universal spaces are isometric.
\end{framed}

We will prove the existence of this unique Polish space, which we denote by $\Ury$, in the following sections.

\subsubsection{Constructing the Urysohn space -- a first approximation}

We first give a construction of a space that has the extension property, but is not Polish. After that we will take additional steps to turn it into a Polish space.

The crucial idea is to observe that if $X$ is a metric space and $x \in X$, then the mapping $f_{x}: X \to \Real^{\geq 0}$ given by

\begin{equation}
f_{x}(y) = d_X(x,y)
\end{equation}

is 1-Lipschitz. Recall that a function $g$ between metric spaces $X$ and $Y$  is \textbf{$L$-Lipschitz}, $L > 0$ if for every $x,y \in X$,

\begin{equation}
d(g(x),g(y)) \leq L \, d(x,y).
\end{equation}

Let $\Lip_1(X)$ be the set of 1-Lipschitz mappings from $X$ to $\Real$. We endow $\Lip_1(X)$ with the supremum metric

\begin{equation}
d(f,g) = \sup \{|f(x) - g(x)| \colon x \in X \}.
\end{equation}

If $\diam(X) \leq \mathrm{d}$ and $f,g$ are 1-Lipschitz, then $d(f,g)$ is indeed finite.
However, we will later  need that the resulting space is also bounded. Let  $\Lip^{\mathrm{d}}_1(X)$ be the space of all 1-Lipschitz functions from $X$ to $[0,\mathrm{d}]$.

Clearly, $\diam(\Lip^{\mathrm{d}}_1(X)) \leq \mathrm{d}$.

With this metric, the mapping $x \mapsto f_{x}(y) = d(x,y)$ becomes an isometry: We have

\begin{equation}
d(f_{x}, f_{z}) = \sup\{ | d(x,y) - d(z,y)| \colon y \in X \}.
\end{equation}

By the reverse triangle inequality, this is always $\leq d(x,z)$. On the other hand, setting $y=z$ yields $d(f_x,f_z) \geq d(x,z)$. This embedding of $X$ into $\Lip^{\mathrm{d}}_1(X)$ is called the \textbf{Kuratowski embedding}.

We use this fact as follows: If $X^* = X \sqcup \{x^*\}$ and $d^*$ is an extension of $d_X$, then $f_{x^*}$ is an element of $\Lip^{\mathrm{d}}_1(X)$, and as above, for any $x \in X$

\begin{equation}
d(f_{x^*}, f_x) = d^*(x^*,x).
\end{equation}

Hence $\Lip^{\mathrm{d}}_1(X)$ has an extension property of the kind we are looking for.

\begin{quote}
\textit{Iterative construction}: Let $X_0$ be any non-empty Polish space with finite diameter $\mathrm{d} > 0$. Given $X_n$, let $\mathrm{d}(n) = \diam(X_n)$ and set $X_{n+1} = \Lip^{2\mathrm{d}(n)}_1(X_n)$. Finally, put $X_\infty = \bigcup_n X_n$. Note that $X_\infty$ inherits a well-defined metric $d$ from the $X_n$, which embed isometrically into it.
\end{quote}

We wan to verify that $X_\infty$ has the extension property needed to be Urysohn universal. Let $F$ be a finite subset of $X_\infty$. There exists $N$ such that $F \subset X_N$. Suppose $F^* = F \sqcup \{x^*\}$ and $d^*$ is an extension of $d$ to $F^*$. Let $\mathrm{d}^* = \diam(F^*)$. Note that $\diam(X_n) = 2^n \mathrm{d}$. Choose $M$ so that $M \geq N$ and $\diam(X_M) \geq \mathrm{d}^*$. The next lemma ensures that we can find $f \in X_{M+1}$ such that $f(x) = d^*(x^*,x)$ for all $x \in F$.

\begin{lemma}[McShane-Whitney]Let $X$ be a metric space with $\diam(X) \leq \mathrm d$, $A \subseteq X$, and $f \in \Lip^{\mathrm{d}}_1(A)$, then $f$ can be extended to a 1-Lipschitz function $f^*$ on all of $X$ such that

\begin{equation}
f^*|_A = f \quad \text{ and } \quad f^* \in \Lip^{2\mathrm{d}}_1(X).
\end{equation}

\end{lemma}\begin{proof}For each $a \in A$ define $f_a: X \to \Real$ as

\begin{equation}
f_a(x) = f(a) + d(a,x).
\end{equation}

Then $f_a$ is 1-Lipschitz, by the reverse triangle inequality. Let

\begin{equation}
f^*(x) = \inf \{f_a(x) \colon a \in A\}.
\end{equation}

Then $f^*(a) = f(a)$ for all $a \in A$. Let $x,y \in X$ and $\eps > 0$. Wlog assume $f^*(y) \geq f^*(x)$. Pick $a \in A$ so that $f_a(x) \leq f^*(x)   + \eps$. Then
\begin{align*}
	|f^*(x) - f^*(y)| = f^*(y) - f^*(x) & \leq f^*(y) - f_a(x) + \eps \\
		& \leq f_a(y) - f_a(x) + \eps \leq d(x,y) + \eps.
\end{align*}
Since $\eps > 0$ was arbitrary, we have $|f^*(x) - f^*(y)| \leq d(x,y)$.

Finally, we have $f(a) \leq f_a(x) \leq f(a) + \mathrm{d}$ and thus $0 \leq f^*(x) \leq f_a(x) \leq 2\mathrm{d}$.

\end{proof}\subsubsection{Finishing the construction}\label{ury-finishing-construction}

The set $X_\infty$ we constructed has two deficiencies with respect to our goal of constructing a Urysohn universal space: $X_\infty$ is not necessarily separable, and $X_\infty$ is not necessarily complete.

To make $X_\infty$ separable, we observe that if $X$ is compact, then the set $\Lip^{\mathrm{d}}_1(X)$ is closed in $\mathcal{C}(X)$ (the set of all real-valued continuous functions on $X$), bounded, and equicontinuous. By the \textbf{Arzelà-Ascoli Theorem}, $\Lip^{\mathrm{d}}_1(X)$ is compact.
Every compact metric space is separable: For every $\eps > 0$, there exists a finite covering of the space with sets of $\diam < \eps$. Letting $\eps$ traverse all positive rationals and picking a point from each set in an $\eps$-covering yields a countable dense subset. Hence if we start with $X_0$ compact, each $X_n$ will be compact, too. A countable union of separable spaces is separable, thus $X_\infty$ is separable.

To obtain a complete space, we can pass from $X_\infty$ to its \textit{completion} $\Cl{X_\infty}$. First note that if a metric space $X$ is separable, so is its completion $\Cl{X}$. However, we also have to ensure that $\Cl{X_\infty}$ retains the universality property of $X_\infty$.

\begin{lemma}If a complete metric space $(Y,d)$ admits a dense Urysohn universal subspace $\mathcal{U}$, then $Y$ is Urysohn universal.

\end{lemma}\begin{proof}We follow \cite{Gromov:1999a}. Let $F = \{x_1, \dots, x_n\} \subset Y$ and assume $F^* = F \sqcup \{x^*\}$ is an extension with metric $d^*$.

We first note that $Y$ is \textbf{approximately universal}. This means that for any $\eps > 0$, there exists a point $y^* \in Y$ such that
\begin{equation*}
\tag{$*$}
	|d(y*,x) - d^*(x^*,x)| < \eps \quad \text{ for all $x \in F$}.
\end{equation*}
This can be seen as follows. Since $\mathcal{U}$ is dense in $Y$, we can find a finite set $F_\eps = \{z_1, \dots, z_n\} \subset \mathcal{U}$ such that

\begin{equation}
d(x_i, z_i) < \eps \quad \text{ for $1 \leq i \leq n$}.
\end{equation}

Now use the Urysohn universality of $\mathcal{U}$ for the set $G^* = \{z_1, \dots , z_n\} \sqcup \{x^*\}$ with the metric

\begin{equation}
d^{**}(x^*, z_i) = d^*(x^*, x_i) \qquad (i = 1, \dots, n)
\end{equation}

to find $z \in \mathcal{U}$ with

\begin{equation}
d(z,z_i) = d^{**}(x^*,z_i) = d^*(x^*,x_i) \qquad (i = 1, \dots, n)
\end{equation}

Then, by the reverse triangle inequality,

\begin{equation}
\left | d(z,x_i) - d^*(x^*,z_i) \right | = \left | d(z,x_i) - d(z,z_i) \right | \leq d(z_i, x_i) = \eps,
\end{equation}

as required.

%  
% To keep the proof technically simple, wlog we assume $\eps$ is much smaller than the individual distances between the $x_i$. Consider the extension $F^*_\eps = F_\eps \sqcup \{x^*\}$ with metric
% 
% $$
% 	e^*(x^*, z_i) = d^*(x^*,x_i) + d(x_i,z_i).
% $$
% 
% Since $\mathcal{U}$ has the finite extension property, we can find $y^* \in \mathcal{U}$ such that
% 
% $$
% 	d(y^*,z_i) = e^*(x^*,z_i) 
% $$
% 
% Hence 
% \begin{align*}
% 	|d(y^*,x_i) - d^*(x^*,x_i)| & = | e^*(x^*,z_i) - d^*(x^*,x_i)|  \\
% 	 	& = | d^*(x^*,x_i) + d(x_i,z_i) - d^*(x^*,x_i) | <  \eps.
% \end{align*}

We use this approximate universality to construct a Cauchy sequence $(y_k)$ in $Y$ of `approximate' extension points that satisfy $(*)$ for smaller and smaller $\eps$.

Let $0 < \delta = \max \{d^*(x^*,x_i) \colon 1 \leq i \leq n \}$.
The formal requirements for the sequence $(y_i)$ are as follows.

\begin{enumerate}
\item $|d(y_k,x_i) - d^*(x^*,x_i)| \leq 2^{ -k} \delta$.
\item $d(y_{k+1},y_k) \leq 2^{ -k}\delta$.
\end{enumerate}

The sequence necessarily converges in $Y$ and the limit point must be a true extension point, due to (1.)

Suppose we have already constructed $y_1, \dots, y_k$ satisfying (1.), (2.). Add an (abstract) point $y^*_{k+1}$ to $F_k = F \cup \{y_1, \dots, y_k\}$. Let $F^*_{k+1} = F_k \sqcup \{y^*_{k+1}\}$.

We want to use approximate universality on $F^*_{k+1}$. To this end we have to define a metric $e^*$ on $F^*_{k+1}$ that has the following properties

\begin{align*}
	(i) 	\qquad	& e^*|_{F_k} = d|_{F_k} \\
	(ii)  	\qquad & e^*(y^*_{k+1},x_i) = d^*(x^*,x_i) \quad (1 \leq i \leq n) \\
	(iii)	\qquad 	& e^*(y^*_{k+1}, y_k) = 2^{-k-1}\delta 
\end{align*}

Indeed such a metric exists: The condition $(i)$ already defines a metric on the set $F_k$. The conditions $(i)$-$(iii)$ also define a metric on $F \cup \{y_k,y^*_{k+1}\}$ -- the only thing to check for this is the triangle inequality for $y_k, y^*_{k+1}$:

\begin{equation}
|e^*(x_i,y_k) - e^*(y^*_{k+1},x_i)| = |d(x_i,y_k) - d^*(x^*,x_i) | \leq 2^{-k}\delta = e^*(y_k, y^*_{k+1}),
\end{equation}

by (1.). These metrics agree on the set

\begin{equation}
F_k \cap (F \cup \{y_k,y^*_{k+1}\}) = F \cup \{y_k\}.
\end{equation}

Therefore, we can ``merge'' them to a metric on all of $F^*_{k+1}$ by letting

\begin{equation}
e^*(y^*_{k+1}, y_j) = \inf \{e^*(y^*_{k+1}, z) + e^*(z,y_j) \colon z \in \{y_1, \dots, y_{k-1}\} \}.
\end{equation}

Now choose $\eps < 2^{ -k -1}\delta$ and apply approximate universality to $F^*_{k+1}$. This yields a point $y_{k+1} \in Y$ such that

\begin{equation}
|d(y_{k+1}, z)  - e^*(y^*_{k+1}, z) | < 2^{-k-1}\delta
\end{equation}

for all $z \in F_k$. By definition of $e^*$, we have

\begin{equation}
|d(y_{k+1}, x_i)  - d^*(y^*_{k+1}, z) | < 2^{-k-1}\delta
\end{equation}

for $1 \leq i \leq n$, and $(iii)$ yields

\begin{equation}
d(y_{k+1}, y_k) < e^*(y^*_{k+1},y_k) + \eps \leq 2^{-k-1}\delta + 2^{-k-1}\delta = 2^{-k}\delta
\end{equation}

as required.

\end{proof}

\include{descriptive_set_theory-trees}

\section{The Borel hierarchy}

\include{descriptive_set_theory-borel}

\subsection{Subspaces of Polish Spaces}

Closed subsets of Polish spaces (with the subspace topology) are Polish (Proposition~\ref{properties-polish}).

What about open subsets like $(0,1) \subset \Real$? It is clear from this example that we have to find a different compatible metric.

\begin{proposition}\label{prop-open-subset-polish}Any open subset of a Polish space $X$ is Polish.

\end{proposition}\begin{proof}Let $U \subset X$ be open, where we assume that $U \neq X$.  Consider the set

\begin{equation}
F = \{(t,x) \in \Real \times X \colon \; t \cdot d(x,X\setminus U) =1 \}.
\end{equation}

The mapping $x \mapsto d(x,X\setminus U)$ from $X$ to $\Real$ is continuous. Therefore, $F$ is closed and thus Polish.

If we restrict the projection $\pi_2: \Real \times X \to X$ to $F$, we obtain a homeomorphism between $F$ and $U$. As homeomorphic images of Polish spaces are Polish (Proposition~\ref{properties-polish}), $U$ is Polish.

\end{proof}\begin{framed}
\textbf{Exercise}\\
Give an alternative proof of the preceding result by considering the following:

Let $d$ be a compatible metric on $X$ and define

\begin{equation}
\overline{d}(x,y) = \frac{d(x,y)}{1+d(x,y)}.
\end{equation}

Show that this is a metric that induces the same topology.

Now let

\begin{equation}
d_U(x,y) = \overline{d}(x,y) + \left | \frac{1}{\overline{d}(x,X\setminus U)} - \frac{1}{\overline{d}(y,X\setminus U)} \right|.
\end{equation}

Verfiy that this is a metric on $U$ compatible with the subspace topology
with respect to which $U$ is complete.
\end{framed}

\begin{proposition}\label{prop-intersection-polish}Let $X$ be a Polish space, and suppose $(Y_n)$ is a sequence of Polish subspaces of $X$. Then $\bigcap_n Y_n$ is a Polish subspace of $X$.

\end{proposition}\begin{proof}Consider the mapping $f: X \to X^\Nat$ given by $x \mapsto (x, x, x, \dots)$. The restriction of $f$ to $\bigcap_n Y_n$ is a homeomorphism between $\bigcap_n Y_n$ and the diagonal $\Delta \subseteq \prod_n Y_n$,

\begin{equation}
\Delta = \{(x,x,x, \dots) \colon \: x \in Y_n \text{ for all } n \in \Nat \}.
\end{equation}

$\Delta$ is closed in the product space $\prod_n Y_n$ and hence Polish, and this property pushes over to $\bigcap_n Y_n$ (see Proposition~\ref{properties-polish}).

\end{proof}Hence every $G_\delta$ subset of a Polish space is Polish. This is as far as we can get.

\begin{theorem}[Mazurkiewicz]\label{thm-subsets-polish}A subset of a Polish space is Polish if and only if it is $G_\delta$.

\end{theorem}We have already established the ``if'' direction of this result.
For the other direction, we need a lemma that is interesting in its own right.

\begin{lemma}[Kuratowski extension lemma]\label{lemma-kuratowski-extension}Suppose $X, Y$ are Polish spaces, $A \subseteq X$, and $f: A \to Y$ continuous. Then there exists a $G_\delta$ set $G$ with $A \subseteq G \subseteq \Cl{A}$ and a continuous extension $g : G \to Y$ of $f$.

\end{lemma}Compare this with the last lecture, where we showed that the points of continuity of a function is always a $G_\delta$ set (Theorem~\ref{thm-Young}).

\begin{proof}We can adapt the \textit{$\eps$-oscillation set} $C_\eps$ used in the proof of Theorem~\ref{thm-Young} to the domain $A$:

\begin{equation*}
\tag{$*$}
    C^A_\eps = \left \{ x \in X \colon \exists \delta > 0 \: \forall a,b \in A \; [ a,b \in U_\delta(x) \: \Rightarrow \: d(f(a),f(b)) < \eps  ] \right \}.
\end{equation*}

As before, $C^A_\eps$ is open and hence

\begin{equation}
G = \Cl{A} \cap \bigcap_n C^A_{1/n}
\end{equation}

is $G_\delta$ and since $f$ is continuous, $A \subseteq G \subseteq \Cl{A}$.

To extend $f$ to $G$, let $x \in G$. Since $x \in \Cl{A}$, there exists $(a_n)$ in $A$ with $x = \lim_n a_n$. As $x \in \bigcap_n C^A_{1/n}$, $(f(a_n))$ is Cauchy. $Y$ is complete, so there exists $y = \lim_n f(a_n) \in Y$. It is straightforward to verify that $y$ is independent of the choice of $(a_n)$ and agrees with $f(x)$ for $x \in A$. Hence we can put

\begin{equation}
g(x) = y,
\end{equation}

which yields the desired continuous extension.

\end{proof}Now assume $Y \subset X$ is Polish but not $G_\delta$. Then, by the previous lemma, the identity mapping $\operatorname{id}: Y \to Y$ has a proper continuous extension $g: G \to Y$ to a $G_\delta$ set $G$ with $Y \subsetneq G \subseteq \Cl{Y}$. Let $x \in G\setminus Y$. $Y$ is dense in $G$, so there exists $(y_n)$ in $Y$ with $x = \lim_n y_n$. By continuity

\begin{equation}
x = \lim_n y_n = \lim_n g(y_n) = g(x) \in Y,
\end{equation}

contradiction. This completes the proof of Theorem~\ref{thm-subsets-Polish}.

\subsubsection{Borel set as clopen sets}

More complicated Borel sets in Polish spaces are not Polish anymore in the subspace topology, as we just saw. But what if we are allowed to change the topology? In the process, we would like to ``preserve'' as much as possible of the original space. It turns out we can change the topology so that a given Borel set becomes clopen while inducing the same family of Borel sets overall.

We start with closed sets.

\begin{lemma}\label{lem-closed-clopen}If $X$ is a Polish space with topology $\mathcal{O}$, and $F \subseteq X$ is closed, then there exists a finer topology $\mathcal{O}' \supseteq \mathcal{O}$ such that $\mathcal{O}$ and $\mathcal{O}'$ give rise to the same class of Borel sets in $X$, and $F$ is clopen with respect to $\mathcal{O}'$.

\end{lemma}\begin{proof}By Proposition~\ref{properties-polish} and Proposition~\ref{prop-open-subset-Polish}, respectively, $F$ and $X \setminus F$ are Polish spaces with compatible metrics $d_F$ and $d_{X\setminus F}$,  respectively. Wlog $d_F, d_{X\setminus F} < 1$. We form the \textit{disjoint union of the spaces $F$ and $X \setminus F$}: This is the set $X = F \,\sqcup\, X \setminus F$ with the following topology, $\mathcal{O}'$. $U \subseteq F \,\sqcup\, X \setminus F$ is in $\mathcal{O}'$ if and only if $U \cap F$ is open (in $F$) and  $U \cap X\setminus F$ is open (in $X\setminus F$).

The disjoint union is Polish, as witnessed by the following metric.

\begin{equation}
d_\sqcup(x,y) =  
    \begin{cases}
            d_F(x,y) &\text{if } x,y \in F, \\
            d_{X\setminus F}(x,y) &\text{if } x,y \in X\setminus F, \\
            \;  2  &\text{otherwise}.
    \end{cases}
\end{equation}

It is straightforward to check that $d$ is compatible with $\mathcal{O}'$. Furthermore, let $(x_n)$ be Cauchy in $(X,d_\sqcup)$. Then the $x_n$ are completely in $F$ or in $X\setminus F$ from some point on, and hence $(x_n)$ converges.

Under the disjoint union topology, $F$ is is clopen. Moreover, an open set in this topology is a disjoint union of an open set in $X\setminus F$, which also open the original topology $\mathcal{O}$, and an intersection of an open set from $\mathcal{O}$ with $F$. Such sets are Borel in $(X,\mathcal{O})$, hence $(X,\mathcal{O})$ and $(X,\mathcal{O}')$ have the same Borel sets.

\end{proof}\begin{theorem}\label{thm-borel-clopen}Let $X$ be a Polish space with topology $\mathcal{O}$, and suppose $B \subseteq X$ is Borel. Then there exists
a finer Polish topology $\mathcal{O}' \supseteq \mathcal{O}$ such that $\mathcal{O}$ and $\mathcal{O}'$ give rise to the same class of Borel sets in $X$, and $B$ is clopen with respect to $\mathcal{O}'$.

\end{theorem}\begin{proof}Let $\mathcal{S}$ be the family of all subsets $A$ of $X$ for which a finer topology exists that has the same Borel sets as $\mathcal{O}$ and in which $A$ is clopen.

We will show that $\mathcal{S}$ is a $\sigma$-algebra, which by the previous Lemma contains the closed sets. Hence $\mathcal{S}$ must contain all Borel sets, and we are done.

$\mathcal{S}$ is clearly closed under complements, since the complement of a clopen set is clopen in any topology.

So assume now that $\{A_n\}$ is a countable family of sets in $\mathcal{S}$. Let $\mathcal{O}_n$ be a topology on $X$ that makes $A_n$ clopen and does not introduce new Borel sets.

Let $\mathcal{O}_\infty$ be the topology generated by $\bigcup_n \mathcal{O}_n$. Then $\bigcup_n A_n$ is open in $(X, \mathcal{O}_\infty)$, and we can apply Lemma~\ref{lem-closed-clopen}. For this to work, however, we have to show that $(X, \mathcal{O}_\infty)$ is Polish and does not introduce any new Borel sets.

We know that the product space $\prod (X,\mathcal{O}_n)$ is Polish. Consider the mapping $\phi: X \to \prod_n X$

\begin{equation}
x \mapsto (x,x,x, \dots).
\end{equation}

Observe that $\phi$ is a continuous mapping between $(X,\mathcal{O}_{\infty})$ and $\prod_n X$.  The preimage of a basic open set $U_1 \times U_2 \times \cdots \times U_n \times X \times X \times \cdots$ under $\phi$ is just the intersection of the $U_i$. Furthermore, $\phi$ is clearly one-to-one, and the inverse mapping between $\phi(X)$ and $X$ is continuous, too.

If we can show that $\phi(X)$ is closed in $\prod_n X$, we know it is Polish as a closed subset of a Polish space, and since $(X,\mathcal{O}_\infty)$ is homeomorphic to $\phi(X)$, we can conclude it is Polish.

To see that $\phi(X)$ is closed in $\prod_n X$, let $(y_1,y_2,y_3, \dots) \in \Co{\phi(X)}$. Then there exist $i < j$ such that $y_i \neq y_j$. Since $(X, \mathcal{O})$ is Polish, we can pick $U,V$ open, disjoint such that $y_i \in U$, $y_j \in V$. Since each $\mathcal{O}_n$ refines $\mathcal{O}$, $U$ is open in $\mathcal{O}_i$, and $V$ is open in $\mathcal{O}_j$. Therefore,

\begin{equation}
X_1 \times X_2 \times \cdots \times  X_{i-1} \times U \times X_{i+1} \times \cdots \times X_{j_1} \times V \times X_{j+1} \times X_{j+2} \times \cdots
\end{equation}

where $X_k = X$ for $k \neq i,j$, is an open neighborhood of $(y_1,y_2,y_3, \dots)$ completely contained in $\Co{\phi(X)}$.

Finally, too see that the Borel sets of $(X, \mathcal{O}_\infty)$ are the same as the ones of $(X,\mathcal{O})$, for each $n$, let $\{U^{(n)}_i\}_{i \in \Nat}$ be a basis for $\mathcal{O}_n$. By assumption, all sets in $\mathcal{O}_n$ are Borel sets of $(X, \mathcal{O})$. The set $\{U^{(n)}_i\}_{i,n \in \Nat}$ is a subbasis for $\mathcal{O}_\infty$. This means that any open set in $(X, \mathcal{O}_\infty)$ is a countable union of finite intersections of the $U^{(n)}_i$. Since every $U^{(n)}_i$ is Borel in $(X, \mathcal{O})$, this means that any open set in $\mathcal{O}_\infty$ is Borel in $(X, \mathcal{O})$. Since the Borel sets are closed under complementation and countable unions, this in turn implies that every Borel set of $(X, \mathcal{O}_\infty)$ is already Borel in $(X, \mathcal{O})$.

\end{proof}\begin{corollary}[Perfect subset property for Borel sets; Alexandroff, Hausdorff]\label{cor-perfect-borel}In a Polish space, every uncountable Borel set has a perfect subset.

\end{corollary}\begin{proof}Let $(X,\mathcal{O})$ be Polish, and assume $B \subseteq X$ is Borel. We can choose a finer topology $\mathcal{O}' \supseteq \mathcal{O}$ so that $B$ becomes clopen, but the Borel sets stay the same. By Theorem~\ref{thm-subsets-Polish}, $B$ is Polish with respect to the subspace topology $\mathcal{O}'|_B$

Suppose $B$ is uncountable. By Theorem~\ref{thm-Cantor-embedding} there exists a continuous injection $f$ from $\Cant$ (with respect to the standard topology) into $(B,\mathcal{O}'|_B)$.

Since $\mathcal{O}'$ is finer than $\mathcal{O}$, $f$ is continuous with respect to $\mathcal{O}$, too. Since $\Cant$ is compact, $f(\Cant)$ is closed with respect to $\mathcal{O}$. Finally, $f(\Cant)$ has no isolated points with respect to $\mathcal{O}'$, which then also holds for the coarser topology $\mathcal{O}$.

Therefore, $B$ has a perfect subset.

\end{proof}

\subsection{Measure and Baire Category}

At the end of the previous section, we saw that Borel sets are well-behaved in the sense that they possess the \textbf{perfect subset property}. Two other important regularity properties are \textbf{measurability} and the \textbf{Baire property}, which we will introduce in this section.

\subsubsection{Filters and Ideals}

The most common measure of size is, of course, cardinality. In the presence of uncountable sets (like in a perfect Polish space), the usual division is between countable and uncountable sets. The smallness of the countable sets is reflected, in particular, by two properties: A subset of a countable set is countable, and countable unions of countable set are countable. These characteristics are shared with other notions of smallness, two of which we will encounter in this lecture.

\begin{definition}\label{def-ideal}A non-empty family $\mathcal{I} \subseteq \Pow(X)$ of subsets of a given set $X$ is an \textbf{ideal} if
\begin{align*}
(\Op{I1}) & \qquad A \in \mathcal{I} \text{ and } B \subseteq A \text{ implies } B \in \mathcal{I},\\
(\Op{I2}) & \qquad A, B \in \mathcal{I}  \text{ implies } A \cup B \in\mathcal{I}.
\end{align*}
\end{definition}If we have closure even under \textit{countable unions}, we speak of a \textbf{$\sigma$-ideal}. For example, while the countable sets in $\Real$ form a $\sigma$-ideal, the finite subsets only form an ideal.

Another example of ideals are the so-called \textbf{principal ideals}. These are ideals of the form

\begin{equation}
\langle Z \rangle = \{ A \colon A \subseteq Z\}
\end{equation}

for a fixed $Z \subseteq X$.

The \textbf{dual notion} to an ideal is that of a \textbf{filter}. It reflects that the sets in a filter share some \textbf{largeness property}.

\begin{definition}\label{def-filter}A non-empty family $\mathcal{F} \subseteq \Pow(X)$ of subsets of a given set $X$ is a \textbf{filter} if
\begin{align*}
(\Op{F1}) & \qquad A \in \mathcal{F} \text{ and } B \supseteq A \text{ implies } B \in \mathcal{F},\\
(\Op{F2}) & \qquad A, B \in \mathcal{F}  \text{ implies } A \cap B \in\mathcal{F}.
\end{align*}
\end{definition}Again, closure under countable intersections yields \textbf{$\sigma$-filters}.

If $\mathcal{I}$ is a ($\sigma$-) ideal, then $\mathcal{F} = \{\Co{A} \colon A \in \mathcal{I}\}$ is a ($\sigma$-) filter. Hence the co-finite subsets of $\Real$ form a filter, and the co-countable subsets form a $\sigma$-filter.

Note that the complement of a ($\sigma$-) ideal (in $\Pow(X)$) is not necessarily a ($\sigma$-) filter. This is true, however, for a special class of ideals/filters.

\begin{definition}\label{def-ultrafilter}A non-empty family $\mathcal{I} \subseteq \Pow(X)$ is a \textbf{prime ideal} if it is an ideal for which

\begin{quote}
for every $A \in X$, either $A\in \mathcal{I}$ or $\Co{A} \in \mathcal{I}$ (but not both).
\end{quote}

An \textbf{ultrafilter} is a filter whose complement in $\Pow(X)$ is a prime ideal.

\end{definition}In light of the small-/largeness motivation, prime ideals and ultrafilters provide a \textit{complete} separation of $X$: Each set is either small or large.

\subsubsection{Measures}

Coarsely speaking, a measure assigns a size to a set in a way that reflects our basic geometric intuition about sizes: The size of the union of disjoint objects is the sum of their sizes. The question whether this can be done in a consistent way for \textit{all} subsets of a given space is of fundamental importance and has motivated many questions in set theory.

The formally, a measure $\mu$ on $X$ is a $[0,\infty]$-valued function defined on subsets of $X$ that satisfies

\begin{align*}
    (\Op{M1}) & \qquad \mu(\emptyset) = 0 \\
    (\Op{M2}) & \qquad \mu(\bigcup_n A_n) = \sum_n \mu(A_n), \\
              & \qquad \qquad \text{whenever the $A_n$ are pairwise disjoint.}
\end{align*}

The question is, of course, which subsets of $X$ can be assigned a measure. The condition (M2) suggests that this family is closed under countable unions. Furthermore, if $A \subseteq X$, then the equation $\mu(X) = \mu(A) + \mu(\Co{A})$ suggests that $\Co{A}$ should be measurable, too. In other words, the sets who are assigned a measure form a $\sigma$-algebra.

\begin{definition}\label{def-measure}A \textbf{measurable space} is a pair $(X, \mathcal{S})$, where $X$ is a set and $\mathcal{S}$ is a $\sigma$-algebra on $X$. A \textbf{measure} on a measurable space $(X, \mathcal{S})$ is a function $\mu: \mathcal{S} \to [0,\infty]$ that satisfies (M1) and (M2) for any pairwise disjoint family $\{A_n\}$ in $\mathcal{S}$. If $\mu$ is a measure on $(X, \mathcal{S})$, then the triple $(X,\mathcal{S}, \mu)$ is called a \textbf{measure space}.

\end{definition}If we want the measure $\mu$ to reflect also some other basic intuition about geometric sizes, this often puts restrictions on the $\sigma$-algebra of measurable sets. For example, in $\Real$ the measure of an interval should be its \textit{length}. We will see later that, if we assume the Axiom of Choice, it is impossible to assign every subset of $\Real$ a measure, so that (M1) and (M2) are satisfied, and the measure of an interval is its length.

To have some control over what the $\sigma$-algebra of measurable sets should be, one can construct a measure more carefully, start with a measure on basic objects such as intervals or balls, and then extend it to larger classes of sets by approximation.

An essential component in this extension process is the concept of an \textbf{outer measure}.

\begin{definition}\label{def-outer-measure}An \textbf{outer measure} on a set $X$ is a function $\mu^*: \Pow(X) \to [0,\infty]$ such that
\begin{align*}
    (\Op{O1}) & \qquad  \mu^*(\emptyset) = 0,\\
    (\Op{O2}) & \qquad  A \subseteq B \text{ implies } \mu^*(A) \leq \mu^*(B), \\
    (\Op{O3}) & \qquad  \mu^*(\bigcup_n A_n) \leq \sum_n \mu^*(A_n), \\
         & \qquad  \qquad \text{for any countable family  $\{A_n\}$ in $X$}.
\end{align*}
\end{definition}An outer measure hence weakens the conditions of \textbf{additivity} (M2) to \textbf{subadditivity} (O3). This makes it possible to have non-trivial outer measures that are defined on \textit{all} subsets of $X$.

The usefulness of outer measures lies in the fact that they can always be restricted to subset of $\Pow(X)$ on which they behave as measures.

\begin{definition}\label{def-measurable}Let $\mu^*$ be an outer measure on $X$. A set $A \subseteq X$ is \textbf{$\mu^*$-measurable} if

\begin{equation}
\mu^*(B) = \mu^*(B \cap A) + \mu^*(B \setminus A) \quad \text{ for all $B \subseteq X$}.
\end{equation}

\end{definition}This definition is justified rather by its consequences than by its intuitive appeal. Regarding the latter, suffice it to say here that outer measures may be rather far from being even \textit{finitely} additive. The definition singles out those sets that split all other sets correctly, with regard to measure.

\begin{proposition}\label{prop-measurable-sets}The class of $\mu^*$-measurable sets forms a $\sigma$-algebra $\mathcal{M}$, and the restriction of $\mu^*$ to $\mathcal{M}$ is a measure.

\end{proposition}A proof can be found in any standard book on measure theory, for instance \cite{Halmos:1950a} or \cite{RoydenFitzpatrick_1988n}.

The size of the $\sigma$-algebra of measurable sets depends, of course, on the outer measure $\mu^*$. If $\mu^*$ is behaving rather pathetically, we cannot expect $\mathcal{M}$ to contain many sets.

\subsubsection{Lebesgue measure}

A standard way to obtain ``nice'' outer measures is to start with a well-behaved function defined on a certain class of sets, and then approximate. The paradigm for this approach is the construction of \textbf{Lebesgue measure} on $\Real$.

\begin{definition}\label{def-lebesgue}The \textbf{Lebesgue outer measure} $\lambda^*$ of a set $A \subseteq \Real$ is defined as

\begin{equation}
\lambda^*(A) = \inf \left \{ \sum_n |b_n - a_n| \colon A \subseteq \bigcup_n (a_n,b_n) \right \}.
\end{equation}

\end{definition}\begin{framed}
\textbf{Exercise}\\
Show that $\lambda^*$ indeed defines an outer measure.
\end{framed}

We call the $\lambda^*$-measurable sets \textbf{Lebesgue measurable}.

The following two facts are also standard \cite{RoydenFitzpatrick_1988n}.

\begin{proposition}\label{prop-outer-meas-interval}If $I \subseteq \Real$ is an interval, then $\lambda^*(I)$ is equal to the length of $I$ (possibly infinite).

\end{proposition}\begin{proposition}\label{prop-interval-measurable}Any interval $I \subseteq \Real$ is Lebesgue measurable.

\end{proposition}\begin{corollary}\label{cor-borel-measurable}Any Borel set in $\Real$ is Lebesgue measurable

\end{corollary}\begin{proof}This follows from Proposition~\ref{prop-measurable-sets}, Proposition~\ref{prop-interval-measurable} and the fact that any open set in $\Real$ is a countable union of intervals.

\end{proof}The construction of Lebesgue measure can be generalized and extended to other metric spaces, for example through the concept of \textbf{Hausdorff measures}.

All these measures are \textbf{Borel measures}, in the sense that the Borel sets are measurable. However, there are measurable sets that are not Borel sets. The reason for this lies in the presence of \textbf{nullsets}, which are measure theoretically ``easy'' (since they do not contribute any measure at all), but can be topologically quite complicated.

\subsubsection{Nullsets}

Let $\mu^*$ be an outer measure on $X$. If $\mu^*(A) = 0$, then $A$ is called a \textbf{$\mu^*$-nullset}.

\begin{proposition}\label{prop-nullsets-measurable}Any $\mu^*$-nullset is $\mu^*$-measurable.

\end{proposition}\begin{proof}Suppose $\mu^*(A)=0$. Let $B \subseteq X$. Then, since $\mu^*$ is subadditive and monotone,

\begin{equation}
\mu^*(B) \leq \mu^*(B \cap A) + \mu^*(B \cap \Co{A}) = \mu^*(B \cap \Co{A}) \leq \mu^*(B),
\end{equation}

and therefore $\mu^*(B) = \mu^*(B \cap A) + \mu^*(B \cap \Co{A})$.

\end{proof}The next result confirms the intuition that nullsets are a notion of smallness.

\begin{proposition}\label{prop-null-sigmaideal}The $\mu^*$-nullsets form a $\sigma$-ideal.

\end{proposition}\begin{proof}(I1) follows directly from monotonicity (O2). Countable additivity follows immediately from subadditivity (O3).

\end{proof}In case of Lebesgue measure, we can use Proposition Proposition~\ref{prop-nullsets-measurable} to further describe the Lebesgue measurable subsets of $\Real$.

\begin{proposition}\label{measurable-diff-borel}A set $A \subseteq \Real$ is Lebesgue measurable if and only if it is the difference of a $\bPi^0_2$ set and a Lebesgue nullset.

\end{proposition}\begin{proof}We first assume $\lambda^*(A) < \infty$.  Let $G_n \subseteq \Real$ be an open set such that $G_n \supseteq A$ and  $\lambda^*(G_n) \leq \lambda^*(A) + 1/n$. The existence of such a $G_n$ follows from the definition of $\lambda^*$, and the fact that every open set is the disjoint union of open intervals.  Then $G = \bigcap_n G_n$ is $\bPi^0_2$, $A \subseteq G$, and for all $n$,

\begin{equation}
\lambda^*(A) \leq \lambda^*(G) \leq \lambda^*(A) + 1/n
\end{equation}

hence $\lambda^*(A) = \lambda^*(G)$. Hence for $N = G \setminus A$, since $A$ is measurable,

\begin{equation}
\lambda^*(N) = \lambda^*(G) - \lambda^*(A)  = 0 \quad \text{ and } \quad A = G \setminus N.
\end{equation}

If $\lambda^*(A) = \infty$, we set $A_m = A \cap [m,m+1)$ for $m \in \Integer$. By monotonicity, each $\lambda^*(A_m)$ is finite. For each $m \in \Integer$, $n \in \Nat$, pick $G^{(m)}_n$ open such that $\lambda^*(G^{(m)}_n) \leq \lambda^*(A) + 1/2^{n+2|m|+1}$. Then, with

\begin{equation}
\bigcap_{n \in \Nat} \bigcup_{m \in \Integer} G^{(m)}_n,
\end{equation}

$N = G\setminus A$ is the desired set.

For the other direction, note that the measurable sets form a $\sigma$-algebra which contains both the Borel sets and the nullsets. Hence any set that is the difference of a Borel set and a nullset is measurable, too.

\end{proof}\begin{framed}
\textbf{Exercise}\\
Show that each Lebesgue measurable set can be written as a disjoint union of a $\bSigma^0_2$ set and a nullset.
\end{framed}

Hence if a set is measurable, it differs from a (rather simple) Borel set only by a nullset.

We also obtain the following characterization of the $\sigma$-algebra of Lebesgue measurable sets.

\begin{proposition}\label{prop-measurable-sigma-algebra}The $\sigma$-algebra of Lebesgue measurable sets in $\Real$ is the smallest $\sigma$-algebra containing the open sets and the nullsets.

\end{proposition}As mentioned before, there are Lebesgue measurable sets that are not Borel sets. We will eventually encounter such sets. The question which sets exactly are Lebesgue measurable was one of the major questions that drove the development of descriptive set theory, just like the question which uncountable sets have perfect subsets.

\subsubsection{Baire category}

The basic paradigm for smallness here is of topological nature. A set is small if it does not look anything like an open set, not even under closure. In the following, let $X$ be a Polish space.

\begin{definition}\label{def-nowhere-dense}A set $A \subseteq X$ is \textbf{nowhere dense} if its complement contains an open, dense set.

\end{definition}Being nowhere dense means for any open set $U \subseteq X$ we can find a non-empty open subset $V \subseteq U$ such that $V \subseteq \Co{A}$. In other words, a nowhere dense set is ``full of holes''.

Examples of nowhere dense sets are all finite, or more generally, all discrete subsets of a perfect Polish space, i.e. sets all whose points are isolated. There are non-discrete nowhere dense sets, such as $\{0\} \cup \{1/n \colon n \in \Nat \}$ in $\Real$, even uncountable ones, such as the middle-third Cantor set.

The nowhere dense sets form an ideal, but not a $\sigma$-ideal: Every singleton set is nowhere dense, but there are countable sets that are not, such as the rationals $\Rat$ in $\Real$.

To obtain a $\sigma$-ideal, we close the nowhere dense sets under countable unions.

\begin{definition}\label{def-meager}A set $A \subseteq X$ is \textbf{meager} or of \textbf{first category} if it is the countable union of nowhere dense sets. Non-meager sets are also called sets of \textbf{second category}. Complements of meager sets are called \textbf{comeager} or \textbf{residual}.

\end{definition}The meager subsets of $X$ form a $\sigma$-ideal. Examples of meager sets are all countable sets, but there are uncountable ones (Cantor set).

The concept of Baire category is often used in existence proofs: To show that a set with a certain property exists, one shows that the set of points \textit{not having the property} is meager. A famous example is Banach's proof of the existence of continuous, nowhere differentiable functions. For this to work, of course, we have to ensure that the complements of meager sets are non-empty.

\begin{theorem}[Baire Category Theorem]\label{thm-baire-category}For any Polish space $X$, the following statements hold.

\begin{itemize}
\item \textbf{(a)} For every meager set $M \subseteq X$, the complement $\Co{M}$ is dense in $X$.
\item \textbf{(b)} No non-empty open set is meager.
\item \textbf{(c)} If $\{D_n\}$ is a countable family of open, dense sets, then  $\bigcap_{n} D_n$ is dense.
\end{itemize}

\end{theorem}\begin{proof}(a) Assume  $M = \bigcup_n N_n$, where each $N_n$ is nowhere dense. Then $\Co{M} = \bigcap D_n$, where each $D_n$ contains a dense, open set. Let $U \subseteq X$ be open.

We construct a point $x \in U \cap \Co{M}$ by induction. We can find an open ball $B_1$ of radius $<1$ such that $\Cl{B_1} \subseteq U \cap D_1$, since $D_1$ contains a dense open set. In the next step, we use the same property of $D_2$ to find an open ball $B_2$ of radius $<1/2$ whose closure is completely contained in $B_1 \cap D_2$.
Continuing inductively, we obtain a  nested sequence of balls $B_n$ of radius $<1/n$ such that $\Cl{B_n} \subseteq B_{n -1} \cap D_n$.

Let $x_n$ be the center of $B_n$. Then $(x_n)$ is a Cauchy sequence, so $x = \lim_n x_n$ exists in $X$. Since for any $n$, all but finitely many $x_i$ are in $B_n$, we have $x \in \Cl{B_n}$ for all $n$. Therefore, by construction

\begin{equation}
x \in \bigcap_n \Cl{B_n} = \bigcap_n B_n \subseteq U \cap \bigcap_n D_n \subseteq U.
\end{equation}

(b) follows immediately from (a), the proof of (c) is exactly the same as that for (a). In fact, the three statements are equivalent.

\end{proof}Any topological space that satisfies the three equivalent conditions (a)-(c) is called a \textbf{Baire space} (not to be confused with \textit{the} Baire space $\Baire$ -- the latter is, of course, a Baire space, too).

As an application, we determine the exact location of $\Rat$ in the Borel hierarchy of $\Real$.

\begin{corollary}\label{cor-rationals-not-pi2}$\Rat$ is not a $\bPi^0_2$ set, hence a true $\bSigma^0_2$ set.

\end{corollary}\begin{proof}Note that $\Real$ cannot be meager, by (b). Since $\Rat$ is meager, $\Real \setminus \Rat$ cannot be meager either.
If $\Rat$ were a $\bPi^0_2$ set, it would be the intersection of open, dense sets and hence its complement $\Real \setminus \Rat$ would be meager.

\end{proof}\subsubsection{The Baire property}

We have seen that the measurable sets are precisely the ones that differ from a $\bPi^0_2$ set by a nullset.
We can introduce a similar concept for Baire category.

\begin{definition}\label{def-bp}A set $B \subseteq X$ has the \textbf{Baire property} if there exists an open set $G$ and a meager set $M$ such that

\begin{equation}
B \bigtriangleup G = M,
\end{equation}

where $\bigtriangleup$ denotes the \textit{symmetric difference} between two sets: $A \bigtriangleup B = (A\setminus B) \cup (B \setminus A)$.

\end{definition}\begin{framed}
\textbf{Exercise}\\
Show that $\bigtriangleup$ is commutative, associative, and satisfies the distributive law

\begin{equation}
A \cap (B \bigtriangleup C) = (A \cap B) \bigtriangleup (A \cap C).
\end{equation}
\end{framed}

In the above definition, we can replace open sets by closed sets.

\begin{lemma}\label{lem-bp-closed}A set $B$ has the Baire property if and only if it can be represented in the form $B = F \bigtriangleup M$,  where $F$ is closed and $M$ is meager.

\end{lemma}\begin{proof}Suppose $B = G \bigtriangleup M$, $G$ open and $M$ meager.

Then $N = \Cl{G} \setminus G$ is nowhere dense and closed. Furthermore, $Q = M \bigtriangleup N$ is meager (it is the union of two meager sets). We easily verify that $G = \Cl{G} \bigtriangleup N$, and therefore

\begin{equation}
B = G \bigtriangleup M = (\Cl{G} \bigtriangleup N) \bigtriangleup M = \Cl{G} \bigtriangleup (N \bigtriangleup M) = \Cl{G} \bigtriangleup Q,
\end{equation}

as desired.

The converse direction is similar, using the interior instead of the closure.

\end{proof}\begin{proposition}\label{prop-bp-sigma-algebra}The sets having the Baire property form a $\sigma$-algebra.

\end{proposition}\begin{proof}To show closure under complement, note that $\Co{(A \bigtriangleup B)} = \Co{A} \bigtriangleup B$. Therefore,
if $B = G \bigtriangleup M$ with $G$ open and $M$ meager, we have $\Co{B} = \Co{G} \bigtriangleup M$, and we can use Lemma~\ref{lem-BP-closed}.

Now assume $B = \bigcup B_i$, and for each $i$ there exist open $G_i$ and meager $M_i$ such that $B_i = G_i \bigtriangleup M_i$.

Let $G = \bigcup G_i$ and $M = \bigcup M_i$. Then $G$ is open and $M$ is meager (since the meager sets for a $\sigma$-ideal).

We easily check that

\begin{equation}
G \setminus M  \subseteq B \subseteq G \cup M.
\end{equation}

This implies $B \bigtriangleup G \subseteq M$ and hence $B \bigtriangleup G$ is meager.

Since

\begin{equation}
B = G \bigtriangleup (B \bigtriangleup G),
\end{equation}

we conclude that $B$ has the Baire property.

\end{proof}\begin{corollary}\label{cor-bp-algebra-small}The $\sigma$-algebra of sets having the Baire property is the smallest $\sigma$-algebra containing all open and all meager sets.

\end{corollary}\begin{framed}
\textbf{Exercise}\\
Show that $B$ has the Baire property if and only if it can be represented as a $G_\delta$ set plus a meager set.
\end{framed}

As in the case of measure, there exist non-Borel sets with the Baire property, and using the Axiom of Choice one can show that there exists set that do not have the Baire property.

We conclude this lecture with a note on the relationship between measure and category. From the results so far it seems that they behave quite similarly. This might lead to the conjecture that maybe they more or less coincide. This is not so, in fact, they are quite orthogonal to each other, as the next result shows.

\begin{proposition}The real numbers can be partitioned into two subsets, one a Lebesgue nullset and the other one meager.

\end{proposition}\begin{proof}Let $(G_n)$  be a sequence of open sets witnessing that $\Rat$ is a nullset, i.e. each $G_n$ is a union of disjoint open intervals that covers $\Rat$ and whose total length does not exceed $2^{ -n}$. Then $G = \bigcap_n G_n$ is a nullset, but at the same time it is an intersection of open dense sets, thus comeager, hence its complement is meager.

\end{proof}

\include{descriptive_set_theory-choice}

\include{descriptive_set_theory-codingborel}

\subsection{The Structure of Borel Sets}

In this chapter, we further investigate the structure of Borel sets. We will use the results of the previous lecture to derive various closure properties and other structural results. As an application, we see that the Borel hierarchy is indeed proper.

\subsubsection{Notation}

Before we go on, we have to address some notational issues. So far we have used notation quite liberally, especially when it came to product sets. We will continue to do so, but we want to put this on a firmer footing.

Using coding, we can identify any product space $\Nat^m \times (\Baire)^n$ with $\Nat^\Nat$. One way to do this is to fix, for each $n \geq 1$, an effective homeomorphism $\theta_n: (\Baire)^n \to \Baire$ and map

\begin{equation}
(k_1, \dots, k_m, \alpha_1, \dots, \alpha_n) \mapsto (k_1,\dots, k_m, \theta_n(\alpha_1, \dots, \alpha_n)).
\end{equation}

Here $(k_1,\dots, k_m, \theta_n(\alpha_1, \dots, \alpha_n))$ is just a suggestive way of writing the concatenation

\begin{equation}
\Tup{k_1} \Conc \cdots \Conc \Tup{k_m} \Conc \theta_n(\alpha_1, \dots, \alpha_n).
\end{equation}

We have already used this notation in the previous lecture. In the following, we will continue to switch freely between product sets and their coded counterparts, as subsets of $\Baire$.

Another notation identifies sets and relations. We will identify sets $A \subseteq \Nat^m \times (\Baire)^n$ with the relation they induce and write $A(k_1, \dots, k_m, \alpha_1, \dots, \alpha_n)$ instead of $(k_1, \dots, k_m, \alpha_1, \dots, \alpha_n) \in A$. Conversely, we will identify relations with the set they induce.

\subsubsection{Normal forms}

Theorem Theorem~\ref{thm-Borel-arith} tells us that a set $A \subseteq \Baire$ is $\bSigma^0_n$ if and only if it is definable by a $\Sigma^0_n$ formulas over $\mathcal{A}^2$, relative to some parameter. That means that there exists a \textbf{bounded formula} $\phi(x_1, \dots, x_n,\alpha,\underline{\gamma})$ (i.e. all quantifiers are bounded) such that

\begin{equation}
A(\alpha) \iff \exists x_1 \: \dots \: \Qu x_n \; \phi(x_1, \dots, x_n, \alpha,\gamma) \text{ holds (in the standard model).}
\end{equation}

Here $\gamma$ is the parameter, and $\Qu$ is ``$\exists$'' if $n$ is odd, and ``$\forall$'' if $n$ is even.

Similarly, $A \subseteq \Baire$ is $\bPi^0_n$ if and only if it is definable as

\begin{equation}
A(\alpha) \iff \forall x_1 \: \dots \:  \Qu x_n \; \phi(x_1, \dots, x_n, \alpha,\gamma) \text{ holds (in the standard model).}
\end{equation}

where $\phi(x_1, \dots, x_n,\alpha,\underline{\gamma})$ is bounded, and $\Qu$ is ``$\forall$'' if $n$ is odd, and ``$\exists$'' if $n$ is even.

What do sets defined by bounded formulas look like? An atomic formula (without parameters) either contains no function variable at all, or it is of the form $\alpha(t_1) = t_2$. This implies that the truth of an atomic formula is determined by \textit{finitely many positions} in $\alpha$. This remains true if we consider logical combinations of atomic formulas, or even bounded quantification. Hence a bounded formula defines an open subset of $\Baire$.

On the other hand, the reals for which a bounded formula does not hold are definable by a bounded formula, too, since the negation of a bounded formula is again a bounded formula. We conclude that \textbf{bounded formulas define clopen subsets of $\Baire$}. On the other hand, if we have $\bSigma^0_1$-code for a set $A$ and its complement, we can decide the relation $A(\alpha\Rest{n})$ computably in the code.

Hence we can formulate the Normal Form above as follows.
$A \subseteq \Baire$ is $\bSigma^0_n$ if and only if there exists a clopen set $R \subseteq \Nat^n\times \Baire$

\begin{equation}
A(\alpha) \iff \exists x_1 \: \dots \: \Qu x_n \; R(x_1, \dots, x_n, \alpha),
\end{equation}

and similarly for $\bPi^0_n$ sets.

\subsubsection{Closure properties}

We can use the Normal Form to derive several closure properties of $\bSigma^0_n$ ($\bPi^0_n$).

If $P \subseteq \Nat \times \Baire$, we define the \textbf{projection of $P$ along $\Nat$}, $\exists^\Nat P$, as

\begin{equation}
\exists^\Nat P = \{ \alpha \colon \exists n \: P(n,\alpha)\}.
\end{equation}

We already encountered this operation in the definition of the effective Borel hierarchy (Definition~\ref{def-effective-Borel}). The dual operation is

\begin{equation}
\forall^\Nat P = \{ \alpha \colon \forall n \: P(n,\alpha)\}.
\end{equation}

\begin{proposition}For each $n \geq 1$, $\bSigma^0_n$ is closed under $\exists^\Nat$, and $\bPi^0_n$ is closed under $\forall^\Nat$.

\end{proposition}\begin{proof}We prove the result for $\Sigma^0_n$ (lightface). The boldface case follows by relativization, and the proof for $\bPi^0_n$ is completely dual.

Let $\phi(x_1, \dots, x_n, z, \alpha)$ be a bounded formula such that

\begin{equation}
A(z,\alpha) \iff \exists x_1 \: \dots \: \Qu x_n \; \phi(x_1, \dots, x_n, z, \alpha) \text{ holds}.
\end{equation}

Then

\begin{equation}
\exists^\Nat A(\alpha) \iff \exists x_0 \exists x_1 \: \dots \: \Qu x_n \; \phi(x_1, \dots, x_n, x_0,\alpha)
\end{equation}

We can collect two existential number quantifiers into one by using the pairing function $\Tup{.,.}$, or rather, its inverses, which we will denote by $(.)_0$ and $(.)_1$. (Recall that the pairing function is definable by a bounded formula.)
Then

\begin{equation}
\exists^\Nat A(\alpha) \iff \exists z_1  \: \dots \: \Qu z_n \; \phi((z_1)_1, \dots, z_n, (z_1)_0,\alpha),
\end{equation}

as desired.

\end{proof}One can use similar applications of coding and quantifier manipulation to prove a number of other closure properties, Often they follow also directly from the topological definitions, but it is good to have several techniques at hand.

\begin{proposition}\label{prop-borel-closure-finite}For all $n \geq 1$,

\begin{itemize}
\item \textbf{(a)}  $\bSigma^0_n$ is closed under countable unions and finite intersections.
\item \textbf{(b)} $\bPi^0_n$ is closed under finite unions and countable intersections.
\item \textbf{(c)} $\bDelta^0_n$ is closed under finite unions, finite intersections, and complements.
\end{itemize}

\end{proposition}\begin{proof}One can prove this by induction along the hierarchy. To obtain the closure under finite unions and intersections, one can use the following logical equivalences.
\begin{align*}
    \exists x \, P(x) \, \wedge \,  \exists y \, R(y) &\iff \exists x  \exists y \, (P(x) \, \wedge \,  R(y)) \\
    \forall x \, P(x) \, \vee \,  \forall y \, R(y) &\iff  \forall x  \forall y \, (P(x) \, \vee \,  R(y))
\end{align*}
\end{proof}Given $P \subseteq \Nat \times \Baire$, the \textbf{bounded projection} along $\Nat$
is defined as

\begin{equation}
\exists^\leq P = \{ (n,\alpha) \colon \exists m \leq n \: P(m,\alpha)\}.
\end{equation}

and the dual is

\begin{equation}
\forall^\leq P = \{ (n,\alpha) \colon \forall m \leq n \: P(m,\alpha)\}.
\end{equation}

\begin{proposition}\label{prop-borel-closure-bounded-projection}For all $n \geq 1$, $\bSigma^0_n$, $\bPi^0_n$, and $\bDelta^0_n$ are closed under $\exists^\leq$ and $\forall^\leq$.

\end{proposition}\begin{proof}In this case we use the computable coding function $\pi: \Nat \to \Nstr$.
We have the following equivalence, which immediately implies the closure properties for $\bSigma^0_n$ and $\bPi^0_n$, respectively, and hence also for $\bDelta^0_n$.

\begin{align*}
    \forall m \le n \, \exists k \; P(m,k) &\iff \exists k  \, \forall m \le n \:  P(m,\pi(k)_m)\\
    \exists m \le n  \, \forall k \; P(m,k) &\iff \forall k  \, \exists m \le n \;   P(m,\pi(k)_m)
\end{align*}
\end{proof}Finally, the levels of the Borel hierarchy are closed under continuous preimages.

\begin{proposition}\label{prop-borel_closure-preimages}For all $n \geq 1$, for any $A \subseteq \Baire$, and for any continuous $f: \Baire \to \Baire$, if $A$ is $\bSigma^0_n$ $(\bPi^0_n$, $\bDelta^0_n)$ then $f^{ -1}(A)$ is $\bSigma^0_n$ $(\bPi^0_n$, $\bDelta^0_n)$.

\end{proposition}\begin{proof}This follows easily by induction on $n$, since open and closed sets are closed under continuous preimages.

However, we can also argue via definability, since by Proposition~\ref{prop-product-continuous} one can represent a continuous function through a monotone mapping $\psi$ from finite strings to finite strings.  We have

\begin{equation}
f^{-1}(A) = \{ \alpha \colon A(f(\alpha)) \}.
\end{equation}

Let $R$ be clopen such that

\begin{equation}
A(\alpha) \iff \exists x_1 \: \dots \: \Qu x_n \; R(x_1, \dots, x_n, \alpha).
\end{equation}

Since clopen predicates depend only on a finite initial segment of $\alpha$, we can substitute $f(\alpha)$ for $\alpha$. The resulting formula defines $f^{ -1}(A)$, and is equivalent to a $\Sigma^0_n$-formula relative to a parameter coding the mapping $\psi$.

\end{proof}\subsubsection{Universal sets}

Let $\Gamma$ be a family of subsets defined in various Polish spaces. Of course we have in mind the classes $\bSigma^0_n$ or $\bPi^0_n$, but the concept of a \textbf{universal set} can be defined quite generally.

\begin{definition}\label{def-universal}Let $Y$ be a set. A set $U \subseteq X \times Y$ is \textbf{$Y$-universal for $\Gamma$} if $U \in \Gamma$, and for every set $A$ in $\Gamma$, there exists a $y \in Y$ such that

\begin{equation}
A = \{ x \colon (x,y) \in U \}.
\end{equation}

\end{definition}A universal set for $\Gamma$ can be thought of as a \textbf{parametrization} of $\Gamma$, the second component providing a \textbf{code} or \textbf{parameter} for each set in $\Gamma$.

A well-known example of a universal set is the \textbf{generalized halting problem},

\begin{equation}
K_0 = \{ (x,e) \colon \text{ the $e$-th Turing machine halts on input $x$} \}.
\end{equation}

In the sense of the above definition, $K_0$ is $\Nat$-universal for the family of recursively enumerable sets.

\begin{proposition}\label{prop-borel-universal}For any $n \geq 1$, there exists a set $U \subseteq \Baire \times \Baire$ that is $\Baire$-universal for $\bSigma^0_n$ ($\bPi^0_n$).

\end{proposition}\begin{proof}We can use the Borel codes defined in the previous lecture.

First of all, notice that for each $n \geq 1$, the set of all $\bSigma^0_n$ ($\BP{n}$)-codes is homeomorphic to $\Baire$. This follows easily from the definition of the Borel codes. Hence, if we fix $n$, every $\gamma\in \Baire$ represents a $\bSigma^0_n$ ($\bPi^0_n$)-code of a $\bSigma^0_n$ ($\bPi^0_n$) set, and every such set in turn has a code $\gamma \in \Baire$.

For fixed $n$, we let

\begin{equation}
U_n = \{ (\alpha,\gamma) \colon \gamma \in \Baire \text{ and $\alpha$ is in the $\bSigma^0_n$ $(\bPi^0_n)$ set coded by $\gamma$}\}.
\end{equation}

It follows easily from Theorem~\ref{thm-fundamental} that $U_n$ is $\bSigma^0_n$ ($\bPi^0_n$), too, and it is clear from the definition of $U$ that it parametrizes $\bSigma^0_n$ ($\bPi^0_n$).

\end{proof}The result can be generalized to hold for arbitrary Polish spaces $X$, i.e. for any $n \geq 1$, there exists a set $U \subseteq \Baire \times X$ that is $\Baire$-universal for $\bSigma^0_n(X)$ ($\bPi^0_n(X)$). To achieve this, one has to define Borel codes for $X$. This can be done by fixing a countable basis $(V_n)$ of the topology of $X$, and assign a sequence $\gamma \in \Baire$ the open set

\begin{equation}
U_\gamma = \bigcup_{n \in \Nat} V_{\gamma(n)}.
\end{equation}

The definition of codes for higher levels is then similar to Definition Definition~\ref{def-Borel-codes}.

As in the case of the halting problem, we can use the existence of universal sets to show that the levels of the Borel hierarchy are proper. The crucial point is that we can use universal sets to \textbf{diagonalize}.

\begin{theorem}\label{thm-borel-proper}For any $n \geq 1$, $\bSigma^0_n \neq \bPi^0_n$.

\end{theorem}\begin{proof}Let $U$ be an $\Baire$-universal set for $\bSigma^0_n$. Put

\begin{equation}
D = \{ \alpha \colon (\alpha, \alpha) \in U \}.
\end{equation}

Since $U$ is $\bSigma^0_n$, $D$ is $\bSigma^0_n$, too. Then $\Co{D}$ is $\bPi^0_n$, but cannot be $\bSigma^0_n$, for then there would exist $\beta$ such that

\begin{equation}
\Co{D} = \{ \alpha \colon (\alpha, \beta) \in U \},
\end{equation}

and thus

\begin{equation}
\beta \in D \iff (\beta, \beta) \in U \iff \beta \in \Co{D},
\end{equation}

a contradiction.

\end{proof}The diagonal set $D$ can obviously be defined for any universal set $U$, and hence the same proof yields a $\bPi^0_n$ set that is not $\bSigma^0_n$.

\begin{corollary}\label{cor-hier-proper}For any $n \geq 1$,
\begin{gather*}
    \bDelta^0_n \subsetneq \bSigma^0_n \subsetneq \bDelta^0_{n+1} \\
    \bDelta^0_n \subsetneq \bPi^0_n \subsetneq \bDelta^0_{n+1}.
\end{gather*}
\end{corollary}\begin{proof}Since $\BS{n} \nsubseteq \BP{n}$ and $\BP{n} \nsubseteq \BS{n}$, $\bDelta^0_n \subsetneq \BS{n},\BP{n}$. On the other hand if $\BS{n} = \bDelta^0_{n+1}$, then $\BS{n}$ would be closed under complements, and hence $\BS{n} = \BP{n}$, contradicting Theorem~\ref{thm-Borel-proper}.

\end{proof}\subsubsection{Borel sets of transfinite order}

We saw that the \textbf{Borel sets of finite order}

\begin{equation}
\operatorname{Borel}_\omega = \bigcup_{n < \omega} \BS{n}
\end{equation}

form a proper hierarchy. This fact also implies that $\Op{Borel}_\omega$ does not exhaust all Borel sets.

\begin{proposition}\label{prop-nonfinite-borel}There exists a Borel set $B$ that is not $\BS{n}$ for any $n \in \Nat$.

\end{proposition}\begin{proof}For every $n \in \Nat$, pick a set $B_n$ in $\BP{n} \setminus \BS{n}$. Put

\begin{equation}
B = \bigcup_{n \in \Nat} \{(n,\alpha) \colon \alpha \in B_n \}.
\end{equation}

Each of the sets in the union is Borel and hence $B$ is Borel. If $B$ were of finite order, it would be $\BS{k}$ for some $k \geq 1$. Since each $\BS{n}$ is closed under finite intersections, it follows that for all $m \geq 1$,

\begin{equation}
B \cap \Cyl{\Tup{m}}
\end{equation}

is $\BS{k}$. But $B \cap \Cyl{\Tup{m}}$ is homeomorphic to $B_m$, hence $B_m$ in $\BS{k}$ for all $m \geq 1$, contradiction.

\end{proof}We can extend the Borel hierarchy to arbitrary ordinals.

\begin{definition}\label{def-transfinite-borel}Let $X$ be a Polish space. Given an ordinal $\xi$, we define
\begin{align*}
    & \bSigma^0_\xi(X) = \{ \bigcup_k A_k \colon A_k \in \bPi^0_{\zeta_k}(X),\; \zeta_k < \xi \}, \\
    & \bPi^0_\xi(X) = \{ \Co{A} \colon A \in \bSigma^0_\xi(X) \} = \Co{\bSigma^0_\xi(X)}, \\
    & \bDelta^0_\xi(X) = \bSigma^0_\xi(X) \cap \bPi^0_\xi(X). \\
\end{align*}
\end{definition}It actually suffices to consider ordinals up to $\omega_1$, the first uncountable ordinal.

\begin{proposition}\label{prop-borel-omega1}For every Borel set $B$ there exists $\xi < \omega_1$ such that $B \in \BS{\xi}$.

\end{proposition}\begin{proof}If $B$ is open, this is clear. It is also clear if $B$ is the complement of a Borel for which the statement has been verified.

Assume finally that

\begin{equation}
B = \bigcup_n B_n, \; \text{ where each $B_n$ is Borel},
\end{equation}

and assume the statement holds for each $B_n$. For each $n$, let $\xi_n$ be a countable ordinal such that

\begin{equation}
B_n \in \BP{\xi_n}.
\end{equation}

Then

\begin{equation}
B \in \BS{\xi}, \; \text{ where $\xi = \sup \{\xi_n +1 \colon n \in \Nat\}$}.
\end{equation}

Since each $\xi_n$ is countable, $\xi$ is countable.

\end{proof}Borel sets of infinite order have the \textbf{same closure properties as their counterparts of finite order}. The proofs, however have to proceed by induction using the topological properties of $\BS{\xi}$ and $\BP{\xi}$, since the characterization via definability in arithmetic is no longer available -- the arithmetical hierarchy reaches only to $\omega$.

Similarly, the Hierarchy Theorem (Theorem~\ref{thm-Borel-proper}) extends to the transfinite levels. As the finite levels, this follows from the existence of universal sets for each level, which we now prove for the full hierarchy.

\begin{proposition}\label{prop-universal-general}For each $\xi < \omega_1$, there exists a $\Baire$-universal set for $\BS{\xi}$ $(\BP{\xi})$.

\end{proposition}\begin{proof}If $U$ is $\Baire$-universal for $\bSigma^0_\xi$, then

\begin{equation}
\neg U = \{ (\alpha,\gamma) \colon (\alpha,\gamma) \not\in U \}
\end{equation}

is $\Baire$-universal for $\bPi^0_\xi$, since for any $\bPi^0_\xi$ set $A$, $B = \neg A$ is $\bSigma^0_\xi$ and hence there exists a $\gamma$ such that

\begin{equation}
B = \{\beta \colon (\beta,\gamma) \in U \}
\end{equation}

and hence

\begin{equation}
A = \{ \alpha \colon (\alpha,\gamma) \not\in U \}.
\end{equation}

It remains to show that each $\BS{\xi}$ has an $\Baire$-universal set. By induction hypothesis, for every $\eta < \xi$ exists a $\Baire$-universal set $U_\eta$ for $\BP{\eta}$. Since $\xi$ is countable, we can pick a monotone  sequence of ordinals $(\xi_n)$ such that $\xi = \sup \{\xi_n + 1 \colon n < \omega \}$. Define

\begin{equation}
U_\xi =  \{ (\alpha, \gamma) \colon \exists n (\alpha, (\gamma)_n) \in U_{\xi_n} \},
\end{equation}

where $(\gamma)_n$ denotes the $n$th column of $\gamma$.

It is straightforward to check that $U_\xi$ is $\Baire$-universal for $\BS{\xi}$. (Note that any set $A$ in $\BS{\xi}$ can be represented as $\bigcup_n A_n$ with $A_n \in \BP{\xi_n}$, since $(\xi_n+1)$ is cofinal in $\xi$.)

\end{proof}The construction of the universal $\BS{\xi}$ set bears some resemblance to the construction of a $\bSigma^0_{n+1}$ code. It is indeed possible to formally define Borel codes for \textit{all} Borel sets.

\begin{definition}\label{def-borel-codes-transfinite}Let $\gamma \in \Baire$.

\begin{itemize}
\item Suppose $\gamma \in \Baire$ is such that $\gamma(0) = 1$ and $\gamma' \in \Baire$. $\gamma$ is a Borel code for the open set
\end{itemize}
\begin{equation*}
U = \bigcup_{\gamma'(\sigma) = 0} \Cyl{\sigma}
\end{equation*}

\begin{itemize}
\item If $\gamma$ is such that $\gamma(0)=2$ and $\gamma'$ is a Borel code for $A \subseteq \Baire$, we say $\gamma$ is a Borel code for $\Co{A}$.


\item If $\gamma$ is such that $\gamma(0)=3$ and for each $m$, $\gamma'_m$ is a Borel code of a set $A_m$, we say $\gamma$ is a Borel code for $\bigcup_n A_n$.
\end{itemize}

\end{definition}Any Borel code induces a well-founded tree (given by the coding nodes 1, 2,and 3). We can also consider Borel sets with computable codes. But there is no more straightforward connection with effective definability. It is possible to do this, but it requires a careful development of what it means to take effective unions along countable ordinals. We will return to it later.

Looking further ahead, one can show that \textbf{the set of all Borel codes is not Borel} (exercise -- use a diagonalization argument as in the proof of Theorem~\ref{thm-Borel-proper}). At the heart of this lies the fact that we cannot, in a Borel way, describe whether an arbitrary tree over $\Nat$ is well-founded or not. This will soon be a central topic when we turn our investigation to analytic and co-analytic sets.

%  This general proof of existence of universal sets does not use Borel codes, since those were defined only for Borel sets of finite order. The proof of {prf:ref}`prop-universal-general` provides an idea how we could extend the definition of a code to transfinite orders: Take unions of codes along a cofinal sequence. However, we would like to this in an effective way, and it is not clear how to do this for infinite ordinals in general. 
% 
% We will later return to this question, when we introduce computable ordinals}.
%

\subsection{Continuous Images of Borel sets}

In 1916, Nikolai Lusin asked his student Mikhail Souslin to study a paper by Henri Lebesgue. Souslin found a number of errors, including a lemma that asserted that the projection of a Borel is again Borel. In this lecture we will study the behavior of Borel sets under continuous functions. We will see that on the one hand every Borel set is the continuous image of a closed set, but that on the other hand continuous images of Borel sets are not always Borel.

This gives rise to a new family of sets, the \textbf{analytic} sets, which form a proper superclass of the Borel sets with interesting properties.

\subsubsection{Borel sets as continuous images of closed sets}

We have seen in Theorem~\ref{thm-polish-cont-image-Baire} that every Polish space is the continuous image of Baire space $\Baire$. As we will see now, we can strengthen this result.

\begin{theorem}[Lusin and Souslin]\label{thm-polish-bijection-baire}Let $X$ be a Polish space. Then there exists a closed subset $F \subseteq \Baire$ and a continuous bijection $f: F \to X$ that can be extended to a continuous surjection $g: \Baire \to X$.

\end{theorem}We have seen (Theorem~\ref{thm-Cantor-embedding}) that every uncountable Polish space contains a homeomorphic embedding of Cantor space. This was achieved by means of a \textbf{Cantor scheme}. To prove Theorem~\ref{thm-Polish-bijection-Baire}, we take up this idea again and adapt it to the Baire space.

\begin{definition}\label{def-lusin-scheme}A \textbf{Lusin scheme} on a set $X$ is a family $(F_\sigma)_{\sigma \in \Nstr}$ of subsets of $X$ such that

\begin{itemize}
\item \textbf{(i)} $\sigma \Sleq \tau$ implies $F_\sigma \supseteq F_\tau$,
\item \textbf{(ii)} for all $\tau \in \Nstr$, $i \neq j \in \Nat$, $F_{\tau\Conc \Tup{i}} \cap F_{\tau\Conc \Tup{j}} = \emptyset$.
\end{itemize}

If it has the additional property that

\begin{itemize}
\item \textbf{(iii)} $\diam(F_{\alpha \Rest{n}}) \to 0$  for $n \to \infty$,
\end{itemize}

then we can, similarly to a Cantor scheme, define the set
\begin{equation*}
D = \{\alpha \in \Baire \colon \bigcap_{n \in \Nat} F_{\alpha\Rest{n}} \neq \emptyset \}
\end{equation*}
and an \textbf{associated map} $f: D \to X$ by
\begin{equation*}
\{f(\alpha)\} = \bigcap_{n \in \Nat} F_{\alpha\Rest{n}}.
\end{equation*}
Properties (i)-(iii) ensure that $f$ is continuous and injective.

\end{definition}To prove the theorem we devise a Lusin scheme on $X$ such that $D$ will be closed, and $f$ will be a surjection, too. This is ensured by the following additional properties.

\begin{itemize}
\item \textbf{(a)} $F_\emptyset = X$,
\item \textbf{(b)} Each $F_\tau$ is $\BS[2]$,
\item \textbf{(c)} For each $\tau$, $\diam(F_\sigma) \leq 1/2^{|\sigma|}$,
\item \textbf{(d)} $F_\tau = \bigcup_{i \in \Nat} F_{\tau\Conc \Tup{i}} =  \bigcup_{i \in \Nat} \Cl{F_{\tau\Conc \Tup{i}}}$.
\end{itemize}

For this we have to show that every $\BS[2]$ set $F \subseteq X$ can be written, for given $\eps > 0$, as  $F= \bigcup_{i \in \Nat} F_i$, where the $F_i$ are pairwise disjoint $\BS[2]$ sets of diameter $< \eps$ so that $\Cl{F_i} \subseteq F$:

Let $F= \bigcup_{i \in \Nat} C_i$, where $C_i$ is closed, and $C_i \subseteq C_{i+1}$. Then $F= \bigcup_{i \in \Nat}(C_{i+1} \setminus C_i)$.

Let $(U_n)$ be a covering of $X$ with open sets of diameter $< \eps$. Put $D^{(i)}_n = U_n \cap (C_{i+1} \setminus C_i)$. Then $D^{(i)}_n$ is $\bDelta^0_2$. Now let $E^{(i)}_n = D^{(i)}_n \setminus (D^{(i)}_1 \cup \dots \cup D^{(i)}_{n -1})$.

Then $C_{i+1} \setminus C_i = \bigcup_{n \in \Nat} E^{(i)}_n$ where the $E^{(i)}_j$ are $\BS[2]$ sets of diameter $<\eps$. Therefore,

\begin{equation*}
F =  \bigcup_{i,n\in \Nat} E^{(i)}_n \; \text{ and } \;  \Cl{E^{(i)}_n} \subseteq \Cl{C_{i+1} \setminus C_i} \subseteq C_{i+1} \subseteq F.
\end{equation*}

The mapping $f$ associated with this Lusin scheme is surjective due to (a) and (d).
Furthermore, the domain $D$ of $f$ is closed: Suppose $\alpha_n \in D$, $\alpha_n \to \alpha$. Then $f(\alpha_n)$ is Cauchy, since for $\eps > 0$, there exists $N$ with $\diam(F_{\alpha\Rest{N}}) < \eps$ and $n_0$ such that $\alpha_n\Rest{N} = \alpha\Rest{N}$ for all $n \geq n_0$, so that $d(f(\alpha_n),f(\alpha_m)) < \eps$ whenever $n,m \geq n_0$. Since $X$ is Polish $f(\alpha_n) \to y$ for some $y \in X$.

By (d) we have $y \in \bigcap_n \Cl{F_{\alpha\Rest{n}}} = \bigcap_n F_{\alpha\Rest{n}}$, hence $\alpha \in D$ and $f(\alpha) = y$.

It remains to show that we can extend $f$ to a continuous surjection $g: \Baire \to X$. Say a closed subset $C$ of a topological space $Y$ is a \textbf{retract} of $Y$ if there exists a continuous surjection $g: Y \to C$ such that $g\Rest{C} = \Op{id}$.

\begin{lemma}\label{lem-closed-retract-baire}Every non-empty closed subset of $\Baire$ is a retract of $\Baire$.

\end{lemma}\begin{proof}Let $C \subseteq \Baire$ be closed, and let $T$ be a pruned tree such that $[T] = C$. We define a monotone mapping $\phi:\Nstr \to T$ such that $\phi(\sigma) = \sigma$ for all $\sigma \in T$. Then the induced (continuous) mapping $\phi^*: \Baire \to C$ is the desired retract.

Define $\phi$ by induction. Let $\phi(\Estr) = \Estr$. Given $\phi(\tau)$, let

\begin{equation}
\phi(\tau\Conc\Tup{m}) = \begin{cases}
        \tau\Conc\Tup{m} & \text{ if } \tau\Conc\Tup{m} \in T,\\
        \text{any } \phi(\tau)\Conc\Tup{k} \in T & \text{ otherwise}.
    \end{cases}
\end{equation}

Note that $k$ must exist since $T$ is pruned.

\end{proof}If we combine the retract function with $f$, we then obtain the desired surjection $\Baire \to X$. This concludes the proof of Theorem~\ref{thm-Polish-bijection-Baire}.

Refining the topology as in Theorem~\ref{thm-Borel-clopen}, we can extend the result from Polish spaces to Borel sets.

\begin{corollary}[Lusin and Souslin]\label{cor-borel-image-closed}For every Borel subset $B$ of a Polish space $X$ there exists a closed set $F \subseteq \Baire$ and a continuous bijection $f:F \to B$. Furthermore, $f$ can be extended to a continuous surjection $g:\Baire \to B$.

\end{corollary}\begin{proof}Enlarge the topology $\mathcal{O}$ of $X$ to a topology $\mathcal{O}_B$ for which $B$ is clopen.
By Theorem~\ref{thm-subsets-Polish}, $(B,\mathcal{O}_B\Rest{B})$ is a Polish space. By the previous theorem, there exists a closed set $F \subset \Baire$ and a continuous bijection $f:\Baire \to (B,\mathcal{O}_B\Rest{B})$. Since $\mathcal{O} \subseteq \mathcal{O}_B$, $f:F \to B$ is continuous for $\mathcal{O}$, too.

\end{proof}This theorem can be reversed in the following sense.

\begin{theorem}[Lusin and Souslin]\label{thm-borel-injective}Suppose $X,Y$ are Polish and $f:X \to Y$ is continuous. If $A \subseteq X$ is Borel and $f\Rest{A}$ is injective, then $f(A)$ is Borel.

\end{theorem}For a proof (which uses facts about analytic sets), see \cite{Kechris:1995a} (II.15.1).

\subsubsection{Images of Borel sets under arbitrary continuous functions}

As announced in the introduction, Borel sets are \textit{not} closed under \textit{arbitrary} continuous mappings.

\begin{theorem}[Souslin]\label{thm-souslin-borel-images}The Borel sets are not closed under continuous images.

\end{theorem}\begin{proof}Let $U \subseteq \Baire \times \Baire \times \Baire$ be $\Baire$-universal for $\BP[1](\Baire \times \Baire)$.
Define

\begin{equation*}
F:= \{(\alpha,\beta) \colon \exists \gamma  \; (\alpha,\gamma,\beta) \in U\}.
\end{equation*}

We claim that this set is \textit{$\Baire$-universal for the set of all continuous images of closed subsets of $\Baire$}:

On the one hand $F$ is a projection of a closed set, and projections are continuous. This implies that all the sets $F_\beta = \{ \alpha \colon (\alpha,\beta) \in F \}$ are continuous images of a closed set.

On the other hand, if $f: C \to \Baire$ is continuous with $C \subseteq \Baire$ closed (possibly empty) and $f(C) = A$, then
\begin{equation*}
\alpha \in A \iff \exists \gamma \; (\gamma,\alpha) \in \Op{Graph}(f) \iff \exists \gamma \; (\alpha,\gamma) \in  \Op{Graph}(f^{-1}).
\end{equation*}

Since $f$ is continuous, $\Op{Graph}(f)$ and hence also $\Op{Graph}(f^{ -1})$ are closed subsets of $\Baire \times \Baire$. Thus, by the universality of $U$, there exists $\beta$ such that

\begin{equation}
\Op{Graph}(f^{-1}) = U_\beta = \{ (\alpha,\gamma) \colon (\alpha,\gamma,\beta) \in U \},
\end{equation}

and hence

\begin{equation}
A = F_\beta.
\end{equation}

$F$ cannot be Borel: Otherwise $D_F = \{\alpha \colon (\alpha,\alpha) \not\in F \}$ were Borel. By Corollary~\ref{cor-Borel-image-closed}, every Borel set is the image of a closed set under a continuous mapping. This implies that $D_F = F_\beta$. But then

\begin{equation}
\beta \in D_F \iff \beta \in F_\beta \iff (\beta,\beta) \in F \iff \beta \not\in D_F,
\end{equation}

contradiction.

\end{proof}

\section{Projective sets}

\subsection{Analytic Sets}

\begin{definition}A subset $A$ of a Polish space $X$ is \textbf{analytic} if it is empty or there exists a continuous function $f:\Baire \to X$ such that $f(\Baire) = A$.

\end{definition}We will later see that the analytic sets correspond to the sets definable by means of $\Sigma^1_1$ formulas, that is formulas in the language of second order arithmetic that have \textbf{one existential function quantifier}.

Therefore, we will denote the analytic subsets of $X$ also by

\begin{equation}
\PS{1}(X).
\end{equation}

Here are some simple properties of analytic sets.

\begin{proposition}\label{prop-prop-analytic}\begin{itemize}
\item \textbf{(i)} Every Borel set is analytic.
\item \textbf{(ii)} A continuous image of analytic set is analytic.
\item \textbf{(iii)} Countable unions of analytic sets are analytic.
\end{itemize}

\end{proposition}\begin{proof}(i): This follows directly from Corollary~\ref{cor-Borel-image-closed}.

(ii): The composition of continuous mappings is continuous.

(iii): Let $A_n$ be analytic and $f_n:\Baire \to X$ such that $f_n(\Baire) = A_n$. Define $f: \Baire \to X$ by

\begin{equation}
f(m,\alpha) = f_n(\alpha).
\end{equation}

Then $f$ is continuous and $f(\Baire) = \bigcup_n A_n$.

\end{proof}We can use our previous results about Borel sets to give various equivalent characterizations of analytic sets.

\begin{proposition}\label{characterization-analytic}For a subset $A$ of a Polish space $X$, the following are equivalent.

\begin{itemize}
\item \textbf{(i)} $A$ is analytic,
\item \textbf{(ii)} $A$ is empty or there exists a Polish space $Y$ and a continuous $f:Y \to X$ such that $f(Y) = A$,
\item \textbf{(iii)} $A$ is empty or there exists a Polish space $Y$, a Borel set $B \subseteq Y$ and a continuous $f:Y \to X$ such that $f(B) = A$.
\item \textbf{(iv)} $A$ is the \textbf{projection} of a closed set $F \subseteq \Baire \times X$ along $\Baire$,
\item \textbf{(v)} $A$ is the projection of a $\BP{2}$ set $G \subseteq  \Cant \times   X$ along $\Cant$,
\item \textbf{(vi)} $A$ is the projection of a Borel set $B \subseteq X\times Y$ along $Y$, for some Polish space $Y$.
\end{itemize}

\end{proposition}\begin{proof}(i) $\Leftrightarrow$ (ii): Follows from Theorem~\ref{thm-polish-cont-image-Baire} and Proposition~\ref{prop-prop-analytic} (ii).

(ii) $\Leftrightarrow$ (iii): Follows from Corollary Corollary~\ref{cor-Borel-image-closed} and Proposition~\ref{prop-prop-analytic} (ii).

(i) $\Rightarrow$ (iv):  Let $f:\Baire \to X$ be continuous, $f(\Baire) = A$. Then

\begin{equation}
x \in A \iff \exists \alpha \; (\alpha,x) \in \Op{Graph}(f),
\end{equation}

hence $A$ is the projection of the closed set $\Op{Graph}(f)$ along $\Baire$.

(iv) $\Rightarrow$ (iii): Clear, since projections are continuous.

(iv) $\Rightarrow$ (v): $\Baire$ is homeomorphic to a $\BP{2}$ subset of $\Cant$. (Exercise!)

(v) $\Rightarrow$ (vi), (vi) $\Rightarrow$ (iii): Obvious.

\end{proof}\subsubsection{The Lusin Separation Theorem}

In a course on computability theory one learns that there are \textbf{effectively inseparable} disjoint computably enumerable sets. i.e. disjoint c.e. sets $W,Z \subseteq \Nat$ for which no recursive set $A$ exists with $W \subseteq A$ and $A \cap Z = \emptyset$.

In contrast to this, disjoint analytic sets can always be separated by a Borel set -- they are \textbf{Borel separable}.

\begin{theorem}[Lusin]\label{thm-lusin-separation}Let $A, B \subseteq X$ be disjoint analytic sets. Then there exists a Borel $C \subseteq X$ such that

\begin{equation}
A \subseteq C \quad \text{ and } \quad B \cap C = \emptyset,
\end{equation}

\end{theorem}\begin{proof}Let $f:\Baire \to A$ and $g:\Baire \to B$ be continuous surjections.

We argue by contradiction. The key idea is: if $A$ and $B$ are Borel inseparable, then, for some $i,j \in \Nat$, $A_{\Tup{i}} = f(\Cyl{\Tup{i}})$ and $B_{\Tup{j}} = g(\Cyl{\Tup{j}})$ are Borel inseparable.

This follows from the observation

\begin{quote}
$(*)\quad$ if the sets  $R_{m,n}$ separate the sets  $P_m, \, Q_n$ (for each $m,n$), then $R = \bigcup_m \bigcap_n R_{m,n}$ separates the sets $P =  \bigcup_m P_m, \, Q =  \bigcup_n Q_n.$
\end{quote}

So, by using $(*)$ repeatedly, we can construct sequences $\alpha, \beta \in \Baire$ such that for all $n$,
$A_{\alpha\Rest{n}}$ and $B_{\beta\Rest{n}}$ are Borel inseparable, where

\begin{equation}
A_{\sigma} = f(\Cyl{\sigma}) \quad \text{ and } \quad B_{\sigma} = g(\Cyl{\sigma}).
\end{equation}

Then we have $f(\alpha) \in A$ and $g(\beta) \in B$, and since $A$ and $B$ are disjoint, $f(\alpha) \neq g(\beta)$. Let $U,V$ be disjoint open sets such that $f(\alpha) \in U$, $g(\beta) \in V$. Since $f$ and $g$ are continuous, there exists $N$ such that $f(\Cyl{\alpha\Rest{N}}) \subseteq U$, $g(\Cyl{\beta\Rest{N}}) \subseteq V$, hence $U$ separates $A_{\alpha\Rest{N}}$ and $B_{\beta\Rest{N}}$, contradiction.

\end{proof}The Separation Theorem yields a nice characterization of the Borel sets.

\begin{theorem}[Souslin]\label{borel-delta11}If a set $A$ and its complement $\Co{A}$ are both analytic, then $A$ is Borel.

\end{theorem}\begin{proof}In Theorem~\ref{thm-Lusin-separation}, chose $A = A$ and $B = \Co{A}$.

\end{proof}It follows from Theorem~\ref{thm-Souslin-Borel-images} and the Theorem~\ref{thm-Lusin-separation} that the analytic sets are not closed under complements.

Sets whose complement is analytic are called \textbf{co-analytic}. Analogous to the levels of the Borel hierarchy, the co-analytic subsets of a Polish space $X$ are denoted by

\begin{equation}
\PP{1}(X).
\end{equation}

If we define, again analogy to the Borel hierarchy,

\begin{equation}
\bDelta^1_1(X) = \PS{(1)}(X) \cap \PP{1}(X),
\end{equation}

then Souslin's Theorem states that

\begin{equation}
\Op{Borel}(X) = \bDelta^1_1(X).
\end{equation}

\subsubsection{The Souslin operation}

Souslin schemes give an alternative presentation of analytic sets which will be useful later.

\begin{definition}A \textbf{Souslin scheme} on a set $X$ is a family $P = (P_\sigma)_{\sigma\in \Nstr}$ of subsets of $X$ indexed by $\Nstr$.

The \textbf{Souslin operation} $\mathcal{A}$ for a Souslin scheme is given by

\begin{equation}
\mathcal{A}P = \bigcup_{\alpha \in \Baire}  \bigcap_{n \in \Nat} P_{\alpha\Rest{n}}.
\end{equation}

This means
\begin{equation*}
\tag{$**$}
    x \in \mathcal{A}P \iff \exists \alpha \in \Baire \; \forall n \in \Nat \; x \in P_{\alpha\Rest{n}}.
\end{equation*}
\end{definition}The analytic sets are precisely the sets that can be obtained by Souslin operations on closed sets. If a $\Gamma$ is a family of subsets of a set $X$, we let

\begin{equation}
\mathcal{A}\Gamma = \{\mathcal{A}P \colon \text{ $P = (P_\sigma)$ is a Souslin scheme with $P_\sigma \in \Gamma$ for all $\sigma$} \}.
\end{equation}

\begin{theorem}\label{analytic-souslin-op}\begin{equation}
\PS{1}(X)\; = \; \mathcal{A}\,\BP{1}(X).
\end{equation}

\end{theorem}\begin{proof}Suppose $f: \Baire \to X$ is continuous with $f(\Baire) = A$. Then
\begin{equation*}
x \in A \iff \exists \alpha  \in \Baire \; \forall n \in \Nat \; x \in \Cl{f(\Cyl{\alpha\Rest{n}})}.
\end{equation*}

Hence if we let $P_\sigma = \Cl{f(\Cyl{\sigma})}$, then
\begin{equation*}
A =  \mathcal{A} \,P,
\end{equation*}
for the Souslin scheme $P = (P_\sigma)$.

To see that any set $A$ in $\mathcal{A}\,\BP{1}(X)$ is analytic, consider ($**$). If the $P_\sigma$ are closed, the condition

\begin{equation}
(\alpha,x) \in F \iff \forall n \in \Nat \; x \in P_{\alpha\Rest{n}}
\end{equation}

defines a closed subset of $\Baire \times X$ such that $A$ is the projection of $F$ along $\Baire$.

\end{proof}Note that the Souslin scheme $(P_\sigma)$ used in the previous proof has the additional property that

\begin{equation}
\sigma \Sleq \tau \quad \Rightarrow \quad P_\sigma \supseteq P_\tau.
\end{equation}

Such Souslin schemes are called \textbf{regular}. By replacing any Souslin scheme $P_\sigma$ with

\begin{equation}
Q_\sigma = \bigcap_{\tau \Sleq \sigma} P_\tau,
\end{equation}

we obtain a regular Souslin scheme $Q = (Q_\sigma)$ with $\mathcal{A} \, Q = \mathcal{A}\, P$. Moreover, if the $P_\sigma$ are from a class $\Gamma$, and $\Gamma$ is closed under finite intersections, then the $Q_\sigma$ are also from $\Gamma$. In particular, \textbf{any analytic set can be obtained from a regular Souslin scheme of closed sets} via the Souslin operation.

\include{descriptive_set_theory-regularityanalytic}

\include{descriptive_set_theory-projective}

\subsection{Co-Analytic sets}

We will see that, in many ways, $\PP{1}$ sets form the frontier between classical descriptive set theory and metamathematics. This chapter can be seen as the start of our transition to metamathematics. We will detail the distinguished role well-founded relations play in the analysis of $\PP{1}$ sets.

\subsubsection{Normal forms}

Analytic sets are projections of closed sets. Closed sets are in $\Baire\times \Baire$ are infinite paths through \textbf{trees on $\Nat \times \Nat$}, i.e. two-dimensional trees.

\begin{definition}\label{def-two-dim-tree}A set $T \subseteq \Nstr \times \Nstr$  is a \textbf{two-dimensional tree} if

\begin{itemize}
\item \textbf{(i)} $(\sigma,\tau) \in T$ implies $|\sigma|=|\tau|$ and
\item \textbf{(ii)} $(\sigma,\tau) \in T$ implies $(\sigma\Rest{n},\tau\Rest{n}) \in T$ for all $n \leq |\sigma|$.
\end{itemize}

An \textbf{infinite branch} of $T$ is a pair $(\alpha,\beta) \in \Baire\times \Baire$ so that
\begin{equation*}
\forall n\in \Nat \; (\alpha\Rest{n},\beta\Rest{n}) \in T.
\end{equation*}
\end{definition}As in the one-dimensional case, we use $[T]$ to denote the set of all infinite paths through $T$. It follows that $A \subseteq \Baire$ is analytic if and only if there exists a two-dimensional tree $T$ on $\Nat \times \Nat$ such that
\begin{align*}
	\alpha \in A & \iff  \exists \beta \: (\alpha,\beta) \in [T]\\
	             & \iff  \exists \beta \, \forall n \: (\alpha\Rest{n},\beta\Rest{n}) \in T.
\end{align*}

Another way to write this is to put, for given $T$ and $\alpha \in \Baire$,

\begin{equation}
T(\alpha) = \{ \tau \colon (\alpha\Rest{|\tau|}, \tau) \in T \}.
\end{equation}

Then we have, with $T$ witnessing that $A$ is analytic,

\begin{equation}
\alpha \in A \iff T(\alpha) \text{ has an infinite path } \iff T(\alpha) \text{ is not well-founded}.
\end{equation}

We obtain the following normal form for co-analytic sets.

\begin{proposition}[Normal form for co-analytic sets]\label{prop-norm-form-coanalytic}A set $A \subseteq \Baire$ is $\PP{1}$ if and only if there exists a two-dimensional tree $T$ such that

\begin{equation}
\alpha \in A \iff T(\alpha) \text{ is well-founded}.
\end{equation}

\end{proposition}If $A$ is (lightface) $\Pi^1_1$, then there exists a computable such $T$, and the mapping $\alpha \mapsto T(\alpha)$ is computable, as a mapping between reals and trees (which can be coded by reals). This relativizes, i.e. for a $\Pi^1_1(\gamma)$ set, the mapping $\alpha \mapsto T(\alpha)$ is computable in $\gamma$. Since computable mappings are continuous, we obtain that the in the above proposition, the mapping $\alpha \mapsto T(\alpha)$ is continuous.

\subsubsection{$\mathbf{\Pi}^1_1$-complete sets}

How does one show that a specific set is \textit{not} Borel? A related question is: Given a definition of a set in second order arithmetic, how can we tell that there is not an easier definition (in the sense that it uses less quantifier changes, no function quantifiers etc.)? The notion of \textbf{completeness} for classes in Polish spaces provides a general method to answer such questions.

\begin{definition}\label{def-wadge}Let $X,Y$ be Polish spaces. We say a set $A \subseteq X$ is \textbf{Wadge reducible} to $B \subseteq Y$, written $A \leq_{\W} B$, if there exists a continuous function $f: X \to Y$ such that

\begin{equation}
x \in A \iff f(x) \in B.
\end{equation}

\end{definition}The important fact about Wadge reducibility is that it preserves classes closed under continuous preimages.

\begin{proposition}\label{prop-wadge-preimages}Let $\Gamma$ be a family of subsets in Polish spaces (such as the classes of the Borel or projective hierarchy). If $\Gamma$ is closed under continuous preimages, then $A \leq_{\W} B$ and $B \in \Gamma$ implies $A \in \Gamma$.

\end{proposition}\begin{proof}If $A \leq_{\W} B$ via $f$, then $A = f^{ -1}(B)$.

\end{proof}\begin{definition}\label{def-completeness}A set $A \subseteq X$ is \textbf{$\Gamma$-complete} if $A \in \Gamma$ and for all $B \in \Gamma$, $B \leq_{\W} A$.

\end{definition}$\Gamma$-complete sets can be seen as the most complicated members of $\Gamma$. In particular, for the $\bSigma/\bPi$ classes complete sets cannot be members of the dual class. For instance, a $\PP{1}$-complete set cannot be $\PS{1}$, since this would mean it is Borel, and hence every $\PP{1}$ set would be Borel, which we have seen is not true.

If $A \subseteq \Baire \times \Baire$ is $\Baire$-universal for some class $\Gamma$ in the Borel or projective hierarchy, then the set

\begin{equation}
\{ \alpha \oplus \beta \colon (\alpha,\beta) \in A \}
\end{equation}

is $\Gamma$-complete, where $\oplus$ here denotes the pairing function for reals

\begin{equation}
\alpha\oplus\beta(n) = \begin{cases}
	 	\alpha(k) & n = 2k, \\
		\beta(k) & n = 2k+1.
	\end{cases}
\end{equation}

Since $\oplus$ is continuous, and $B \in \Gamma$ if and only if $B = A_{\gamma}$ for some $\gamma\in \Baire$, we have in that case that $B \leq_{\W} A$ via the mapping

\begin{equation}
f(\beta) = \gamma\oplus\beta.
\end{equation}

It follows that complete sets exist for all levels of the Borel and projective hierarchy. However, the universal sets they are based on are rather abstract objects. Complete sets are most useful when we can show that a \textit{specific property} implies completeness. We will encounter next an important example for the class of co-analytic sets.

\subsubsection{Well-founded relations and well-orderings}\label{sec-well-founded}

Given a real in $\beta \in \Baire$, we can associate with $\beta$ a binary relation $E_\beta$ on $\Nat$:

\begin{equation}
E_\beta(m,n) :\iff \beta(\Tup{m,n}) = 0.
\end{equation}

A binary relation $E$ on a set $X$ is \textbf{well-founded} if every non-empty $Y \subseteq X$ has an $E$-minimal element $y_0$, that is, there is no $z \in Y$ with $E(z,y)$.

Let

\begin{equation}
\WF = \{\beta \in \Baire \colon \text{$E_\beta$ is well-founded} \}.
\end{equation}

Then

\begin{equation}
\beta \in \WF \iff \forall \gamma \in \Baire \: \exists n \: \forall m \; [ \gamma(n) E_\beta \gamma(m) ],
\end{equation}

and hence $\WF$ is $\Pi^1_1$.

A closely related set is

\begin{equation}
\WOrd = \{\beta \in \Baire \colon \text{$E_\beta$ is a well-ordering} \}.
\end{equation}

Then

\begin{equation}
\beta \in \WOrd \iff \beta \in \WF  \text{ and $E_\beta$ is a linear ordering}.
\end{equation}

Coding a linear order is easily seen to be $\Pi^0_1$, hence $\WOrd$ is $\Pi^1_1$, too.

\begin{theorem}\label{thm-wf-wadge-complete}The sets $\WF$ and $\WOrd$ are $\PP{1}$-complete.

\end{theorem}\begin{proof}We have seen in the chapter on \href{/trees}{Trees} that a tree has an infinite path  if and only if the inverse prefix ordering is ill-founded. Trees can be coded as reals, and hence Proposition~\ref{prop-norm-form-coanalytic} yields immediately that $\WF$ is $\PP{1}$-complete.

For $\WOrd$ we use the Kleene-Brouwer ordering and refer to Proposition~\ref{prop-KB-wellorder}.

\end{proof}The theorem lets us gain further insights in the structure of co-analytic sets. If $\alpha \in \Baire$ codes a well-ordering on $\Nat$, let
\begin{equation*}
\Norm{\alpha} = \text{ order type of well-ordering coded by $\alpha$}.
\end{equation*}

It is clear that $\Norm{\alpha} < \omega_1$. For a fixed ordinal $\xi < \omega_1$, we let
\begin{equation*}
\WOrd_\xi = \{ \alpha \in \WOrd \colon \Norm{\alpha} \leq \xi \}.
\end{equation*}

\begin{lemma}\label{lem-bounded-rank-borel}For any $\xi < \omega_1$, the set $\WOrd_\xi$ is Borel.

\end{lemma}\begin{proof}Let $\alpha \in \Baire$. We say $m \in \Nat$ is in the \textbf{domain} of $E_\alpha$, $m \in \Op{dom}(E_\alpha)$, if
\begin{equation*}
\exists n \: [ m E_\alpha n \; \vee \; n E_\alpha m].
\end{equation*}

It is clear from the definition of $E_\alpha$ that $\Op{dom}(E_\alpha)$ is Borel. For $\xi < \omega_1$, let
\begin{equation*}
B_\xi = \{ (\alpha,n) \colon E_\alpha \Rest{\{m \colon m E_\alpha n\}} \text{ is a well-ordering of order type $\leq \xi$} \}
\end{equation*}

We show by transfinite induction that every $B_\xi$ is Borel. Suppose $B_\zeta$ is Borel for all $\zeta < \xi$. Then, since $\xi$ is countable, $\bigcup_{\zeta < \xi} B_\zeta$ is Borel, too. But
\begin{equation*}
(\alpha,n) \in B_\xi \iff \forall m \: [m E_\alpha n \: \Rightarrow \: (\alpha,m) \in \bigcup_{\zeta < \xi} B_\zeta],
\end{equation*}
and from this it follows that $B_\xi$ is Borel. Finally, note that
\begin{equation*}
\alpha \in \WOrd_\xi \iff \forall n \; [n \in \Op{dom}(E_\alpha) \: \Rightarrow \: (\alpha,n) \in B_\xi],
\end{equation*}
which implies that $\WOrd_\xi$ is Borel.

\end{proof}\begin{corollary}\label{cor-coanal-union-borel}Every $\PP{1}$ set is a union of $\aleph_1$ many Borel sets.

\end{corollary}\begin{proof}Since $\WOrd$ is $\PP{1}$-complete, every co-analytic set $A$ is the preimage of $\WOrd$ for some continuous function $f$. We have
\begin{equation*}
\WOrd = \bigcup_{\xi < \omega_1} \WOrd_\xi,
\end{equation*}
and hence
\begin{equation*}
A = \bigcup_{\xi < \omega_1}  f^{-1}(\WOrd_\xi).
\end{equation*}
Since continuous preimages of Borel sets are Borel, the result follows.

\end{proof}If we work instead with the set
\begin{equation*}
C_\xi = \{ \alpha \colon \alpha \in \WOrd_\xi \text{ or } \exists n\in \Op{dom}(E_\alpha) \\ [E_\alpha \Rest{\{m \colon m E_\alpha n\}} \text{ is a well-ordering of order type $\xi$}] \},
\end{equation*}
then we get that $\WOrd = \bigcap_{\xi < \omega_1} C_\xi$, and hence

\begin{corollary}\label{cor-aleph-union-intersect}Every $\PP{1}$ set can be obtained as a union or intersection of $\aleph_1$-many Borel sets. Consequently, the same holds for every $\PS{1}$ set.

\end{corollary}The previous results allow us to solve the cardinality problem of co-analytic sets at least partially.

\begin{corollary}\label{cor-coanalytic-cardinality}Every $\PP{1}$ set is either countable, of cardinality $\aleph_1$, or of cardinality $2^{\aleph_0}$.

\end{corollary}We conclude the chapter with another application of  Lemma~\ref{lem-bounded-rank-Borel} -- a useful tool for analyzing $\bSigma^1_1$ sets:

\begin{theorem}[-bounding]\label{thm-sigma11-bounding}\textbf{$\bSigma^1_1$}\\
For every analytic $A \subseteq \WOrd$ there exists an ordinal $\nu < \omega_1$ such that

\begin{equation}
\forall x \in A \;\; \Norm{x} < \nu.
\end{equation}

\end{theorem}\begin{proof}If such a $\nu$ did not exist, then

\begin{equation}
\alpha \in \WOrd \iff \exists \nu \: [\alpha \in A \; \wedge \; \WOrd_\nu].
\end{equation}

The right-hand side is $\bSigma^1_1$, and hence $\WOrd$ would be $\bSigma^1_1$, contradiction.

\end{proof}An analogous statement holds for $\WF$, with respect to the rank function $\rho$ of a well-founded relation.

%  ## Uniformization
% 
% Two important structural properties of pointclasses are **separation** and **uniformization**.
% 
% We have seen an instance of separation in

\section{Axiomatic set theory}

\subsection{The Axioms of Set Theory}

In the previous chapters, we have repeatedly brought up a metamathematical context, such as the use of the Axiom of Choice, or Solovay's model in which every set of reals is measurable. But we have not really distinguished between results in a formal theory and in the metatheory, mostly because we did not really establish a formal theory to begin with. We have treated descriptive set theory like most other mathematical theories in that we defined our basic objects (Polish spaces, Borel sets, etc.) and then started proving facts about them ``the usual way'', as we would prove facts about commutative rings or locally compact topological spaces. But in order to make better sense of the metamathematical issues, we have to ``pause'' and talk a bit about the axioms of set theory.

\subsubsection{Comprehension and Russell Antinomy}

To develop set theory formally as a theory of first order logic, we first need to fix the \textbf{language}. We want to consider the notion of a \textit{set} as foundational (with the intention to develop everything else from it), we provide only \textbf{one binary relation symbol}, $\in$. We will denote this as the \textbf{language of set theory}, $\mathcal{L}_\in$.

How can we axiomatize the concept of set?

In his famous 1895 paper ``\textit{Beiträge zu Begründung der transfiniten Mengenlehre}'' \cite{Cantor_1895s}, Georg Cantor writes

\begin{quote}
Unter einer ``\textit{Menge}'' verstehen wir jede Zusammenfassung $M$ von bestimmten wohlunterschiedenen Objekten unserer Anschauung oder unseres Denkens (welche die ``\textit{Elemente}'' von  $M$  genannt werden)  zu einem Ganzen.
\end{quote}

This can be translated approximately as: \textit{A set is any collection of certain distinct objects of our intuition or our thought into a whole}.

We can try to make this more precise as follows:

\begin{quote}
For every property $P(x)$ exists a set $M$ of all objects $x$ having property $P$:  $M =\{x: M(x) \}$.
\end{quote}

This can be formalized in the language of set theory as an \textbf{axiom scheme}: For every $\mathcal{L}_\in$-formula $\varphi(x)$,

\begin{quote}
(\textbf{Comprehension})\textsubscript{$\varphi$} $\exists y \, \forall x \; ( x \in y \leftrightarrow \varphi(x))$
\end{quote}

This axiom, however, is \textbf{contradictory}.

\begin{framed}
\textbf{Russell's antinomy (1903)}\\
If we choose $\varphi(x) = \neg x \in x$, then the Comprehension axiom yields the existence of a set $r$ with
\begin{equation*}
\forall x ( x \in  r \leftrightarrow x \not \in x).
\end{equation*}
In particular, for $x = r$:
\begin{equation*}
r \in  r \leftrightarrow r \not \in r ,
\end{equation*}
contradiction!
\end{framed}

We obtain similar contradictions if we choose as $\varphi(x)$ the formula
\begin{align*}
     x = x  \qquad &  \text{antinomy of the set of all sets (Cantor)} \\
     x\;\text{is a cardinal} \qquad & \text{Cantor, around 1899, published 1932}\\
     x \; \text{is an ordinal} \qquad &   \text{antinomy of Burali-Forti}
\end{align*}
These antinomies are, however, not as direct as Russell's and require some further development of the theory in order to derive a contradiction.

Regarding the existence of sets, we have to distinguish between

\begin{itemize}
\item \textbf{classes}, which we will denote by capital letters $A,B,\dots$ (for some specific, important classes we will also use boldface) and
\item \textbf{sets}, denoted in this context by lower case letters $a,b,\ldots,x,y \ldots$.
\end{itemize}

An arbitrary property $\varphi(x)$ will define a corresponding class

\begin{equation}
A = \{x \colon \varphi(x)\}
\end{equation}

As we saw above, classes are not necessarily sets: some are ``\textit{too large}'' to be a set, as for example the \textbf{class of all sets},

\begin{equation}
\mathbf{V} = \{ x \colon x = x \}
\end{equation}

Classes that are not sets are called \textbf{proper classes}.

You should keep in mind that, in the formal theory $\ZF$, we do not have variables for classes, so the definition above is \textit{informal}.  Any class variable, as well as expressions like $a \in A$, should be seen as \textit{abbreviations} for a formal expression using the underlying formula.
There are formal systems (such as \textbf{Bernays-Gödel set theory}) that use classes explicitly, but they are used less frequently.

\subsubsection{The Axioms of $\mathsf{ZFC}$}

We begin with the \textbf{Axiom of Extensionality}, which is essential for the equality relation between sets.

\begin{quote}
(\textbf{Extensionality})  $\qquad \forall x (x \in a  \leftrightarrow x \in b)  \to a=b.$
\end{quote}

Consequently, two sets coincide if they have the same elements.

The basic idea of the \textbf{Zermelo-Fraenkel} axiom system $\ZF$ is that we avoid introducing sets that are ``too large'' (and hence would lead to contradictions) by allowing new sets only if they can be \textbf{``generated'' from a given set} by a number of fixed, well-behaved operations.

So let us postulate that at least one set exists:

\begin{quote}
(\textbf{Set Existence}) $\qquad \exists x ( x = x )$
\end{quote}

This axiom is not strictly necessary, as the existence of a set also follows from other axioms in $\ZF$ (or usually even from the underlying axioms of first-order logic). But it is good to have it as a starting point here for emphasis.

We have seen we cannot use full comprehension for sets. In its place we introduce the scheme of

\begin{quote}
(\textbf{Separation})\textsubscript{$\varphi$} $\qquad  \forall a \exists y \forall x (x \in y \leftrightarrow x \in a \wedge \varphi(x,\ldots))$
\end{quote}

By Extensionality, the set $y$ is unique.

Separation allows us to select from any class $\{x \colon \varphi(x,\ldots)\}$ those elements that are in a given set $a$ and collect them in a \textbf{set}
\begin{equation*}
\{x \in a \colon \varphi(x,\ldots)\}
\end{equation*}

Next, we have

\begin{quote}
(\textbf{Pairing}) $\qquad \exists y \forall x( x \in y \leftrightarrow  x = a \vee x = b)$.
\end{quote}

This axioms allows forming \textit{pairs of sets}, specifically

\begin{align*}
    \{a,b\} & : =  \{x\colon x=a \; \vee \; x=b \} &    \\
    \{a\} &: = \{x\colon x=a  \} &    \text{singleton set}\\
    (a,b) &: =  \{\{a\},\{a,b\}\} &    \text{ordered pair}
\end{align*}

While for the \textit{pair set} $\{a,b\} = \{b,a\}$ the order is not important, we have for the \textit{ordered pair}
\begin{equation*}
(a,b) = (c,d) \leftrightarrow a = c \wedge b = d.
\end{equation*}
Hence, we can introduce \textbf{binary relations} as classes of ordered pairs
\begin{gather*}
    \Op{Rel}(R) :\leftrightarrow  \forall u \in R \; \exists x,y  \; (u =(x,y)).
\end{gather*}

As usual, by identifying \textbf{functions} with their graphs, we can introduce functions as a special kind of relation:
\begin{gather*}
F: A \to B :\leftrightarrow  \forall x \in A \; \exists ! y \in B \; (x,y) \in F.
\end{gather*}
We write $\Op{Fun}(a)$ to denote that fact that $a$ is a function.

Further elementary axioms:

\begin{quote}
(\textbf{Union})   $\qquad \forall a \exists y \forall z (z \in y  \leftrightarrow  \exists x \in a \; z \in x)$
\end{quote}

\begin{quote}
(\textbf{Replacement})\textsubscript{F} \newline

$\qquad \forall a  ((F: a \to \mathbf{V}) \: \rightarrow \: \exists u \forall y (y \in u \leftrightarrow \exists x \in a \; y = F(x)))$
\end{quote}

\begin{quote}
(\textbf{Power Set}) $\qquad \forall a \exists y \forall z (z \in y \leftrightarrow z \subseteq a)$
\end{quote}

It follows that, for a given set $a$, the following classes are sets:

\begin{align*}
\bigcup \, a = \bigcup_{x \in a} x & : = \{z \colon \exists x \in a \; z \in x \}  &  \qquad  \text{union}\\
F[a] = \{F(x)|x \in a\} &: = \{y\colon \exists x \in a \; y = F(x) \}  &  \qquad  \text{image set}\\
\mathcal{P}(a) &: = \{x\colon x \subseteq a\} & \qquad  \text{power set}
\end{align*}

\begin{quote}
(\textbf{Infinity}) $\qquad \exists x ( \emptyset \in x \wedge \forall y ( y \in x \to y \cup \{y\} \in x))$
\end{quote}

The axiom of Infinity is a ``pure'' set existence axiom, that is, it does not depend on another set already existing. It therefore renders the axiom of Set Existence above redundant.
It also implies the existence of the set $\Nat$ of natural numbers (along with the operation $+$), which we will address in more detail below.

Using $\Nat$, we can introduce the other basic number sets:

\begin{itemize}
\item $\Integer = (\Nat\times\Nat)/ \sim_\Integer$, where $(x,y) \sim_\Integer (u,v) :\Leftrightarrow  x+v = y+u$. Multiplication on $\Integer$ can be defined inductively (see below).
\item $\Rat = (\Integer\times\Integer)/\sim_\Rat$, where $(x,y) \sim_\Rat (u,v) :\Leftrightarrow xv = yu$.
\item We can extend the linear order $<$ of $\Nat$ to $\Integer$ and then to $\Rat$ in the usual way. Then we can introduce the \textbf{real numbers} as the set of \textbf{Dedekind cuts}:
\end{itemize}

\begin{equation}
\Real = \{ x \in \mathcal{P}(\Rat) \colon x \neq \emptyset \: \wedge \: x \neq \Rat \: \wedge \: \forall z \in x \forall y \in \Rat \: y < z \to y \in x \}.
\end{equation}

The final axiom of $\ZF$ is

\begin{quote}
(\textbf{Foundation}) $\qquad \forall a \;( a \neq \emptyset \to \exists x \in a \; \forall y \in x \, y \not \in a)$.
\end{quote}

Foundation rules out, for example, that a set can be an element of itself. More precisely, the axiom states that $\in$-relation is \textbf{well-founded} on any set.

We can also formalize the \textbf{Axiom of Choice}:

\begin{quote}
(\textbf{Choice}) $\qquad \forall a ( \forall x \in a \; x \neq \emptyset \;\;\; \to \;\;\; \exists f (\Op{Fun}(f) \:\wedge\: \Op{dom}(f) = a \:\wedge\: \forall x \in a \: f(x) \in x))$
\end{quote}

We denote the axiom system $\ZF + \AC$ as $\ZFC$ -- \textbf{Zermelo-Fraenkel with Choice}.

\subsection{Recursion and the Von-Neumann Hierarchy}

\subsubsection{Transfinite induction}

While the class $\Ord$ of all ordinals is not a set, it is still transitive and well-ordered by $\in$. Regarding the associated order $\leq$, every \textit{set} of ordinals $a$ has a \textbf{supremum} $\bigcup a = \bigcup_{\xi \in a} \xi$ and (if $a \ne \emptyset$) an \textbf{infimum}
$\bigcap a = \bigcap_{\xi \in a} \xi$, which is the \textit{smallest element} of $a$. Such a smallest element exists actually for every (non-empty) \textit{class} $A$ (since if $\xi \in A$, we only need to find the infimum of the \textit{set} of ordinals $\le \xi$.)
This allows us to prove properties about \textit{all} ordinals by \textbf{induction}.

\begin{proposition}[Induction for ordinals, I]\label{prop-induction-ord-i}For every property $\varphi$,
\begin{equation*}
\forall \alpha \; [ \forall \xi < \alpha \; \varphi(\xi) \to \varphi(\alpha)] \to \forall \alpha \, \varphi(\alpha).
\end{equation*}
\end{proposition}We have repeatedly used induction already for ordinals $< \omega_1$, the first uncountable ordinal.

To prove this principle simply observe that if $\forall \alpha \, \varphi(\alpha)$ failed there would have to be a \textit{smallest}  $\alpha$ with  $\neg \varphi(\alpha)$, contradicting the induction hypothesis.

Since every ordinal is either 0, a successor, or a limit ordinal, we have the following variant of induction.

\begin{proposition}[Induction for ordinals, II]\label{prop-induction-ord-ii}For every property $\varphi$, if

\begin{itemize}
\item \textbf{(i)} $\varphi(0)$,
\item \textbf{(ii)} $\forall \alpha (\varphi(\alpha) \to \varphi(\alpha+1))$, and
\item \textbf{(iii)} $(\forall \xi < \lambda \; \varphi(\xi)) \to \varphi(\lambda)\quad$ for all limit $\lambda$,
\end{itemize}

then $\quad \forall \alpha \;  \varphi(\alpha)$.

\end{proposition}(i) and (ii) coincide with the usual induction scheme for natural numbers. To cover \textit{all} ordinals we need to add (iii).

\subsubsection{Ordinal recursion}

The induction principle can be used to define functions by \textbf{recursion}. For example, \textbf{addition} on the natural numbers is given by

\begin{align*}
    x+0 \quad & =  x\\
    x+ (y+1) & =  (x + y)+1. 
\end{align*}

In the case of ordinals, we have to consider the limit case, too.

\begin{theorem}[Recursion on ordinals]\label{thm-ordinal-recursion}If $G :\Ord \times \V \longrightarrow  \V$ is a function and $a$ is a set, then there exists a unique function $F: \Ord \longrightarrow \V$  such that for all $\alpha \in \Ord$,

\begin{equation}
F(\alpha) = G(\alpha, F\Rest{\alpha})
\end{equation}

\end{theorem}\begin{proof}The uniqueness of the function $F$ follows by induction.

To show the existence of $F$, we define the following:

\begin{itemize}
\item Call $h$ \textit{tame} if
\end{itemize}
\begin{equation*}
\exists \alpha \, (h: \alpha \to \V  \wedge \forall \xi \in \alpha \; h(\xi) = G(\xi, h \Rest{\alpha}))
\end{equation*}

\begin{itemize}
\item Say $h$ is \textit{compatible} with $g$ if
\end{itemize}
\begin{equation*}
\forall x \in \Op{Dom}(h) \cap \Op{Dom}(g) \; h(x) = g(x)
\end{equation*}

It follows by induction that any two tame functions are compatible.

This lets us define the desired $F$ as

\begin{equation}
F := \bigcup \{h \colon h\, \text{ tame}\}
\end{equation}

Then $F$ is a function (otherwise there would be two incompatible tame functions), its domain is transitive, and satisfies the recursion condition (since it is the union of tame functions).

It remains to show that $F$ is defined on all of $\Ord$.
If $D = \Op{Dom}(F) \neq \Ord$, then we would have $D = \alpha$ for some ordinal $\alpha$. In particular $B$ is a set therefore $F = f$ is a set, for some tame $f$. This $f$ could be extended to a tame $h = f \cup \{(\alpha,G(\alpha,f \Rest{\alpha}))\}$, contradiction.

\end{proof}Note that we defined $F$ \textbf{explicitly} as a \textit{union} of all partial solution to the recursion equation.

As with induction, we have the following variant of the recursion principle.

\begin{proposition}[Recursion on ordinals, variant]\label{prop-ordinal-recursion-ii}If $G, H: \Ord \times \V \longrightarrow  \V$ are functions and $a$ is a set, then there exists a unique function $F: \Ord \longrightarrow \V$  such that
\begin{align*} 
    F(0) \quad & =  a\\ 
    F(\alpha+1) & =  G(\alpha,F(\alpha)) \\    
    F(\lambda) \quad & =  H(\lambda, \{F(\xi) \colon \xi<\lambda \}) \quad \text{for } \Op{Lim}(\lambda).
\end{align*}
\end{proposition}We can establish a similar principle for a well-ordering $<$ on a class $A$. In case of a proper class, though, we have to require that for every $a \in A$,  the class of all \textbf{predecessors} of $a$
\begin{equation*}
S(a,<): = \{x \in A\colon x < a \},
\end{equation*}
is a set (if $A$ is a set, this follows automatically by \textit{Separation}). If this is the case, the recursion principle yields a function $F: A \to \V$ such that
\begin{equation*}
F(a) = G(a,F\Rest{S(a,<)}).
\end{equation*}

\subsubsection{Recursion for well-founded relations}

More generally, we can define induction and recursion on \textbf{well-founded} relations. We already encountered those in a previous~chapter.

\begin{definition}\label{def-well-founded}A relation $R$ on a class $A$ is \textbf{well-founded} if it satisfies the \textbf{minimality condition}
\begin{equation*}
(\Op{Min}_R) \qquad  \emptyset \neq b \subset A \to \exists x \in b \; \forall y \in b \, (  \neg y R x)
\end{equation*}
and the \textbf{set condition}
\begin{equation*}
\forall x \in A \; S(a,R):= \{x \colon x R a\} \text{ is a set}
\end{equation*}
\end{definition}If $A$ is a set, the minimality condition is again automatically satisfied by \textit{Separation}.

The set condition allows for taking the \textbf{$R$-transitive closure} of a set $a \in A$: the smallest superset $\Op{TC}_R(a)$ of $a$ that is \textit{$R$-transitive}:

\begin{equation}
\forall x \in \Op{TC}_R(a)\; S(x,R) \subseteq  \Op{TC}_R(a)
\end{equation}

This is done by recursion over the natural numbers. The following is an important example.

\begin{example}[Transitive closure of a set]\label{exa-trans-closure}By the axiom of \textit{Foundation}, $\in$ is a well-founded relation on $\V$. (The set condition is satisfied since $S(a,\in)=a$.)

We can form the \textbf{transitive closure}, the smallest transitive superset, of a set $a$ as

\begin{align*}
    \Op{TC}(a):&= a \cup \bigcup a \cup \bigcup \bigcup a \ldots 
 = \bigcup_{n< \omega} U^n(a), \quad \text{ where} \\ 
 &U^0(a) = a, U^{n+1}(a) = \bigcup U^n(a).
\end{align*}

This is an example of definition by recursion along $\Nat$.

\end{example}We can use the existence of $\Op{TC}_R$ as a set to strengthen the minimality condition to \textit{subclasses}, similar to the case of the well-ordering of $\Ord$:

\begin{lemma}\label{lem-min-wf}For every non-empty class $B \subseteq A$, there exists $x \in B$ such that

\begin{equation}
\forall y \in B \;  \neg y R x
\end{equation}

\end{lemma}To prove this lemma, simply pick any $x \in B$, take its transitive $R$-closure, and intersect it with $B$:

\begin{equation}
C = \Op{TC}_R(x) \cap B.
\end{equation}

$C$ is a set, and by the minimality condition $(\Op{Min}_R)$ has an $R$-minimal element $a$. $a$ has to be minimal for $B$, too, since otherwise there exists $b \in B$ with $b R a$. Since $a \in \Op{TC}_R(x)$, $b \in \Op{TC}_R(x)$, and therefore $b \in C$, contradicting the minimality of $a$.

The lemma implies a corresponding \textbf{induction principle for well-founded relations}:

\begin{equation*}
(\Op{Ind}_R) \qquad  \forall x \in A [ \forall y ( yRx \, \to \varphi(y)) \to \varphi(x)] \to \forall x \in A \, \varphi(x)).
\end{equation*}

This in turn yields the following.

\begin{theorem}[Recursion principle for well-founded relations]\label{thm-wf-recursion}Let $R$ be a well-founded relation on a class $A$. The for every function $G : A \times \V \longrightarrow  \V$ exists a unique function $F: A \to \V$ such that

\begin{equation*}
F(a) = G(a,F \Rest{\{x \mid xRa\}})  \text{ for all } a \in A.
\end{equation*}
\end{theorem}\subsubsection{The Von-Neumann hierarchy}

Is there a way to systematically build the $\V$, the universe of all sets, ``from below''?

We start with the empty set:

\begin{equation}
V_0 = \emptyset
\end{equation}

Given $V_\alpha$, the \textit{Power Set} axiom requires the set of all subsets to exist, so we set

\begin{equation}
V_{\alpha+1} = \mathcal{P}(V_{\alpha}).
\end{equation}

Finally, at limit stages we simply collect all sets we have obtained so far:

\begin{equation}
V_ \lambda =  \bigcup_{\xi < \lambda} V_\xi \quad \text{for limit } \lambda
\end{equation}

What we really are doing here is to construct a function $V: \Ord \to \V$ by ordinal recursion. Think $V_\alpha = V(\alpha)$.

Remarkably, if we assume the axiom of \textit{Foundation}, we reach \textit{all} sets this way.

\begin{theorem}\label{thm-von-neumann}For every set $x$ there exists an ordinal $\alpha$ with $x \in V_\alpha$, that is,

\begin{equation}
\V = \bigcup_{\alpha \in \Ord} V_\alpha
\end{equation}

\end{theorem}\begin{proof}Let $C$ be the class of all sets not in any $V_\alpha$. Since $\in$ is well-founded, if $C$ is non-empty, it has a $\in$-minimal element $x$. This implies that for all $z \in x$, $z \in \bigcup_{\alpha \in \Ord} V_\alpha$. Define a function $h$ by mapping each $z\in x$ to the \textit{least} $\alpha$ so that $z \in V_\alpha$. Since $x$ is a set, $h[x]$ is a set of ordinals, by \textit{Replacement}. This set or ordinals has a supremum, say $\gamma$. Then $x \subseteq V_\gamma$ and therefore,

\begin{equation}
x \in \mathcal{P}(V_\gamma) = V_{\gamma+1}.
\end{equation}

Hence $C$ must be empty, and the theorem follows.

\end{proof}We can now split the question of ``how large'' $\V$ is into two sub-questions:

\begin{itemize}
\item How ``\textbf{long}'' is $\V$, that is, how many ordinals are there? Axioms for \textbf{large cardinals} attempt to extend this ``length'' as far as possible.
\item How ``\textbf{wide}'' is $\V$, that is, how large is the power set of a set? A rather ``slim'' universe is given by the \textbf{constructible sets}, which we will encounter soon.
\end{itemize}

\subsection{Models of Set Theory}

You may have noticed that, when introducing the axioms of $\ZFC$, we never \textit{really} answered the question ``\textit{What is a set?}''. Instead, we developed a formal theory of axioms for a binary relation that somehow describe ``\textit{how sets work}'', that is, how we can obtain sets from given ones using well-known operations like power set and union.

We have then seen how we can develop a lot of standard mathematical \textit{objects} (like $\Nat$, $\Real$) and \textit{techniques} (like induction and definition by recursion) \textbf{inside} this formal system. In fact, most of mathematics can be developed formally inside this system. Almost all proofs you find in any standard math book are proofs that can be formalized in $\ZFC$. It is very tedious to do this for us humans, but there is little doubt it can be done, and in fact, looking at the recent work on \textbf{proof assistants} (like Coq or Lean), many parts of mathematics have been formalized (albeit not directly in $\ZFC$).

This expressiveness gives $\ZFC$ its foundational importance, but it is also the cause for much confusion for someone who first studies set theory.

From a pedagogical point of view, in what follows it is helpful to assume a ``\textbf{Platonist}'' perspective of mathematics, and set theory in particular, namely that \textit{sets and the relations between exist independently (and outside) of the $\ZFC$ axioms}. The set of real numbers exists, and our development of $\Real$ inside $\ZFC$ is just a formal way to describe them. From this perspective, the axioms of $\ZF$ ($\AC$ is a little different) are just obvious truths about sets, just like the \textbf{Peano axioms} are obvious truths about natural numbers.

Among other things, this perspective allows us to treat $\ZF$ just like any other mathematical theory, like \textit{group theory} or the theory of \textit{algebraically closed fields}.
In particular, we can think about \textbf{models of set theory} the way we would think about models of group theory, in the sense of model theory.

A \textbf{model} would simply be a set $M$ together with a binary relation $E$ on $S$ such that

\begin{equation}
(M,E) \models \ZF,
\end{equation}

that is, all axioms hold when interpreted in $(M,E)$. Note that we use ``\textit{set}'' in this context not in the formal sense, but in the ``meta''-sense (the Platonist world of sets).

Working in the meta-theory (``\textit{that what is mathematically true}''), we know by Gödel's completeness theorem that

\begin{quote}
if $\ZF$ is consistent, then it has a model.
\end{quote}

This model should be seen as a \textbf{set-theoretic universe}: Its elements can be seen as sets, and the interpretation  $E$ of the $\in$-symbol will tell us how these sets are connected via the element-relation.

Note that $E$ does not have to be the \textit{actual} element relation on a set (of sets), but just some binary relation so that the axioms are satisfied.

In the meta-world, there are, of course, sets other than $M$, but that does not matter here, since al we are interested in is giving \textit{some} universe in which our axioms hold. (Timothy Chow has suggested that set theory should rather be called "\textit{universe theory}. He is right in the sense that what axiomatic set theory does is to define such \textit{universes of sets}, rather than what a set is.)

In the meta-theory, we can then follow the usual techniques to show provability or non-provability results.

If we want to prove that $\CH$ is consistent with $\ZF$ (assuming $\ZF$ is consistent), we need to find a model in which both hold.

One difficulty in working with models of set theory is that they can look very different depending on whether you look at a model ``\textbf{from the inside}'' or ``\textbf{from the outside}''.

To illustrate this, assume $\ZF$ is consistent. Then, by the \textbf{\href{https://en.wikipedia.org/wiki/L\%C3\%B6wenheim\%E2\%80\%93Skolem\_theorem}{Löwenheim-Skolem theorem}}, there exists a \textbf{countable model} for $\ZF$.
Yet it is a theorem of $\ZF$ that \textit{there exists an uncountable set}. This is often referred to as \textbf{\href{https://en.wikipedia.org/wiki/Skolem's\_paradox}{Skolem's paradox}}, although it is not really an antinomy.

If we break this down a bit, we see that the apparent paradox is really just a matter of perspective (\textit{inside} or \textit{outside}). Assume $(M,E)$ is a countable model of $\ZF$. Then there exists $x \in M$ such that there is no injection from $x$ to the natural numbers. Since $M$ is countable, $x$ can have at most countably many elements. So why is this \textit{not} a contradiction? We should really read the statement above as

\begin{quote}
there is no injection \textbf{in $M$} from $x$ to \textbf{$M$'s version} of the natural numbers.
\end{quote}

In other words, even though $x$ is countable \textit{from the outside}, $x$ appears uncountable \textit{inside $M$} since a mapping witnessing its countability does not exist in $M$.

This is a first warning sign that models of $\ZF$ can behave in very unexpected ways. For another example, recall the axiom of \textit{Foundation} asserts that the $\in$-relation is well-founded. But again, this means only ``from the inside''.

\begin{proposition}\label{prop-illfounded-zf-model}If $\ZF$ is consistent, than there exists a model $(M,E)$ of $\ZF$ such that $(M,E)$ is ill-founded.

\end{proposition}\begin{proof}Introduce new constant symbols $c_n$ $(n \in \Nat)$ and add the formulas $\varphi_n \equiv c_{n+1} \in c_n$ to the axioms of $\ZF$. It is not hard to show, using the \href{https://en.wikipedia.org/wiki/Compactness\_theorem}{compactness theorem}, that $\ZF + \bigcup_n \varphi_n$ has a model $(M^*, E^*)$, for which the set $\{c_n \colon n \in \Nat\}$ is ill-founded.

\end{proof}Since, as mentioned above, the model $(M^*,E^*)$ satisfies \textit{Foundation}, the set $\{c_n \colon n \in \Nat\}$ is actually \textit{not in} the model (and neither can be any other set with an infinite descending $\in$-chain).

\subsubsection{Mostowski collapse}

If we restrict ourselves to models on which the $E$-relation is \textit{actually} well-founded (i.e. from the outside), then interestingly these models look in way ``\textit{natural}'': They can be assumed to be the $\in$-relation on a set. Such models are also called \textbf{standard}.

Given a set theoretic structure $(M,E)$ (not necessarily a model of $\ZF$), for each $x \in M$ let
\begin{equation*}
\Op{ext}_E(x) = \{ y \in X \colon y\, E \, x \}
\end{equation*}

If $E$ behaves ``set-like'', then it will respect the axiom of \textit{Extensionality}, i.e. two sets are identical if and only if they have the same elements. Therefore we say that $E$ is \textbf{extensional} if
\begin{equation*}
x,z \in X, \; x\neq z \quad \text{ implies } \quad \Op{ext}_E(x) \neq \Op{ext}_E(z).
\end{equation*}

Furthermore, as stated above, we want to exclude infinite descending $E$-chains. We say that $E$ is \textbf{well-founded} if

\begin{quote}
every non-empty set $Y \subseteq X$ has an $E$-minimal element.
\end{quote}

\begin{theorem}[Mostowski collapse]\label{thm-mostowski-collapse}If $E$ is an extensional and well-founded relation on a set $X$, then there exists a transitive set $S$ and a bijection $\pi: X \to S$ such that
\begin{equation*}
x \, E \, y \iff \pi(x) \in \pi(y) \quad \text{ for all $ x,y \in X$}.
\end{equation*}
Moreover, $S$ and $\pi$ are unique.

\end{theorem}\begin{proof}We construct $\pi$ and $S = \Op{im}(\pi)$ by recursion on $E$, which is possible since it is well-founded.

For each $x \in X$, let
\begin{equation*}
\pi(x) = \{\pi(y) \colon y \, E \, x \},
\end{equation*}
and set $S = \Op{im}(\pi)$.

The injectivity of $\pi$ follows from the extensionality of $\pi$ by induction along $E$: 	
Suppose we have shown
\begin{equation*}
\forall z \; (z E x \to \forall y  \in X (\pi(z) = \pi(y) \to z = y)).
\end{equation*}
and we have to show that it holds for $x$. Assume $\pi(x) = \pi(y)$ for some $y \in X$. Then
\begin{align*}
cEx &\Rightarrow& \pi(c) \in \pi(x) = \pi(y) &\\
&\Rightarrow& \pi(c) = \pi(z) & \qquad \text{ for some } zEy\\
&\Rightarrow& c=z & \qquad  (\text{by ind. hyp., since } cEx)\\
&\Rightarrow& cEy &.
\end{align*}
Similarly, we get $cEy \Rightarrow cEx$, hence $x=y$ as desired due to the extensionality of $E$. Finally we have
\begin{align*}
\pi(x) \in \pi(y) & \Rightarrow & \pi(x) = \pi(c) & \qquad \text{ for some } cEy \\
&\Rightarrow& x = c & \qquad \text{ (since $\pi$ is injective)}\\
&\Rightarrow& xEy &
\end{align*}
Thus $\pi$ is an isomorphism.

To see the uniqueness of $\pi$ and $S$, assume $\rho$, $T$ are such that the statement of the theorem is satisfied. Then $\pi \circ \rho^{ -1}$ is an isomorphism between $(T, \in)$ and $(S,\in)$. Now apply the following lemma.

\end{proof}\begin{lemma}\label{lem-mostowski-unique}Suppose $X,Y$ are sets, and $\theta$ is an isomorphism between $(X,\in)$ and $(Y,\in)$. Then $X=Y$ and $\theta(x) = x$ for all $x \in X$.

\end{lemma}\begin{proof}By induction on the well-founded relation $\in$. Assume that $\theta(z)=z$ for all $z \in x$ and let $y = \theta(x)$.

We have $x \subseteq y$ because if $z \in x$, then $z = \theta(z) \in \theta(x) = y$.

We also have $y \subseteq x$: Let $t \in y$. Since $y \in Y$, there is $z \in X$ with $\theta(z) = t$. Since $\theta(z) \in y$ and $y = \theta(x)$, we have $z \in x$, and thus $t = \theta(z) = z \in x$.

Hence $x = y$, and this also implies $\theta(x) = x$.

\end{proof}

\subsection{Large Cardinals}

\subsubsection{Inaccessible cardinals}

The cardinality of $V_\alpha$ grows rather fast relative to $\alpha$. For example,

\begin{equation}
|V_{\omega+\alpha}| = \beth_\alpha
\end{equation}

where the \textbf{beth function} $\beth_\alpha$ is defined as
\begin{gather*}
    \beth_0 = \aleph_0 \\
    \beth_{\alpha+1} = 2^{\beth_\alpha} \\
    \beth_{\lambda} = \sup \{ \beth_\alpha \colon \alpha < \lambda\} \quad \text{ $\lambda$ limit}
\end{gather*}

This presents difficulties for the axiom of \textit{Replacement} to hold in a $V_\alpha$, since we could define a function on a set of sufficiently high cardinality that maps to sets in $V_\alpha$ whose ranks are cofinal in $\alpha$ (and the image would not be an element of $V_\alpha$).

The existence of \textbf{inaccessible cardinals} ensures that the von-Neumann hierarchy is ``long enough'' for $\alpha$ to eventually ``catch up'' with the cardinality of $V_\alpha$.

Recall the enumeration of all cardinals by means of the \textbf{$\aleph$-sequence}:

\begin{equation*}
\aleph_0 = \omega, \quad \aleph_{\alpha+1} = \aleph_\alpha^+, \quad  \aleph_\lambda = \sup \{ \aleph_\xi \colon \xi<\lambda\} \; \text{ for limit } \lambda.
\end{equation*}

Here $\kappa^+$ is the least cardinal $> \kappa$. Some cardinals are limits of short sequences of cardinals -- for example,
\begin{equation*}
\aleph_\omega = \lim_n \aleph_n
\end{equation*}
is uncountable, but a limit of a countable sequence of smaller cardinals. Generally, cardinals who are a limit of a sequence of cardinals of length smaller than their cardinality are called \textbf{singular}. Non-singular cardinals are called \textbf{regular}:
\begin{equation*}
\Op{reg}(\kappa): \iff \forall \alpha < \kappa \;  \forall f \;( f: \alpha \to \kappa \;  \to \; \sup_{\xi < \alpha} f(\xi) < \kappa).
\end{equation*}

In other words, a regular cardinal $\kappa$ cannot be reached  by less then $\kappa$-many steps.  The first example of a regular cardinal is $\aleph_0$.

\begin{framed}
\textbf{Exercise}\\
Show that all \textbf{successor cardinals}, i.e. cardinals of the form $\aleph_{\alpha+1}$ are regular. (Use the Axiom of Choice.)
\end{framed}

On the other hand,
\begin{equation*}
\aleph_\omega,  \aleph_{\omega+\omega}, \aleph_{\aleph_{\omega}}, \aleph_{\aleph_{\aleph_\omega}}, \ldots
\end{equation*}
are singular and this suggests the question:

\begin{quote}
Are there regular cardinals of the form $\aleph_\lambda$ with $\lambda$ limit?
\end{quote}

This is captured by the notion of \textbf{inaccessibility}.

\begin{definition}[Hausdorff 1908, Tarski, Zermelo 1930]\label{def-inaccessible}An uncountable cardinal $\kappa > \omega$ is

\begin{itemize}
\item \textbf{weakly inaccessible} if
\end{itemize}
\begin{align*}
    \Op{reg}(\kappa) \; &  \wedge \;  \exists \lambda (\Op{lim}(\lambda) \;  \wedge\;  \kappa = \aleph_\lambda)\\
                        (& \Leftrightarrow \Op{reg}(\kappa) \;  \wedge \;  \forall \alpha < \kappa \; \, \alpha^+ < \kappa)
\end{align*}

\begin{itemize}
\item \textbf{(strongly) inaccessible} if
\end{itemize}
\begin{equation*}
\Op{reg}(\kappa) \;  \wedge \;  \forall \alpha < \kappa \;\; 2^{\alpha} < \kappa
\end{equation*}
\end{definition}Under the \textbf{Generalized Continuum Hypothesis},

\begin{quote}
($\mathsf{GCH}) \quad \forall \alpha \;\;  2^{\aleph_\alpha} = \aleph_{\alpha}^+$
\end{quote}

weakly and strongly inaccessible cardinals coincide.

If $\kappa > \omega$ is inaccessible, then $\kappa = \aleph_\kappa$. Moreover, we have

\begin{proposition}\label{prop-cardinality-vkappa}If $\kappa$ is strongly inaccessible, $|V_\kappa| = \kappa$.

\end{proposition}\begin{proof}It suffices to show that $|V_\alpha| < \kappa$ for all $\alpha < \kappa$. This follows by a straightforward induction, using the fact that $\kappa$ is strongly inaccessible.

\end{proof}This in turn implies we can bound the cardinality of elements of $V_\kappa$.

\begin{proposition}\label{prop-inaccessible-cardinality}Suppose $\kappa$ is strongly inaccessible and $x\subset V_\kappa$. Then

\begin{equation}
x \in V_\kappa \; \Leftrightarrow \; |x| < \kappa.
\end{equation}

\end{proposition}\begin{proof}($\Rightarrow$) $x \in V_\kappa$ implies $|x| < |V_\kappa|$. Apply Proposition~\ref{prop-cardinality-Vkappa}.

($\Leftarrow$) Since $x \subseteq V_\kappa$, each $y \in x$ has rank $< \kappa$. Since $|x| < \kappa$, by regularity of $\kappa$,

\begin{equation}
\Op{rank}(x) = \sup\{\Op{rank}(y)+1 \colon y \in x\} < \kappa
\end{equation}

which implies $x \in V_\kappa$.

\end{proof}We have already seen that for limit $\alpha > \omega$, $V_\alpha$ is a model of all $\ZFC$ axioms except \textit{Replacement}.

\begin{theorem}\label{thm-inaccessible-zfc}If $\kappa$ is strongly inaccessible, then $V_\kappa \models \ZFC$.

\end{theorem}\begin{proof}We verify that $V_\kappa$ satisfies the axiom of \textit{Replacement}.
Suppose $x \in V_\kappa$ and $f:x \to V_\kappa$ is a function. Then $f[x] \subseteq V_\kappa$, and by Proposition~\ref{prop-inaccessible-cardinality}, $|f[x]| \leq |x| < \kappa$. Applying the other direction of Proposition~\ref{prop-inaccessible-cardinality} to $f[x]$, we obtain $f[x] \in V_\kappa$, as desired.

\end{proof}Suppose an inaccessible cardinal exists, and let $\kappa$ be the least inaccessible.\newline
It is not hard to verify that

\begin{equation}
V_\kappa \models \ZFC + \text{''there does not exist an inaccessible cardinal''}.
\end{equation}

(You verify that being a inaccessible cardinal is absolute for $V_\kappa$.) Therefore, the existence of an inaccessible cardinal is not provable from $\ZFC$. This fact also follows from Gödel's second incompleteness theorem.

\subsubsection{Measurability}

We have seen that (assuming the Axiom of Choice) there subsets of $\Real$ that are not Lebesgue measurable. Inspecting the proof, we see that we only use the following properties of Lebesgue measure:

\begin{itemize}
\item $\sigma$-additivity,
\item translation invariance ($\lambda(A) = \lambda(A+r)$),
\item $\lambda(A) > 0$ for some $A$.
\end{itemize}

For spaces without an additive structures, instead of translation invariance, we can consider a \textbf{non-triviality condition}:

\begin{equation}
m(\{x\})=0 \quad \text{ for all $x$}
\end{equation}

The \textbf{generalized measure problem} asks whether there exists a set $M$ together with a measure function
\begin{gather*}
m: \mathcal{P}(M) \to [0,\infty),
\end{gather*}
so that the following conditions are met:

\begin{itemize}
\item (\textbf{M1}) $\quad$ $m(M) =1$
\item (\textbf{M2}) $\quad$ $\forall x \in M \; m(\{x\})=0$
\item (\textbf{M3}) $\quad$ if $(A_i)_{i < \omega}$ is a countable sequence of disjoint sets $\subseteq M$, then
\end{itemize}
\begin{equation*}
m\left(\bigcup_{i<\omega} A_i\right ) =  \sum_{i<\omega} m(A_i)
\end{equation*}

The structure of the set $M$ does not play any role here, so we can replace it by a cardinal $\kappa$ outright. One can also consider strengthening $\sigma$-additivity to \textbf{$\kappa$-additivity}:

\begin{quote}
If $\gamma < \kappa$ and  $(A_\xi)_{\xi< \lambda}$ is a sequence of disjoint subsets of $\kappa$, then
\end{quote}
\begin{equation*}
m(\bigcup_{\xi<\gamma} A_\xi) =  \sum_{\xi<\gamma} m(A_\xi).
\end{equation*}

A transfinite sum $\sum_{\xi<\gamma}$ is given as the supremum of all sums over finite subsequences:

\begin{equation}
\sum_{\xi<\gamma} m(A_\xi) = \sup \left \{ \sum_{\xi \in F} m(A_\xi) \colon F \subseteq \gamma \text{ finite}\right \}.
\end{equation}

Hence, $\omega_1$-additive is the same as $\sigma$-additive.

\begin{theorem}[Banach]\label{thm-kappa-additivity}If $\kappa$ is the least cardinal for which a measure satisfying (M1)-(M3) exists, then any such measure on $\kappa$ is already $\kappa$-additive.

\end{theorem}\begin{proof}Suppose $m$ is a measure on $\kappa$ that is not $\kappa$-additive.
Then, for some $\gamma < \kappa$, there exists a sequence $(A_\xi)_{ \xi< \gamma}$ of disjoint subsets of $\kappa$ so that

\begin{equation}
m(\bigcup_{\xi<\gamma} A_\xi) \ne  \sum_{\xi<\gamma} m(A_\xi).
\end{equation}

Since a measure is always $\sigma$-additive, $\gamma > \omega$ has to hold, and there can be at most countably many $A_\xi$ with $m(A_\xi)>0$.

We can drop those $A_\xi$, and by the $\sigma$-additivity of $m$ for the remaining $\xi$ it has to hold that $m(A_\xi)=0$ while $m \left(\bigcup_{\xi<\gamma} A_\xi \right) = r >0$.

By putting
\begin{equation*}
\overline{m}(X) = \frac{m(\bigcup_{\xi \in X} A_\xi)}{r}
\end{equation*}
we obtain a measure on $\gamma < \kappa$, contradicting the minimality of $\kappa$.

\end{proof}\paragraph{Measurable cardinals}

If $m$ is a measure on $\kappa$, the \textbf{associated ideal}

\begin{equation}
\mathcal{I}_m = \{x\subseteq \kappa \colon m(x) = 0 \}
\end{equation}

is a $\sigma$-ideal, or, complementing the notion of $\omega_1$-additivity, a \textbf{$\omega_1$-complete ideal}.

\begin{framed}
\textbf{Exercise}\\
Show that $\mathcal{I}_m$ is not principal.
\end{framed}

The corresponding filter

\begin{equation}
\mathcal{F}_m = \{x\subseteq \kappa \colon m(x) = 1\}
\end{equation}

is then $\omega_1$-complete, too.

A measure $m$ is \textbf{two-valued} if it only assumes the values 0 and 1. In this case the corresponding filter $\mathcal{F}_m$ is an \textbf{ultrafilter} (and $\mathcal{I}_m$ is a \textbf{prime ideal}).

Conversely, if $U$ is $\omega_1$-complete, non-principal ultrafilter on $\kappa$, we can define a two-valued measure $m: \mathcal{P}(\kappa) \to \{0,1\}$ on $\kappa$ by letting

\begin{equation}
m(x) = 
  \begin{cases}
   1  & \text{if } x \in U, \\
   0   & \text{otherwise}.
\end{cases}
\end{equation}

\begin{definition}\label{def-measurable-cardinal}Let $\kappa$ be an uncountable cardinal.

\begin{itemize}
\item $\kappa$ is \textbf{real-valued measurable} if there exists a $\kappa$-additive measure on $\kappa$.


\item $\kappa$ is \textbf{measurable} if there exists a $\kappa$-additive, two-valued measure on $\kappa$, or, equivalently, if there exists a $\kappa$-complete, non-principal ultrafilter on $\kappa$.
\end{itemize}

\end{definition}In the following, we will see that measurability implies inaccessibility.

\begin{lemma}\label{lem-cardinality-kappa-ultrafilter}If $U$ is a $\kappa$-complete, non-principal ultrafilter on $\kappa$, then every $X \in U$ has cardinality $\kappa$.

\end{lemma}\begin{proof}Since $U$ is non-principal, no \textit{singleton} set $\{x\}$ can be in $U$ (for this would imply $\kappa\setminus \{x\} \notin U$ and therefore no subset of it would be in $U$ either, contradicting the non-principality of $U$).

If $X \in U$ and $|X| < \kappa$, then $X$ is the union of $< \kappa$ many singletons. Since $\neg U$ is a $\kappa$-complete prime ideal, this implies $X \in \neg U$, contradiction.

\end{proof}\begin{proposition}\label{prop-measurable-regular}If $\kappa$ is measurable, then it is regular.

\end{proposition}\begin{proof}If $\kappa$ were singular, it would be the union of $<\kappa$-many sets of cardinality $<\kappa$. Applying Lemma~\ref{lem-cardinality-kappa-ultrafilter} leads to a contradiction.

\end{proof}\begin{theorem}\label{thm-measurable-inaccessible}A measurable cardinal is (strongly) inaccessible.

\end{theorem}\begin{proof}By Proposition~\ref{prop-measurable-regular}, any measurable cardinal is regular. Assume for a contradiction there exists $\gamma < \kappa$ with $2^\gamma > \kappa$. As $2^\gamma > \kappa$, there exists a set $S$ of functions $f: \gamma \to \{0,1\}$ with $|S| = \kappa$. Let $U$ be a $\kappa$-complete, non-principal ultrafilter on $S$.

For $\alpha < \gamma, i \in \{0,1\}$, let

\begin{equation}
X_{\alpha,i} = \{ f \in S \colon f(\alpha) = i\}
\end{equation}

and let $g(\alpha) = i$ if and only if $X_{\alpha,i} \in U$. Since $U$ is an ultrafilter, $g$ is well-defined on $\gamma$.

Since $\gamma < \kappa$ and $U$ is $\kappa$-complete,

\begin{equation}
X = \bigcap_{\alpha < \gamma} X_{\alpha, g(\alpha)}
\end{equation}

is in $U$. But $|X| \leq 1$, since the only function possibly in $X$ is $g$. This contradicts Lemma~\ref{lem-cardinality-kappa-ultrafilter}.

\end{proof}\begin{framed}
\textbf{Exercise}\\
Show that every real-valued measurable cardinal is weakly inaccessible.
\end{framed}

\begin{proposition}\label{prop-measurable-vs-real-valued}If $\kappa$ is real-valued measurable, then $\kappa$ is measurable or $\kappa \le 2^{\aleph_0}$.

\end{proposition}Thus, if $\kappa$ is real-valued measurable but not measurable, then the continuum $2^{\aleph_0}$ has to be very large.

\subsubsection{Partition properties}

Another concept of largeness is related to the existence of large \textbf{homogeneous sets} for partitions.

For given set $S$ and $n \in \Nat$, let
\begin{equation*}
[S]^n := \{ X \subseteq S \colon \: |X| = n \}
\end{equation*}
be the set of all $n$-element subsets of $S$. For cardinals $\kappa, \lambda$, we define
\begin{equation*}
\kappa \to (\lambda)^n_k
\end{equation*}
to mean that any partition $F: [S]^n \to \{1, \dots, k\}$ mit $|S| = \kappa$ has an \textbf{$F$-homogeneous subset}  of cardinality $\lambda$, that is, a set $H$, $|H| = \lambda$, such that
\begin{equation*}
F|_{[H]^n} \equiv \text{ constant}.
\end{equation*}

\textbf{Ramsey's theorem} (1929/39) says that for any $n,k \in \Nat$,
\begin{equation*}
\aleph_0 \to (\aleph_0)^n_k.
\end{equation*}

Do there exist uncountable cardinals with similar properties?

A cardinal $\kappa$ is \textbf{weakly compact} if it is uncountable and $\kappa \to (\kappa)^2_2$ holds.

\begin{framed}
\textbf{Exercise}\\
Show that for any cardinal $\kappa$, $2^\kappa \nrightarrow (\kappa^+)^2_2$, and use this to infer that any weakly compact cardinal is inaccessible.

(Thus the existence of weakly compact cardinals cannot be established in $\ZFC$.)
\end{framed}

Measurable cardinals have even stronger homogeneity properties. Let $[S]^{<\omega}$ be the set of all finite subsets of $S$. If $F: [S]^{<\omega} \to I$ is a partition, then $H \subseteq S$ is \textbf{$F$-homogenenous} if
\begin{equation*}
F|_{[H]^n} \equiv \text{ constant}
\end{equation*}
for all $n \in \Nat$.

\begin{theorem}[Rowbottom]\label{thm-measurable-ramsey}Let $\kappa$ be a measurable cardinal and let $F: [\kappa]^{<\omega} \to \lambda$ a partition of $[\kappa]^{<\omega}$ into $\lambda < \kappa$ pieces. Then there exists an $F$-homogeneous set $H \subseteq \kappa$ with $|H| = \kappa$.

\end{theorem}In general, any cardinal that satisfies the statement of the theorem is called \textbf{Ramsey}.

To prove Theorem~\ref{thm-measurable-Ramsey}, we introduce \textbf{normal ultrafilters}.

\begin{definition}\label{def-normal-filter}Given a sequence of sets $(A_\xi)_{\xi < \gamma}$, the \textbf{diagonal intersection} is given as

\begin{equation}
\Delta_{\xi < \gamma} A_\xi = \{ \alpha < \gamma \colon  \alpha \in \bigcap_{\xi < \alpha} A_\xi \}.
\end{equation}

A filter $F$ on a cardinal $\kappa$ is \textbf{normal} if for any $\kappa$-sequence $(A_\xi)_{\xi < \kappa}$, $A_\xi \in F$, the diagonal intersection $\Delta_{\xi < \kappa} A_\xi$ is in $F$.

\end{definition}Let us assume as a convention that a filter on a cardinal $\kappa$ always contains the end-segments $\{\xi \colon \alpha \leq \xi < \kappa\}$.

\begin{framed}
\textbf{Exercise}\\
Show that a normal filter on $\kappa$ is $\kappa$-complete.
\end{framed}

\begin{framed}
\textbf{Exercise}\\
Show that if there is a normal filter over $\kappa$, then $\kappa$ is uncountable and regular.
\end{framed}

\begin{framed}
\textbf{Exercise}\\
Show that if $\kappa$ is measurable, then there is a normal ultrafilter on $\kappa$.
\end{framed}

\begin{proof}(Proof of Theorem~\ref{thm-measurable-Ramsey})

Let $U$ be a normal filter over $\kappa$.
We show that for every $n$, for any $g: [\kappa]^n \to \gamma$ with $\gamma < \kappa$, there is a set $H_n \in U$ such that $g_n \Rest{[H_n]^n} \equiv \text{const}$. The intersection of the $H_n$ is again in $U$ and satisfies the statement of the the theorem.

We proceed by induction. The case $n=1$ follows from the $\kappa$-completeness of $U$. Now assume $g:[\kappa]^{n+1} \to \gamma$, with $\gamma < \kappa$.

For each $S \in [\kappa]^n$, define $g_s : \kappa \to \gamma$ by

\begin{equation}
g_S(\alpha) = \begin{cases}
        g(S \cup \{\alpha\}) & \text{ if } \max S < \alpha \\
        0 & \text{otherwise}
    \end{cases}
\end{equation}

By $\kappa$-completeness of $U$, $g_S$ is constant on a set $Y_S \in U$, say

\begin{equation}
g_S\Rest{Y_S} \equiv \delta_S < \gamma.
\end{equation}

We now define a function $h: [\kappa]^n \to \gamma$
by letting

\begin{equation}
h(S) = \delta_S.
\end{equation}

By induction hypothesis, $h$ is constant on a set $Z \subseteq \kappa$
in $U$ (and hence of size $\kappa$), say $h\Rest{[Z]^n} \equiv \delta < \kappa$.

For each $\alpha < \kappa$, let

\begin{equation}
Y_\alpha = \bigcap \{Y_S \colon \max S \leq \alpha\}
\end{equation}

By $\kappa$-completeness, $Y_\alpha \in U$, and by normality

\begin{equation}
H = Z \cap \Delta_{\alpha < \kappa} Y_\alpha \in U
\end{equation}

By Lemma~\ref{lem-cardinality-kappa-ultrafilter}, $H$ has cardinality $\kappa$.

We claim that $g$ is constant on $[H]^{n+1}$: Let $T \in [H]^{n+1}$. Write $T$ as $S \cup \{\alpha\}$ with $\max S < \alpha$. Then

\begin{align*}
    \alpha \in H & \Rightarrow  & \alpha \in \Delta_{\gamma < \kappa} Y_\gamma \\
                 & \Rightarrow  & \alpha \in \bigcap_{\beta < \alpha} Y_\beta \\
                 & \Rightarrow  & \alpha \in Y_{\max S} \\
                 & \Rightarrow  & \alpha \in Y_S \\
                 & \Rightarrow  & g_S(\alpha) = \delta_S
\end{align*}

On the other hand, $S \subseteq H$ implies $S \subseteq Z$ and hence by definition of $Z$, $h(S) = \delta_S = \delta$, so $g(T) = g_S(\alpha) = \delta_S = \delta$.

\end{proof}

\section{Constructibility}

\include{descriptive_set_theory-constructible}

\include{descriptive_set_theory-v-l}

\subsection{Constructible Reals}

In this lecture we transfer the results about $L$ to the projective hierarchy. The main idea is to relate sets of reals that are defined by set theoretic formulas to sets defined in second order arithmetic.

\subsubsection{The set of constructible reals}

What is the complexity of the set $\Baire \cap L$? In particular, is it in the projective hierarchy?
The set of all constructible reals is defined by a $\Sigma_1$ formula over set theory:

\begin{equation}
\varphi(x_0)	\; \equiv \; \exists y \; [y \text{ is an ordinal }  \; \wedge \; x_0 \in L_y \; \wedge \; x_0 \text{ is a set of natural numbers }  ].
\end{equation}

We would like to replace this formula by an ``equivalent'' one in the language of second order arithmetic. In particular, we would like to replace the quantifier $\exists y$ by a quantifier over the reals.

The key for doing this is Lemma~\ref{lemma-L-GCH}: every constructible real shows up at a countable stage of $L$. Hence if $\alpha \in L \cap \Baire$, there exists a countable $\xi$ such that $x \in L_\xi$. Since $|\xi| = |L_\xi|$, $L_\xi$ is countable, too. Hence we can hope to replace $L_\xi$ by something like ``\textit{there exists a real that codes a model that looks like $L_\xi$}''.

The \textbf{condensation lemma} (Lemma~\ref{lem-condensation}) allows us to do this.
Let $\varphi_{\VL}$ be the conjunction of the axioms in $T$ and the axiom $\VL$.

Recall that any real $\alpha \in \Baire$ codes a set theoretic structure
\begin{equation*}
(\omega, E_\alpha) \qquad \text{ where } E_\alpha = \{ \Tup{m,n} \colon \alpha(\Tup{m,n}) =  0 \}.
\end{equation*}

Unfortunately, the condensation lemma only holds for \textbf{transitive sets} (and $(\omega, E_\alpha)$ may look very different from a transitive set model), so simply requiring $(\omega,E_\beta) \models \varphi_{\VL}$ is not enough. But we know from Theorem~\ref{thm-Mostowski-collapse} (\textbf{Mostowski collapse}) that if $E_\beta$ is well-founded and extensional, we can map it isomorphically to a transitive set $S$ on which we interpret $E_\beta$ as $\in$. By the condensation lemma, this $S$ must then be an $L_\xi$.

So, for reals, we can formulate membership in $L$ now as follows:

\begin{align*} \tag{$*$}
	\alpha \in L \cap \Baire \iff \exists \beta \exists m \: & [E_\beta \text{ is  extensional and well-founded} \\ & \quad \: \wedge \: (\omega,E_\beta) \models \varphi_{\VL} \: \wedge \: \pi_\beta(m) = \alpha ],
\end{align*}
where $\pi_\beta$ is the Isomorphism of the Mostowski collapse of $E_\beta$.

It remains to show that the notions occurring inside the square brackets are definable in second order arithmetic.

\begin{proposition}\label{prop-satisfaction-arithmetic}\begin{itemize}
\item (\textbf{a}) $\quad$  For any $n \in \Nat$, the following set is $\Sigma^0_n$:
\end{itemize}
\begin{equation*}
\{(m,\sigma,\gamma) \in \Nat\times \Nstr\times \Baire \colon m = \GN{\varphi} \: \wedge \: \varphi \text{ is $\Sigma_n$} \: \wedge \: (\omega,E_\gamma) \models \varphi[\sigma] \}
\end{equation*}

\begin{itemize}
\item (\textbf{b}) $\quad$  If $\alpha \in \Baire$ and $E_\alpha$ is well-founded and extensional, then the following set is arithmetic in $\alpha$:
\end{itemize}
\begin{equation*}
\{ (m,\gamma) \in \Nat\times \Baire \colon \pi_\alpha(m) = \gamma\}
\end{equation*}
\end{proposition}\begin{proof}(a) can be established similar to showing that $\Op{Sat}$-predicate of Theorem~\ref{thm-sat-predicate} is $\Delta_1$-definable. One does this first for $\Sigma_1$ formulas and then uses induction. Using Gödelization, one carefully defines all syntactical notions using arithmetic formulas. Then, one uses the recursive definition of truth to establish the definability of the satisfaction relation.

Since we work with relations over $\Nat$ now instead of arbitrary sets, it is not that easy anymore to keep quantifiers bounded. But since we are only interested in the complexity of $\models$ for $\Sigma_n$-formulas, this helps us bound the overall complexity at $\Sigma^0_n$

(c) By analyzing the recursive definition and using the definition of $\Nat$ in $\ZF$, one first shows that the set

\begin{equation}
\{ (m,p) \in \Nat\times \Nat  \colon \pi_\alpha(m) = p \}
\end{equation}

is arithmetic in $\alpha$.

Let $\psi(v_0, v_1, v_2)$ be the formula $\Tup{v_0, v_1} \in v_2$. Then

\begin{align*}
    & \pi_\alpha(m) = \gamma \iff \\ 
    & \qquad \forall p, q \: \left (\gamma(p) = q \: \leftrightarrow \: \exists i,j \: (\pi_\alpha(i) = p \wedge \pi_\alpha(j)=q \wedge (\omega,E_\alpha) \models \psi[i,j,m]) \right )
\end{align*}

Now apply the previous observation and (a).

\end{proof}Finally, note that

\begin{equation}
\text{$E_\beta$ is extensional} \iff \forall m,n \: [\forall k (k E_\beta m \; \leftrightarrow \; k E_\beta n) \; \to \; m=n ].
\end{equation}

Hence this is arithmetical. And we have already seen that coding a well-founded relation over $\Nat$ is $\Pi^1_1$.

Now we know the complexity of all parts of ($*$) and can put everything together.

\begin{theorem}\label{thm-l-reals}The set $L \cap \Baire$ is $\Sigma^1_2$.

\end{theorem}In a similar way we can show that the relation $\alpha <_L \beta$ over $(L\cap\Baire)^2$ is $\Sigma^1_2$ (using that $<_L$ is $\Delta_1$-definable).

\begin{framed}
\textbf{Exercise}\\
Recall that given $\alpha \in \Baire, n \in \Nat$, $(\alpha)_n$ denotes the $n$-th column of $\alpha$.

Show that the following relation $R$ over $(L\cap\Baire)^2$ is $\Sigma^1_2$.

\begin{equation}
(\alpha, \beta) \in R :\iff \{(\alpha)_n\colon n \in \Nat \} = \{\gamma \colon \gamma <_L \beta \}
\end{equation}

In other words, $\alpha$ codes the (countable) $<_L$-initial segment restricted to $\Baire$.
\end{framed}

If $\VL$, then the set is actually $\Delta^1_2$, since then

\begin{equation}
\alpha <_L \beta \iff \alpha \neq \beta \: \wedge \: \neg(\beta <_L \alpha).
\end{equation}

Finally, since $\VL$ implies $\CH$, we can use Proposition~\ref{prop-non-meas} to show the existence of non-measurable sets under $\VL$.

\begin{corollary}\label{cor-l-reals}If $\VL$, then there exists a $\Delta^1_2$ set that is not Lebesgue-measurable and does not have the Baire property.

\end{corollary}\subsubsection{An uncountable $\Pi^1_1$ set without a perfect subset}

We now show that under the assumption $\VL$, the \textbf{perfect set property} fails at level $\Pi^1_1$.

We start with constructing an example at the $\Sigma^1_2$ level.

Recall that if $\alpha \in \Baire$ codes a well-ordering on $\Nat$, then
\begin{equation*}
\Norm{\alpha} = \text{ order type of well-ordering coded by $\alpha$}.
\end{equation*}

\begin{proposition}\label{prop-sigma12-perfect}If $\VL$, there exists an uncountable $\Sigma^1_2$ set in $\Baire$ without a perfect subset.

\end{proposition}\begin{proof}Let $A \subseteq \Baire$ be given by
\begin{equation*}
x \in A \iff x \in \WOrd \, \wedge \, \forall y <_L x \,  ( \parallel y \parallel \ne \parallel x \parallel).
\end{equation*}

In other words, $A$ collects the $<_L$-least code for every ordinal $< \omega_1$.

Clearly $A$ is uncountable, since it has a representative for every countable ordinal and hence of cardinality $\omega_1$.

Moreover, $A$ is $\Sigma^1_2$: Let $R$ be the $\Sigma^1_2$-relation of the exercise above. Then
\begin{equation*}
x \in A \iff x \in \WOrd \, \wedge \, \exists z \;(R(z,x) \, \wedge \; \forall n \,  ( \Norm {(z)_n} \neq \Norm{x}).
\end{equation*}

The relation $\Norm{(z)_n} \neq  \Norm{x}$ $\Pi^1_1$, hence $A$ is $\Sigma^1_2$.

Finally, we see that \textbf{$A$ does not have an uncountable $\bSigma^1_1$ subset} (hence, since all perfect sets are closed, no perfect subset): By $\bSigma^1_1$-boundedness (Theorem~\ref{thm-sigma11-bounding}), for any $\bSigma^1_1$ subset $X \subseteq A$ the set $\{ \Norm{x} \colon x \in X\}$ bounded by an ordinal $\gamma < \omega_1$, hence countable.

\end{proof}It is possible to get this example down to $\Pi^1_1$ using the powerful technique of \textbf{uniformization}.

\begin{definition}\label{def-uniformization}Suppose $A \subseteq \Baire \times \Baire$. We say $A^* \subseteq A$ \textbf{uniformizes} $A$ if

\begin{equation}
\forall x  \; [ \exists y \; A(x,y) \to  \exists ! y \; A^*(x,y)]
\end{equation}

A pointclass $\Gamma$ has the \textbf{uniformization property} if
\begin{equation*}
A \subseteq \Baire  \times \Baire \, \wedge \, A \in \Gamma \quad \Rightarrow \quad \exists A^* \in  \Gamma \; (A^*  \text{ uniformizes } A).
\end{equation*}
\end{definition}\begin{theorem}[Kondo]\label{thm-kondo}$\bPi^1_1$ has the uniformization property.

\end{theorem}\begin{theorem}\label{thm-prefect-set-l}If $\VL$, then there exists an uncountable $\bPi^1_1$ set without a perfect subset.

\end{theorem}\begin{proof}Let $A$ be the $\Sigma^1_2$ set from the proof of Proposition~\ref{prop-sigma12-perfect}. $A \subseteq \Baire$ is the projection of a $\Pi^1_1$ set $B \subseteq \Baire \times \Baire$. If we apply uniformization to $B$, we obtain a uniformizing set $B^*$ whose projection is still $A$.

$B^*$ is uncountable, but does not contain a perfect subset: If $P \subset B^*$ were such a subset, then $P$ would be (the graph of) a function and uncountable, and the projection $\exists^{\Baire} \; P$ would be an uncountable  $\bSigma^1_1$ subset of $A$, contradiction.

\end{proof}

\include{descriptive_set_theory-shoenfield}

\section{Reference}

\include{descriptive_set_theory-bibliography}




\bibliography{main.bib}

\end{document}
