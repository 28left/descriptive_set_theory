\subsection{Polish Spaces}

The proofs in the \href{/perfect-subsets-r}{introduction section} are quite general, that is, they make little use of specific properties of $\Real$. If we scan the arguments carefully, we see that we can replace $\Real$ by any metric space that is \textbf{complete and contains a countable basis of the topology}.

\subsubsection{Review of some concepts from topology}

\paragraph{Basis}

Let $(X, \mathcal{O})$ be a topological space. A family $\mathcal{B} \subseteq \mathcal{O}$ of subsets if $X$ is a \textbf{basis} for the topology if every open set from $\mathcal{O}$ is the union of elements of $\mathcal{B}$. For example, the open intervals with rational endpoints form a basis of the standard topology of $\Real$. A family $\mathcal{S} \subseteq \mathcal{O}$ is a \textbf{subbasis} if the set of finite intersections of sets in $\mathcal{S}$ is a basis for the topology.

Finally, if $\mathcal{S}$ is any family of subsets of $X$, the \textbf{topology generated by $\mathcal{S}$} is the smallest topology on $X$ containing $\mathcal{S}$. It consists of all unions of finite intersections of sets in $\mathcal{S} \cup \{X,\emptyset\}$.

\paragraph{Density}

A set $D \subset X$ is \textbf{dense} if for any open $U \neq \emptyset$ there exists $z \in D \cap U$. If a topological space $(X, \mathcal{O})$ has a countable dense subset, the space is called \textbf{separable}.

\paragraph{Products}

If $(X_i)_{i \in I}$ is a family of topological spaces, one defines the \textbf{product topology} on $\Pi_{i \in I} X_i$ to be the topology generated by the sets $\pi_i^{ -1}(U)$, where $i \in I$, $U \subseteq X_i$ is open, and $\pi_i: \Pi_{i \in I} X_i \to X_i$ is the $i$th projection.


\bigskip
\centerline{\rule{13cm}{0.4pt}}
\bigskip

Now suppose $(X,d)$ is a metric space. With each point $x \in X$ and every $\eps > 0$ we associate an \textbf{$\eps$-neighborhood} or \textbf{$\eps$-ball}

\begin{equation}
U_\eps(x) = \{y \in X \colon d(x,y)<\eps\}.
\end{equation}

The topology generated by the $\eps$-neighborhoods is called the \textit{topology of the metric space} $(X,d)$. If this topology agrees with a given topology $\mathcal{O}$ on $X$, we say the metric $d$ is \textbf{compatible} with the topology $\mathcal{O}$. If for a topological space $(X, \mathcal{O})$ there exists a compatible metric, $(X, \mathcal{O})$ is called \textbf{metrizable}.

If a topological space $(X,\mathcal{O})$ is separable and metrizable, then the balls with center in a countable dense subset $D$ and rational radius form a \textit{countable base of the topology}.

\subsubsection{Polish spaces -- the basics}

\begin{definition}\label{def-polish}A \textbf{Polish space} is a separable topological space $X$ for which exists a compatible metric $d$ such that $(X,d)$ is a complete metric space.

\end{definition}There may be many different compatible metrics that make $X$ complete. If $X$ is already given as a complete metric space with countable dense subset, then we call $X$ a \textit{Polish metric space}.

The standard example is, of course, $\Real$, the set of real numbers. One can obtain other Polish spaces using the following basic observations. (We leave the proof as an exercise.)

\begin{proposition}\label{properties-polish}\begin{enumerate}
\item A closed subset of a Polish space is Polish.
\item The product of a countable (in particular, finite) sequence of Polish spaces is Polish.
\item Any topological space homeomorphic to a Polish space is Polish.
\end{enumerate}

\end{proposition}We conclude that $\Real^n$, $\C$, $\C^n$, the unit interval $[0,1]$, the unit circle $\Ci = \{z \in \C \colon |z| = 1\}$, and the infinite dimensional spaces $\Real^\Nat$ and $[0,1]^\Nat$ (the \textit{Hilbert cube}) are Polish spaces.

Any countable set with the \textbf{discrete topology} is Polish, by means of the \textbf{discrete metric} $d(x,y) = 1 \: \Leftrightarrow \: x \neq y$.

Some subsets of Polish spaces are Polish but not closed.

\begin{framed}
\textbf{Exercise}\\
By choosing a suitable metric, show that $(0,1)$, the open unit interval, is a Polish space.
\end{framed}

We will later characterize all subsets of Polish spaces that are Polish themselves.

\subsubsection{Product spaces}\label{polish-product-spaces}

In a certain sense, the most important Polish spaces are of the form $A^\Nat$, where $A$ is a countable set carrying the discrete topology. The standard cases are

\begin{quote}
$\Cant$, the Cantor space $\qquad$ and $\qquad$ $\Baire$, the Baire space.
\end{quote}

We will, for now, denote elements from $A^\Nat$ by lower case greek letters from the beginning of the alphabet. The $n$-th term of $\alpha$ we denote by either $\alpha(n)$ or $\alpha_n$, whichever is more convenient.

We endow $A$ with the discrete topology.
The product topology on these spaces has a convenient characterization. Given a set $A$, let $\Str[A]$ be the sets of all finite sequences over $A$.
Given $\sigma, \tau \in A^{<\Nat}$, we write $\sigma \Sleq \tau$ to indicate that $\sigma$ is an initial segment of $\tau$. $\Sle$ means the initial segment is proper. This notation extends naturally to hold between elements of $\Str[A]$ and $A^\Nat$, $\sigma \Sle \alpha$ meaning that $\sigma$ is a finite initial segment of $\alpha$.

A basis for the product topology on $A^\Nat$ is given by the \textbf{cylinder sets}

\begin{equation}
\Cyl{\sigma} = \{\alpha \in A^\Nat \colon \sigma \Sle \alpha \},
\end{equation}

that is, the set of all infinite sequences extending $\sigma$. The complement of a cylinder is a union of cylinders and hence open. Therefore, each set $\Cyl{\sigma}$ is clopen.

A compatible metric is given by

\begin{equation}
d(\alpha,\beta) = \begin{cases}
    2^{-N} & \text{ where } N \text{ is least such that $\alpha_N \neq \beta_N$ }\\
    0 & \text{ if $\alpha = \beta$}.
\end{cases}
\end{equation}

The representation of the topology via cylinders (which are characterized by finite objects) allows for a combinatorial treatment of many questions and will be essential later on.

\begin{proposition}[Topological properties of ]\label{prop-topological-product}\textbf{$A^\Nat$}\\
Let $A$ be a countable set, equipped with the discrete topology. Suppose $A^\Nat$ is equipped with the product topology. Then the following hold.

\begin{enumerate}
\item $A^\Nat$ is Polish.
\item $A^\Nat$ is \textit{zero-dimensional}, i.e. it has a basis of \textit{clopen} sets.
\item $A^\Nat$ is compact if and only if $A$ is finite.
\end{enumerate}

\end{proposition}Via the mapping

\begin{equation}
\alpha \mapsto \sum_{i = 0}^\infty \frac{2\alpha_i}{3^{i+1}},
\end{equation}

$\Cant$ is homeomorphic to the middle-third Cantor set in $\Real$, whereas the \textbf{continued fraction} mapping

\begin{equation}
\beta \mapsto \beta_0 + \cfrac{1}
        {\beta_1 + \cfrac{1}{
                \beta_2 + \cfrac{1}{
                        \beta_3 + \ldots}}}
\end{equation}

provides a homeomorphism between $\Z \times (\N\setminus\{0\})^\N$ and the irrational real numbers.

The universal role played by the discrete product spaces is manifested in the following results.

\begin{theorem}\label{thm-cantor-embedding}Every uncountable Polish space contains a homeomorphic embedding of Cantor space $\Cant$.

\end{theorem}The proof is similar to the proof of Theorem~(Cantor,~1884). Note that the proof actually constructs an embedding of $\Cant$. The continuity of the mapping is straightforward.

In a similar way we can adapt the proof of Cantor-Bendixson~Theorem to show that the \textit{perfect subset property} holds for closed subsets of Polish spaces.

\begin{theorem}[Cantor-Bendixson Theorem for Polish spaces]\label{thm-cantor_bendixson-polish}Every uncountable closed subset of a Polish space contains a perfect subset.

\end{theorem}Finally, we can characterize Polish spaces as continuous images of Baire space.

\begin{theorem}\label{thm-polish-cont-image-baire}Every Polish space $X$ is the continuous image of $\Baire$.

\end{theorem}\begin{proof}Let $d$ be a compatible metric on $X$, and let $D = \{x_i \colon i \in \Nat\}$ be a countable dense subset of $X$. Every point in $X$ is the limit of a sequence in $D$. Define a mapping $g:\Baire \to X$ by putting

\begin{equation}
\alpha = \alpha(0)\, \alpha(1)\, \alpha(2)\dots \mapsto \lim_n x_{\alpha(n)}.
\end{equation}

The problem is, of course, that the limit on the right hand side not necessarily exists. We have to proceed more carefully.
Given $\alpha \in \Nat$, we put $y^\alpha_0 = x_{\alpha(0)}$ and
define iteratively

\begin{equation}
y^\alpha_{n+1} = \begin{cases}
    x_{\alpha(n+1)} & \text{ if $d(y^\alpha_n,x_{\alpha(n+1)}) < 2^{-n}$}, \\
    y^\alpha_n & \text{ otherwise }.
\end{cases}
\end{equation}

The resulting sequence $(y^\alpha_n)$ is clearly Cauchy in $X$, and hence converges to some point $y^\alpha \in X$, by completeness. We define

\begin{equation}
f(\alpha) = y^\alpha.
\end{equation}

$f$ is continuous, since if $\alpha$ and $\beta$ agree up to length $N$ (that is, their distance is at most $2^{ -N}$ with respect to the above metric), then the sequences $(y^\alpha_n)$ and $(y^\beta_n)$ will agree up to index $N$, and all further terms are within $2^{ -N}$ of $y^\alpha_N$ and $y^\beta_N$, respectively.

Finally, since $D$ is dense in $X$, $f$ is a surjection.

\end{proof}