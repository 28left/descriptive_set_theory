% Created with jtex v.1.0.20
\documentclass{article}
\PassOptionsToPackage{short, nodayofweek}{datetime}


% Start Curvenote Definitions

% Pass Options Section
% base
\PassOptionsToPackage{normalem}{ulem}
\PassOptionsToPackage{utf8}{inputenc}

% template
\PassOptionsToPackage{framemethod=TikZ}{mdframed}
\PassOptionsToPackage{x11names, svgnames}{xcolor}

%%% PACKAGES

% base
\usepackage{inputenc}
\usepackage{url}
\usepackage{graphicx}
\usepackage{adjustbox}
\usepackage{amssymb}
\usepackage{amsfonts}
\usepackage{amsmath}
\usepackage{enumitem}
\usepackage{nicefrac}
\usepackage{booktabs}
\usepackage{microtype}
\usepackage{hyperref}
\usepackage{ulem}
\usepackage{enumitem}
\usepackage{float}
\usepackage{datetime}
\usepackage{xkeyval}
\usepackage{framed}
\usepackage{doi}

% template
\usepackage{natbib}
\usepackage{fancyvrb}
\usepackage{mdframed}
\usepackage{xcolor}

%%%


%%%% Setup Section

% base
\graphicspath{{.}}
% template
\sloppy
\newenvironment{aside}{\begin{framed}}{\end{framed}}
\newmdenv[linewidth=2pt,linecolor=CornflowerBlue,topline=false,bottomline=false,rightline=false,leftline=true,skipabove=20,skipbelow=20,leftmargin=20,rightmargin=20]{callout}
\newfloat{code}{thp}{loc}
\floatname{code}{Program}
\raggedbottom
\bibliographystyle{abbrvnat}
\setcitestyle{authoryear,open={(},close={)},semicolon,aysep={,}}

% End Curvenote Definitions


%%%%%%%%%%%%%%%%%%%%%%%%%%%%%%%%%%%%%%%%%%%%%%%%%%
%%%%%%%%%%%%%%%%%%%%  imports  %%%%%%%%%%%%%%%%%%%
\usepackage{amsthm}
%%%%%%%%%%%%%%%%%  math commands  %%%%%%%%%%%%%%%%
\newcommand{\R}{\mathbb{R}}
\newcommand{\Op}[1]{\operatorname{#1}}
\newcommand{\Ord}{\operatorname{Ord}}
%%%%%%%%%%%%%%%%%%%%%%%%%%%%%%%%%%%%%%%%%%%%%%%%%%
%%%%%%%%%%%%%%%%%%%%%%%%%%%%%%%%%%%%%%%%%%%%%%%%%%
%%%%%%%%%%%%%%%%%%%%  theorem  %%%%%%%%%%%%%%%%%%%
\newtheorem{theorem}{Theorem}[section]
\newtheorem{corollary}{Corollary}[theorem]
\newtheorem{lemma}[theorem]{Lemma}
\newtheorem{proposition}{Proposition}[section]
\newtheorem{definition}{Definition}[section]
\newtheorem{example}{Example}[section]
\newtheorem{remark}{Remark}[section]
\newtheorem{axiom}{Axiom}[section]
\newtheorem{conjecture}{Conjecture}[section]
\newtheorem{observation}{Observation}[section]
%%%%%%%%%%%%%%%%%%%%%%%%%%%%%%%%%%%%%%%%%%%%%%%%%%




% colors for hyperlinks
\hypersetup{colorlinks=true, allcolors=blue}
\hypersetup{
pdftitle={\@title},
pdfsubject={},
pdfauthor={\@author},
pdfkeywords={},
addtopdfcreator={Written in Curvenote}
}

\usepackage{curvenote}

\title{Ordinals}

\newdate{articleDate}{12}{1}{2025}
\date{\displaydate{articleDate}}

\author{}

\begin{document}

\maketitle

It will be important for us to extend the usual counting process beyond the natural numbers. To give an example, let us return for a moment to perfect subsets of the reals. To show that every uncountable closed subset of $\mathbb{R}$ contains a perfect subset, we considered the \textit{condensation points} of the set. There is another, more gradual way, to arrive at a perfect subset. When Cantor studied convergence of Fourier series, he introduced the \textbf{derivative} of a set:
\begin{equation}

A' = \{ x \in A \colon x \text{ is a limit point of } A\}
\end{equation}
We can iterate the derivative and consider $A^\prime, A^{\prime\prime}, A^{\prime\prime\prime}, \dots$. This yields a descending sequence of sets
\begin{equation}

A^\prime \supseteq  A^{\prime\prime} \supseteq A^{\prime\prime\prime} \supseteq \dots \supseteq A^{(n)} \supseteq \dots
\end{equation}

As the sequence is nested, we can take a ``limit'':
\begin{equation}

A^{\infty} = \bigcap_n A^{(n)}
\end{equation}
But the process does not necessarily stop here. $A^\infty$ may have isolated points again, so that $A^\infty \supsetneq (A^\infty)^\prime$.

Let us introduce $\omega$ as a new number to be used in place of $\infty$ above. We can continue the counting process:
\begin{equation}

1,2,3, \dots, \omega, \omega+1, \omega+2, \dots, \omega + \omega, \omega + \omega +1, \dots, \omega + \omega + \omega, \dots, \omega\cdot \omega, \dots
\end{equation}

We can then define, for example,
$A^{\omega+1} := (A^{\omega})'$. As intuitively clear from above, the new transfinite numbers come with a natural ordering, so we can also put $A^{\omega+\omega} := \bigcap_{\alpha < \omega+\omega} A^{\alpha}$

Another way to count into the transfinite is to reorder the natural numbers and first enumerate all powers of two, followed by all powers of three and so on:
\begin{gather*}
	1, 2, 4, 8, \ldots 3, 9, 27, \ldots 5, 25, 125, \ldots
\end{gather*}
This still leaves an infinite reservoir of numbers like $0, 6, 10, \dots$.

A number of questions arises:

\begin{itemize}
\item Can this process continued indefinitely?
\item Is there a \textit{unifying} principle behind the various ways to count into the transfinite?
\item Can we define operations like $+$ and $\cdot$ on these infinite numbers independent of the way we represent these numbers, and without leading to contradictions?
\end{itemize}

These questions can be addressed by developing the theory of \textit{ordinal numbers} via the concept of a \textit{well-founded order}.

\section{Orders and well-orders}

\begin{definition}A (\textbf{reflexive} or \textbf{non-strict}) \textbf{partial order} on a set $A$ is a binary relation $\leq$ on $A$ such that for all $a,b,c \in A$,

\begin{itemize}
\item $a\leq a$   (reflexive)


\item $a \leq b \; \wedge \; b \leq a \quad \Rightarrow \quad a = b$   (anti-symmetric)


\item $a \leq b \; \wedge \; b \leq c \quad \Rightarrow \quad a \leq c$   (transitive).
\end{itemize}

A \textbf{linear} (or \textbf{total}) \textbf{order} additionally satisfies for all $a,b \in A$,

\begin{itemize}
\item $a \leq b \; \vee \; b \leq a$    (connected).
\end{itemize}

\end{definition}With any reflexive partial order $\leq$ we can associate an \textit{irreflexive} one by letting
\begin{equation}
a < b \quad :\iff \quad a \leq b \; \wedge \; a \neq b.
\end{equation}
Likewise, we can obtain a reflexive order from an irreflexive one by defining
\begin{equation}
a \leq b \quad :\iff \quad a < b \; \vee \; a = b.
\end{equation}
In light of this, we will usually just speak of partial or linear orders, without further specifying whether it is reflexive or irreflexive.

\begin{example}\begin{itemize}
\item The usual orders on $\mathbb{Z}$, $\mathbb{Q}$, and $\mathbb{R}$ are linear orders.
\item The relation
\begin{equation}

f \leq g \quad : \iff \quad \forall x \in \mathbb{R} \;\; f(x) \leq g(x)
\end{equation}


is a partial order on real valued functions on $\mathbb{R}$ but not a linear order.
\item The subset relation $\subseteq$ is a partial order on the power set of any set $A$, but it is a linear order only when $A = \emptyset$.
\end{itemize}

\end{example}We can enumerate the natural numbers one after another, but for the other standard ordered number domains this is not possible: We cannot find a place to begin counting (as in the case of $\mathbb{Z}$) or there is no ``next bigger'' element (as in the case of $\mathbb{Q}$ or $\mathbb{R}$).

To enumerate these domains in the form $\{a_0, a_1,\ldots, a_n, a_{n+1}, \ldots \}$ we have to reorder them in a way that

\begin{itemize}
\item we can start with a smallest element,
\end{itemize}

and once we have a arrived at an element $a$,

\begin{itemize}
\item we know with which element we continue the enumeration (i.e. there is an \textit{immediate successor} to $a$),
\item we can continue the enumeration even if we have already enumerated infinitely many elements before $a$ (but elements of the domain still remain).
\end{itemize}

These requirements can be combined into a single property: every non-empty subset (i.e. the elements not enumerated yet) has a least element (to be enumerated next).

\begin{definition}A linear order $(A,<)$ is a \textbf{well-order} if
\begin{equation}

\forall Z \; (\emptyset \ne Z \subseteq A \; \Rightarrow \; \exists x \in Z \; \forall y \in Z \;  x \leq y)
\end{equation}

\end{definition}Orders themselves can be compared using \textit{embeddings}.

\begin{definition}An \textbf{embedding} of a partial order $(A,<_A)$ into another partial order $(B,<_B)$ is a mapping $f:A \to B$ such that for all $x,y \in A$
\begin{equation}

x <_A y \quad \iff \quad f(x) <_B f(y).
\end{equation}

Two orders are \textbf{isomorphic} if there exists a bijective embedding of one into the other.

\end{definition}Of course every order is isomorphic to itself (\textit{automorphic}) via the identity. But many orders allow automorphisms other than the identity (e.g. $\mathbb{Z}$ with $z \mapsto z+1$), or are isomorphic to a proper subset (for example, $\mathbb{R}$ and $(0,1)$). As we will see, well-orders are very rigid in this regard.

We start with a simple observation.

\begin{proposition}Let $(A,<)$ be a well-order and assume $f:A \to A$ is a self-embedding. Then for all $x \in A$, $x \leq f(x)$.

\end{proposition}\begin{proof}If the set $\{x \in A \colon f(x) < x\}$ is non-empty, it has a minimal element $z$. But since $f$ is increasing, this would imply $f(f(z)) < f(z)$, contradicting the minimality of $z$.

\end{proof}We immediately obtain

\begin{corollary}The only automorphism of a well-order is the identity.

\end{corollary}An \textbf{initial segment} of an order $(A,<)$ is given by all elements that are smaller than a given element $b$. We denote this initial segment by $A\mid_b$.

\begin{corollary}No well-order is isomorphic to an initial segment of itself.

\end{corollary}\begin{proof}Suppose $f: A \to A\mid_b$ is an isomorphism. Then $\operatorname{ran}(f) = A\mid_b$ and $f(x) < b$ for all $x \in A$. In particular, $f(b) < b$, contradicting Lemma

\end{proof}\section{Ordinal numbers}

\textit{Cantor} defined ordinal numbers (or ordinals) as \textbf{isomorphism classes of well-orders}. Later, \textit{von Neumann} suggested a system of representatives particularly suitable for set theoretic considerations. The idea is to define the order $<$ through the $\in$-relation on a set. For example,
\begin{equation}

\{ \emptyset, \{\emptyset\}, \{\emptyset, \{\emptyset\}\}\}
\end{equation}
represents a 3-element well-order.

We will impose some conditions on the sets we allow as ordinals. Given a set $A$, we use $\in_A$ to denote the $\in$-relation \textit{on} $A$:
\begin{equation}

\in_A = \{ (x,y) \colon x,y \in A \; \wedge \; x \in y \}
\end{equation}

\begin{definition}\label{def-transitive}A set $A$ is \textbf{transitive} if
\begin{equation*}
\tag{trans}
    \forall x \in A \; \; x \subseteq A
\end{equation*}
\end{definition}In other words, transitive sets cannot ``hide'' elements in subsets.

\begin{definition}\label{def-ordinal}A set $A$ is an \textbf{ordinal} if it is transitive and well-ordered by $\in_A$

\end{definition}It is customary to use \textit{lower case Greek letters} $\alpha, \beta, \gamma, \dots$ to denote ordinals.

If we exclude certain pathological sets from the beginning, we can further simplify this definition.

\begin{definition}A set $A$ is \textbf{well-founded} if every non-empty subset has a $\in$-minimal element:

\begin{equation}
\forall B \subseteq A \; (B \ne \emptyset \; \Rightarrow \; \exists y \in B \;  \forall z \in B \; \; z \not \in y)
\end{equation}

\end{definition}Sets which contain themselves ($A \in A$) are not well-founded$-$ $\{A\}$ would be a subset without a $\in$-minimal element. Similarly, well-founded sets cannot have cycles like $a \in b \in c \in a$.

\begin{proposition}\label{pro-ordinal-linear-order}Assume every set is well-founded. A set $\alpha$ is an \textbf{ordinal} if and only if it is transitive and linearly ordered by $\in_\alpha$.

\end{proposition}If we write out the formulas in full, we see Definition~\ref{def-ordinal} is much simpler than the original one. Most notably, in Definition~\ref{def-ordinal} we only use only \textbf{bounded quantifiers} (of the form $\forall y \in a$), whereas in the original form we have to quantify over arbitrary subsets of $a$. This is an important difference whose impact will become clear later on.

We can now develop the theory of ordinals based on this definition.

\begin{proposition}\label{pro-ordinal-elements-are-ordinals}Any element of an ordinal is an ordinal.

\end{proposition}\begin{proof}Any subset of a linear order is again a linear order under the induced order relation. It remains to show that $(b, \in_b)$ is transitive. Let $\alpha$ be an ordinal, and assume $b \in \alpha$. Let $x \in c \in b$. We claim $x \in b$. Since $\alpha$ is transitive, $b \subseteq \alpha$ and hence $c \in \alpha$. By transitivity of $\alpha$ again, $c \in \alpha$. Thus $x,b \in \alpha$, and since $\in_\alpha$ linearly orders $\alpha$, we must have
\begin{equation}

x \in b \; \vee \; x = b \; \vee \; b \in x.
\end{equation}
If $x = b$, we get $b \in c \in b$, contradicting well-foundedness. Similar for $b \in x$. Therefore, $x \in b$.

\end{proof}\section{The well-ordering of ordinals}

Proposition~\ref{pro-ordinal-elements-are-ordinals} suggests we can order ordinals by letting
\begin{equation}

\alpha < \beta \; :\iff \; \alpha \in \beta.
\end{equation}
By Proposition~\ref{pro-ordinal-elements-are-ordinals}, an ordinal then contains precisely the ordinals smaller than it:
\begin{equation}

\alpha = \{ \beta : \beta < \alpha \}.
\end{equation}

$\in$ defines a partial order on all ordinals:
As all sets are well-founded, irreflexivity holds, and since ordinals are transitive sets, $<$ is a transitive relation.

\begin{proposition}\label{pro-order-ordinals}For any ordinals $\alpha, \beta$,
\begin{equation}

\alpha < \beta \; \iff \; \alpha \subset \beta.
\end{equation}

\end{proposition}\begin{proof}The $\Rightarrow$ direction follows directly from the transitivity of ordinals.
For $\Leftarrow$, we show something more general, namely that any transitive proper subset of an ordinal is itself an ordinal and is an element of the superset ordinal:
\begin{equation}

\Op{trans}(a) \; \wedge \; a \subset \beta \quad \Rightarrow \quad \Op{Ord}(a) \; \wedge \; a \in \beta.
\end{equation}
If $a \subset \beta$, $a$ is linearly ordered by $\in$ (as a subset of $\beta$). Further, if $a$ is transitive, $a$ is an ordinal.

It remains to show $a \in \beta$. Since $a$ is a proper subset of $\beta$, by well-foundedness there exists a $\in$-minimal element of $\gamma \in \beta \setminus a$. We claim $a = \gamma$. By $\in$-minimality of $\gamma$, every element of $\gamma$ cannot be in $\beta\setminus a$ and therefore has to be in $a$. Hence $\gamma \subseteq a$. On the other hand, if $x \in a$, then, by assumption $x \in \beta$, and since $\in$ linearly orders $\beta$,
\begin{equation}

x \in \gamma \; \vee \; x = \gamma \; \vee \; \gamma \in x.
\end{equation}
The latter two are impossible due to $\gamma \notin a$. Hence $x \in \gamma$ and therefore $a \subseteq \gamma$, yielding $a =\gamma$.

\end{proof}\begin{framed}
\textbf{Theorem (well-ordering of ordinals)}\\
The ordinal numbers are well-ordered by $<$.
\end{framed}

\begin{framed}
\textbf{Hint}\\
Most properties follow directly from well-foundedness and the fact that ordinals are transitive as sets.

To show that ordinals are linearly ordered by $<$, look at the intersection of two ordinals and try to apply Proposition~\ref{pro-order-ordinals}.
\end{framed}

\begin{proof}We first show $<$ is a linear order. Irreflexivity follows from well-foundedness of $\in$. Transitivity of $<$ follow from the transitivity of ordinals as sets. To show
\begin{equation}

\alpha < \beta \; \vee \; \alpha = \beta \; \vee \; \beta < \alpha,
\end{equation}
observe that the intersection of two ordinals is an ordinal, the \textit{minimum} of the two ordinals. Let $\gamma = \alpha \cap \beta$. Then $\gamma \subseteq \alpha$, so by Proposition~\ref{pro-order-ordinals}, $\gamma \in \alpha$ or $\gamma = \alpha$ and similarly $\gamma \in \beta$ or $\gamma = \beta$. But in the case $\gamma \in \alpha, \gamma \in \beta$ we would have $\gamma \in \alpha \cap \beta = \gamma$, contradicting well-foundedness.

Finally, if $A$ is a non-empty set of ordinals, the well-ordering condition on $<$ spells out as
\begin{equation}

\exists \alpha \in A \forall \beta \in A \; & \beta \notin \alpha.
\end{equation}
But this holds since we assume all sets are well-founded.

\end{proof}\section{Basic properties of ordinals}

Using the results obtained so far. we can now deduce some basic facts about the structure of ordinals:

\begin{itemize}
\item $0 = \emptyset$ is the \textit{smallest ordinal}.


\item Every ordinal $\alpha$ has an \textbf{immediate successor} under the ordering $<$:
\begin{equation}
\alpha' =  \alpha+1 = \alpha \cup \{\alpha\}.
\end{equation}
Clearly $\alpha < \alpha+1$. If $\alpha < \beta$, then by Proposition~\ref{pro-order-ordinals}, $\alpha \subset \beta$ and $\alpha \in \beta$. Hence $\alpha+1 \subseteq \beta$ and therefore $\alpha+1 \leq \beta$.


\item The \textit{finite ordinals} are exactly the \textbf{natural numbers} (``set theoretic version''):
\end{itemize}

\begin{equation}
0 = \emptyset, \quad 1 = 0 + 1 = \emptyset \cup \{\emptyset\} = \{\emptyset\}, \quad 2 = 1+1 = \{\emptyset, \{\emptyset\} \}, \dots
\end{equation}

\begin{itemize}
\item The set of all natural numbers is transitive and well-ordered by $\in$ and thus itself an ordinal, the first infinite ordinal $\omega$:
\begin{equation}

  \omega = \{\emptyset, \{\emptyset\}, \{\emptyset, \{\emptyset\}\}, \dots \}
\end{equation}


\item $\omega$ is also the first instance of a \textit{limit ordinal}: A \textbf{successor ordinal} is any ordinal of the form $\alpha+1$. Any ordinal $\lambda$ that is not a successor is called a \textbf{limit ordinal}. Being limit is equivalent to the following property:
\begin{equation}

\lambda \neq 0 \: \wedge \: \forall \alpha < \lambda \; (\alpha+1 < \lambda).
\end{equation}
This shows immediately that $\omega$ is limit.


\item More generally, if $A$ is a set of ordinals, $\sup A = \bigcup_{\alpha \in A} \alpha$ is an ordinal and is the least upper bound for $A$.


\item The first limit ordinal $\omega$ is followed by a number of successor ordinals as well as their limits as limit ordinals:
\end{itemize}
\begin{gather*}
\omega, \omega+1, \omega+2, \ldots \omega+\omega, \omega+\omega+1, \omega + \omega+2, \ldots \omega+\omega+ \omega, \qquad \quad \\ \omega + \omega+ \omega+1, \omega + \omega+ \omega+2, \ldots  \omega \cdot \omega,  \omega \cdot \omega +1,\ldots, \omega^{\omega} \ldots \omega^{\omega^{\omega}} \ldots
\end{gather*}
All of the ordinals listed here are still countable (as sets). The supremum of the set of all countable ordinals is denoted by $\omega_1$, the \textbf{first uncountable ordinal}. After $\omega_1$, we have again successors, limits, and so on.

\section{Metamathematical issues}

Is there a \textit{set} $\Ord$ of \textit{all} ordinals? If so, it would be well-ordered by $\in$ and also transitive (since, by Proposition~\ref{pro-ordinal-elements-are-ordinals}, every element of an ordinal is an ordinal) and therefore an ordinal. But then $\Ord+1$ would be an ordinal not contained in $\Ord$ (by well-foundedness), contradiction.

This is know as the \textbf{Anomaly of Burali-Forti}. It tells us that somehow the collection of all ordinals is \textit{too big} to form a set. It also warns us that if we handle the intuitive concept of a set too carelessly, it might lead to contradictions and inconsistencies.

Later on we will develop an axiomatic approach to sets which aims to exclude antinomies like this. In this framework, we will be able to formally show that $\Ord$ is not a set. It forms what we will call a \textbf{proper class}.

\section{Representing well-orders as ordinals}

We introduced ordinals with the goal to have a specific representation for any well-order.

\begin{framed}
\textbf{Theorem (representation theorem for well-orders)}\\
Any well-ordered set $(A,<)$ is order-isomorphic to a unique ordinal $\alpha$. The isomorphism is unique.
\end{framed}

The ordinal $\alpha$ is called the \textbf{order type} of $(A,<)$.
We will delay the proof of this theorem for a while, until we learn how to extend \textit{induction} and \textit{recursion} into the transfinite.
\end{document}
